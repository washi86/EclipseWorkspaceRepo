%\documentstyle[12pt]{report}
%\setlength{\parindent}{0mm}
%\setlength{\parskip}{14pt}
%\renewcommand{\baselinestretch}{1.5}
%\setlength{\topmargin}{0pt}
%\setlength{\headheight}{0pt}
%\setlength{\headsep}{0pt}
%\setlength{\footskip}{45pt}
%\setlength{\footheight}{0pt}
%\setlength{\textwidth}{465pt}
%\setlength{\textheight}{660pt}
%\setlength{\oddsidemargin}{5pt}
%\newcommand{\RR}{\mathrm{I\!R\!}}
%\newcommand{\FF}{\mathrm{I\!F\!}}
%\newcommand{\LL}{{\cal L}}
%\newcommand{\dx}{\displaystyle{\frac{\partial}{\partial x}}}
%\newcommand{\dy}{\displaystyle{\frac{\partial}{\partial y }}}
%\newcommand{\dz}{\displaystyle{\frac{\partial}{\partial y' }}}
%\newtheorem{defi}{Definition}[chapter]
%\newtheorem{theo}{Theorem}[chapter]
%\newtheorem{lem}{Lemma}[chapter]
%\begin{document}

\chapter{Noether Symmetries of \boldmath{$y''=f(x)y^n$} with
Applications to Non--Static Spherically Symmetric Perfect Fluid Solutions}
We study the Noether point symmetries and integrability
of the equation $y''=f(x)y^n$, $n\ne -1, 0, 1$. The
case $n=2$ is applied to shear--free spherically symmetric perfect fluid
solutions. As a result
we re-obtain the Kustaanheimo and Qvist 
class and other known solutions in a systematic manner without
resorting to {\em ad hoc} methods. Moreover, we show how one can construct 
new solutions as well. For the case $n=-1/3$, it is shown that all existing solutions 
arise as a result of the equation admitting a Noether point symmetry.
In addition, the Noether integral gives rise to a class of solutions
which contains the existing solutions. This work has been reported in
Wafo Soh and Mahomed (1999b)
\section{ Introduction}
In the study of solutions of Einstein's field equations, the equation
\begin{equation} y''=f(x)y^n  \label{ch3:i1} \end{equation} occurs on two
occasions. 

The first occurence of Eq. (\ref{ch3:i1})  is found in the theory of non--static
spherically symmetric perfect fluid solutions with no shear. In their
celebrated paper, Kustaanheimo and Qvist (1948)
(see also Kramer {\em et al} 1980)  showed that
non--static spherically symmetric perfect fluid solutions with no shear
are obtained by solving the second--order ordinary differential equation
(ode) \begin{equation} y''=f(x)y^2. \label{ch3:i2} \end{equation} In general,
Eq. (\ref{ch3:i2}) cannot be solved in closed form. But for some
specifications of $f(x)$ it can. The problem is then the following one:
how does one prescribe the form of $f(x)$ such that Eq. (\ref{ch3:i2}) is
solvable by quadratures? Kustaanheimo and Qvist (1948) resorted to
what is nowadays known as Lie symmetry analysis
(Ovsiannikov 1982, Ibragimov 1985, 1999, Olver 1986, Bluman and Kumei 1989,
Stephani 1989, Hill 1992, see also Chapter 1)
to provide an answer to this question. After the pioneering work of Sophus
Lie (cf. Lie and Scheffers 1891), it is  known that the more symmetries an equation has the
easier its integration. Kustaanheimo and Qvist (1948) found that Eq. (\ref{ch3:i2})
admits a point symmetry if and only if 
\begin{equation} \frac{d^3}{dx^3}\left (
\alpha f^{-2/5}+\beta f^{-2/5}\int f^{2/5}dx \right ) =(\gamma x+\delta)f, 
\label{ch3:i3} \end{equation} 
where $\alpha,\;\beta,\; \gamma$ and
$\delta$ are constants. As it can immediately be seen, Eq. (\ref{ch3:i3}) is a
third--order, nonlinear integro--differential equation for which a
closed form solution exists only in special cases.
Kustaanheimo and Qvist (1948)
found the particular solution 
\begin{equation} f=(ax^2+bx+c)^{-5/2},
\end{equation} 
where $a,\;b$ and $c$ are constants, under the assumption that
$\beta=\gamma=\delta=0$. Since this is just a particular solution
to Eq. (\ref{ch3:i3}), there might still be other possibilities of $f(x)$ for which
Eq. (\ref{ch3:i2}) possesses point symmetries. Recently, Leach {\em et al}
(1992), Herlt and Stephani (1992) and  Stephani and Wolf (1996)
further analyzed  Eq.
(\ref{ch3:i3}). 

The second occurence of Eq. (\ref{ch3:i1}) with $n=-1/3$ is in the study of non--static spherically
symmetric perfect fluid solutions with non-zero shear (geodesic flow). 

Since the direct Lie method does not lead to a
complete classification of Eq. (\ref{ch3:i1}) as mentioned earlier, our
concern in this chapter is to study Noether point symmetries of Eq. (\ref{ch3:i1}). It
is worth mentioning that Noether symmetries  are particular Lie symmetries
which have the interesting property of being associated with conservation
laws or first integrals via  the celebrated Noether theorem
(Noether 1918). Furthermore, a scalar
second--order ode  with only one Noether point symmetry is integrable by
quadratures. Thus, the advantage of studying Noether point symmetries 
corresponding to a Lagrangian of Eq. (\ref{ch3:i1}) is that it is a more
fundamental approach for integrability in that a Noether symmetry directly gives 
via the Noether theorem an explicit integral which itself is invariant. Some previously known
integrable cases of Eq. (\ref{ch3:i1}) are recovered and new integrable cases are uncovered. It is important
to stress that the Lie point symmetry approach
(used in Leach {\em et al } 1992, Herlt and Stephani 1992,
Stephani and Wolf 1996)
and the Noether point symmetry approach used here are complementary
in the sense that the Noether approach picks 
out the one symmetry cases that are solvable whereas the Lie approach is more general
and does not guarantee integrability for the one symmetry cases in a straightforward
manner. Thus it becomes natural and vital, if one can
find a Lagrangian, to employ the Noether approach which we do here. 

 
The outline of this chapter is as follows. Section 3.2 deals with the
classification of Eq. (\ref{ch3:i1}) with respect to the Noether point symmetries of the natural
Lagrangian of Eq. (\ref{ch3:i1}),
i.e., we find all the forms of $f(x)$ for which a Noether point symmetry
exists. For the $n=2$ case, we recover the Kustaanheimo and Qvist, and Stephani solutions.
Notwithstanding, we obtain new solutions. Also, for the case $n=-1/3$, we provide a unified
approach to obtaining the existing solutions. In Section 3.3 we give
applications to non--static spherically symmetric perfect fluid solutions.

To place this chapter in its proper perspective, we would like
to refer the reader to  recent works by
Wyman (1976), Leach {\em et al} (1992), Stephani and Herlt (1992),
Stephani and Wolf (1996), Maharaj {\em et al} (1996),
Wafo Soh and Mahomed (1999c).
In  Leach {\em et al} (1992), Stephani and Herlt (1992) and
Stephani and Wolf (1996),
the direct Lie method is applied , the Painlev\'e analysis is
utilised in  Wyman (1976) and the method of preliminary group classification
is employed in Wafo Soh and Mahomed (1999c). The subject of
Maharaj {\em et al} (1996) is the study of first integrals of
Eq. (\ref{ch3:i2}) using an {\em ad hoc} method which is later justified
in the same paper by the Lie method.

\section{ Noether symmetries  and integration of \\
$\boldmath{y''=f(x)y^n}$} 

Let us begin with some important facts about Noether's first integral
(Chapter 1, Section 1.5). 
\begin{theo} \label{ch3:t2} \begin{em} The first integral $I$
associated with the Noether point symmetry $X$ satisfies 
\begin{equation}
X^{[1]}I=0,  \label{ch3:n7}
\end{equation}
i.e. $X$ is a point symmetry of the first
integral $I$.
\end{em} \end{theo}

{\bf Proof.} See for instance Lutzky (1979), Sarlet and Cantrijn (1981) and
Kara {\em et al} (1994). The
references Kara {\em et al} (1994) and Ibragimov {\em et al} (1998)
contain a more general proof.

\begin{theo}
\label{ch3:t3} \begin{em} A scalar second--order ordinary differential equation
admitting a single Noether point symmetry is solvable by quadratures. 
\end{em} \end{theo} 

{\bf Proof.} Use Theorem \ref{ch3:t2}.

{\bf Illustrative example.} Consider the Emden--Fowler equation
\begin{equation}
\label{ch3:x1}
y''+\frac{2}{x}y'=3y^5 .
\end{equation}
Its natural Lagrangian is
\begin{equation}
\label{ch3:x2}
L=\frac{1}{2}x^2y'^2+\frac{1}{2}x^2y^6.
\end{equation}
Using the determining equations for Noether symmetries
(cf. Chapter 1, Section 1.5), we find
\begin{equation}
\label{ch3:x3}
X=2x\dx-y\dy,\;\;A=0.
\end{equation}
Applying Noether's theorem (Chapter 1, Section 1.5), we determine
\begin{equation}
\label{ch3:x4}
I=-x^3y'^2-x^2yy'+x^3y^6.
\end{equation}
Hence the reduced equation can be written as
\begin{equation}
\label{ch3:x5}
x^3y'^2+x^2yy'-x^3y^6=C_1,
\end{equation}
where $C_1$ is an arbitrary constant. According to Theorem \ref{ch3:t2}, $X$
is also a symmetry of the reduced equation. In order to solve Eq.
(\ref{ch3:x5}) we use the invariant of $X$ as the new dependent variable. This
invariant is found from the characteristic equation associated with $X$,
viz.
\[\frac{dx}{2x}=\frac{dy}{-y}\;\cdot\]
We obtain an invariant $u=yx^{1/2}$.
Eq. (\ref{ch3:x5}) in terms of $u$ reads
\begin{equation} \label{ch3:x6}
x^2u'^2-\frac{1}{4}u^2-u^6=C_1.
\end{equation}
Eq. (\ref{ch3:x6}) is obviously variables separable
(cf., e.g., handbooks Kamke 1971, Polyanin and Zaitsev 1995 for another
approach to (\ref{ch3:x1})).
Here it should be pointed out that a single
Noether point symmetry which turns out to be the only Lie point symmetry
of the equation, leads to quadrature. In the usual Lie approach, a single 
point symmetry of a second--order ode in general does not result
in quadrature.

Henceforth we consider Eq. (\ref{ch3:i1})
with $n\ne -1, 0, 1$.
Its natural Lagrangian is
\begin{equation}
L=\frac{y'^2}{2}+f(x)\frac{y^{n+1}}{n+1}, \; n\ne -1 . \label{ch3:n8}
\end{equation}
By using the determining equations for Noether point symmetries
(Chapter 1) and separating with respect to powers of $y'$,
we get the system

\begin{eqnarray} \xi_y & =& 0, \label{ch3:n9}\\
\xi_x-2\eta_y &=& 0,
\label{ch3:n10}\\
\eta_x &= & A_y , \label{ch3:n11} \\ y^{n+1}\left (\xi_x f+\xi
f'\right )+(n+1)y^n f\eta & =& (n+1)A_x . \label{ch3:n12}
\end{eqnarray}

After simple calculations, the system Eqs. (\ref{ch3:n9})--(\ref{ch3:n12}) reduces to
\begin{eqnarray}
\xi &=& a(x), \label{ch3:n13}\\
\eta &=& \frac{a'}{2}y+b(x), \label{ch3:n14}\\
A &=& \frac{a''}{4}y^2+b'y+B(x), \label{ch3:n15}\\
y^{n+1}(af'+fa')+(n+1)y^n f\left (\frac{a'}{2}y+b \right) &=&
(n+1)\left (\frac{a'''}{4}y^2+b''y+B'\right ).\;\;
\label{ch3:n16}
\end{eqnarray}
Eq. (\ref{ch3:n16}) prompts the consideration of the following cases. 

{\bf Case I. \boldmath{$\; n\ne -1,\;0,\;1,\; 2.$}}\\
We find that
\begin{eqnarray}
a & = & \alpha x^2+2\beta x +\gamma ,\quad b=0 ,\label{ch3:n17}\\
\eta &= & (\alpha x+\beta)y, \label{ch3:n18}\\
f &=& f_0 (\alpha x^2+2\beta x+\gamma)^{-\frac{n+3}{2}}, \label{ch3:n19} \\ A
&=& \frac{\alpha}{2}y^2+\mbox{ const.},\label{ch3:n20} \end{eqnarray}
where $\alpha,\;\beta$, $\gamma$ and $f_0$ are constants.
Using Theorems \ref{ch3:t2} and
\ref{ch3:t3}, we find that the solution to Eq. (\ref{ch3:i1}) is given by
\begin{equation} y=u(\alpha x^2+2\beta x+\gamma)^{1/2}, \label{ch3:n21}
\end{equation} where $u$ is defined by \begin{equation} \int
\frac{du}{\left [ (\beta^2-\alpha \gamma)u^2+2f_0u^{n+1}/(n+1)+C_1 \right
]^{1/2}} =\int \frac{dx}{\alpha x^2+2\beta x+\gamma}  +C_2, \label{ch3:n22}
\end{equation} where $C_1$ and $C_2$ are arbitrary constants of
integration (this solution is also given, e.g. in
the handbooks Kamke 1971, Polyanin and Zaitsev 1995).
The implications of this solution in the physical case $n=-1/3$ will
be discussed in the next section.

{\bf Case II. \boldmath{$\;n=2$.}}\\
Eq. (\ref{ch3:n16}) splits into the following 
\begin{eqnarray}
af'+\frac{5}{2}a'f &=& 0, \label{ch3:n23} \\
4bf &=& a''', \label{ch3:n24}\\
b &=& \alpha x+\beta , \; B =  \mbox{const.}, \label{ch3:n25}
\end{eqnarray}
where $\alpha$ and $\beta$ are constants. The following subcases arise. 

{\bf Case II.1 \boldmath{$\;\alpha=0=\beta$}, i.e. \boldmath{$\;b=0.$} }\\
We  find \begin{eqnarray} a &=& rx^2+2sx+q, \label{ch3:n27} \\ f &=&
f_0(rx^2+2sx+q)^{-5/2}, \label{ch3:n28} \end{eqnarray} where $r,\;s,\;q$ and
$f_0$ are constants. This is the case which was investigated by Kustaanheimo
and Qvist (1948). Since there is a vast literature
(see Kramer {\em et al} 1980 and
references therein) on this case, we will not elaborate any further.

{\bf Case II.2  \boldmath{$\alpha\ne0 \mbox{ or } \beta\ne 0$}, i.e.
\boldmath{$b\ne 0$.}}\\ We  obtain \begin{eqnarray} a &=& a_0f^{-2/5},
\label{ch3:n29}\\ \eta &=& \frac{1}{2}ya'+b(x),\label{ch3:n30}\\ A &=&
\frac{1}{4}a''y^2+b'y+\mbox{const.}, \label{ch3:n31}\\
b''=0,\;\;4a^{-5/2}b &= & a''', \label{ch3:n32} \end{eqnarray} where
$a_0$ is an arbitrary constant. Since Noether symmetries form a Lie
algebra, without loss of generality we can set $a_0=1$ and then 
$a=f^{-2/5}$.
The  first integral of Eq. (\ref{ch3:i1}) associated with
the Noether point symmetry \begin{equation} X= a\dx+\left
(\frac{1}{2}a'y+b\right )\dy, \label{ch3:n38} \end{equation} is (Chapter 1)
\begin{equation}
I=-\frac{1}{2}ay'^2+\frac{1}{2}a'yy'+\frac{1}{3}a^{-3/2}y^3-\frac{1}{4}a''y^2 
+by'-b'y, \label{ch3:n39} \end{equation} provided Eqs.
(\ref{ch3:n32}) and $a=f^{-2/5}$ are satisfied. A basis of invariant of $X$ is
given by \begin{equation} \omega=a^{-1/2}y-\int ba^{-3/2}dx . \label{ch3:n40}
\end{equation} Taking $\omega$ as the new dependent variable, the first
integral Eq. (\ref{ch3:n39}) becomes
\begin{equation}
I=-\frac{1}{2}a^2\omega'^2+\frac{1}{3}\omega^3+D\omega^2+E\omega+F,
\label{ch3:n41}
\end{equation}
where
\begin{eqnarray}
D &=& \frac{1}{8}a'^2-\frac{1}{4}aa''+\int ba^{-3/2}dx, \label{ch3:n42} \\ E
&=& \frac{1}{4}a'^2\int ba^{-3/2}dx+\left (\int ba^{-3/2}dx\right )^2
-\frac{1}{2}aa''\int ba^{-3/2}dx \nonumber\\
 & & +\frac{1}{2}ba^{-1/2}a'-b'a^{1/2},
\label{ch3:n43}\\
F &=& \frac{1}{8}a'^2\left (\int ba^{-3/2}dx\right )^2
+\frac{1}{2}b^2a^{-1} +\frac{1}{2}a^{-1/2}ba'\int
ba^{-3/2}dx+\frac{1}{3}\left ( \int b a^{-3/2} dx \right )^3 \nonumber \\
& & -\frac{1}{4}aa''\left (\int ba^{-3/2}dx \right )^2-a^{1/2}b'\int
ba^{-3/2}dx. \label{ch3:n44} \end{eqnarray} 
Straightforward calculations reveal that
$E,\; D$ and $F$ are constants. Let us for instance show that $D$ is a
constant. Differentiate Eq. (\ref{ch3:n42}) once and obtain (\ref{ch3:n32}b). Similarly,
to prove that $E$ and $F$ are constants, take their derivative with respect
to $x$ and use Eqs. (\ref{ch3:n32}) to conclude. 


Now (\ref{ch3:n41}) can be written as the reduced
equation $I=-C_1$, viz.
\begin{equation}
\frac{1}{2}a^2\omega'^2=\frac{1}{3}\omega^3+D\omega^2+E\omega+F+C_1,
\label{ch3:n45} \end{equation} where $C_1$ is an arbitrary constant. The
solution to Eq. (\ref{ch3:n45}) is given by 
\begin{equation}
\int {dx\over a}=\sqrt{3\over2} \int  \frac{d\omega}{\left [ \omega^3
+3D\omega^2 +3E\omega+3F+3C_1\right]^{1/2}}+C_2 ,\label{ch3:n46}
\end{equation} 
where $C_2$ is an arbitrary constant of integration. In
order to complete the integration, we need to solve Eqs
(\ref{ch3:n32}). 

Eq. (\ref{ch3:n32}a) gives 
\begin{equation} 
b=\alpha
x+\beta . \label{ch3:n47} 
\end{equation} 

The above relations (\ref{ch3:n42}), (\ref{ch3:n43}) and (\ref{ch3:n44}) for $D$, $E$
and $F$ can systematically and easily be used to solve the equation (\ref{ch3:n32}b) which 
up to now has been solved in an {\em ad hoc} manner
(cf. Polyanin and Zaitsev 1995, Maharaj {\em et al} 1996). The solution of (\ref{ch3:n32}b)
naturally comes out from the relations (\ref{ch3:n42}), (\ref{ch3:n43}) and (\ref{ch3:n44}) straightforwardly by
substituting (\ref{ch3:n42}) into (\ref{ch3:n43}), (\ref{ch3:n44}) and finally the resulting expression for 
$ba^{-1/2}a'/2$ in (\ref{ch3:n43}) into (\ref{ch3:n44}). This gives 
\begin{equation}
F=b^2a^{-1}/2+E\int ba^{-3/2}dx-D(\int ba^{-3/2}dx)^2+(\int ba^{-3/2}dx)^3/3\,.\label{ch3:52}
\end{equation}
Now set 
\begin{equation}
\tau=-\int ba^{-3/2}dx \label{ch3:53}
\end{equation}  
and insert (\ref{ch3:53}) into (\ref{ch3:52}). Then (\ref{ch3:52}) becomes 
\begin{equation}
-{dx\over b^2}=\left(\frac32\right)^{3/2} {d\tau\over [\tau^3 +3D\tau^2+3E\tau+3F]^{3/2}}\label{ch3:54}
\end{equation} 
with $b$ given by (\ref{ch3:n47}). In order to obtain $\tau(x)$ one need to first solve (\ref{ch3:54})
for $\tau(x)$ and then from (\ref{ch3:53}) 
\begin{equation}
a=(-b)^{2/3}\left({d\tau\over dx}\right)^{-2/3}\,.\label{ch3:55}
\end{equation}

Two cases arise in solving (\ref{ch3:54}). The solutions for each case, in general,
are expressed in parametric form.

{\bf Case (i)}: $\alpha=0$, $\beta\ne0$.

For convenience we let
\begin{equation}
h(\tau)=-\left({3\over2}\right)^{3/2}\int {d\tau\over[\tau^3+3D\tau^2+3E\tau+3F]^{3/2}}\,\cdot\label{ch3:56}
\end{equation}
It is easy to deduce that for this case $a$ is defined in parametric form by
\begin{equation} \label{ch3:57} 
\begin{array}{lll} x &=&
\beta^2h(\tau)-k,\\ 
a &=& \beta^2[2\tau^3/3+2D\tau^2+2E\tau+2F]^{-1}, 
\end{array} 
\end{equation}
where $k$ is  a constant. 

{\bf Case (ii)}: $\alpha\ne0$.

In this case $a$ is given in parametric form by
\begin{equation}\label{ch3:58}
\begin{array}{lll} \alpha x+\beta &= & (k-\alpha h(\tau))^{-1},\\
a&=&(k-\alpha h(\tau))^{-2}[\frac23\tau^3+2D\tau^2+2E\tau+2F]^{-1},
\end{array}
\end{equation}
where $k$ is a constant.

The integral on the left hand side of (\ref{ch3:n46}) contains $a$ and $dx$ which are known in parametric
form from (\ref{ch3:57}) and (\ref{ch3:58}).

We firstly invoke (\ref{ch3:57}) into (\ref{ch3:n46}). This yields
\begin{equation}
- \int {d\tau\over \left [\tau^3 +3D\tau^2+3E\tau+3F\right ]^{1/2}}= 
\int \frac{d\omega}{\left [ \omega^3
+3D\omega^2 +3E\omega+3F+3C_1\right]^{1/2}}+C_2.   \label{ch3:59}
\end{equation}

Let us next insert (\ref{ch3:58}) into (\ref{ch3:n46}). Precisely the same equation (\ref{ch3:59})
results.

We are now in a position to write down the general solution of Eq.\ (\ref{ch3:i2}) in terms of quadratures
for the Case II.2. There are naturally two cases (i) and (ii) that
arise. We find that (remember that $y=a^{1/2}(\omega-\tau)$ from (\ref{ch3:n40})):

{\bf Case (i)}: $\alpha=0$, $\beta\ne0$.

\begin{equation}\label{ch3:60}
\begin{array}{lll}  x &= & \beta^2 h(\tau) -k,\\
y&=&\beta[\frac23\tau^3+2D\tau^2+2E\tau+2F]^{-1/2}(\omega-\tau),\\
f&=&\beta^{-5}[\frac23\tau^3+2D\tau^2+2E\tau+2F]^{5/2}\,.
\end{array}
\end{equation}

{\bf Case (ii)}: $\alpha\ne0$.

 \begin{equation}\label{ch3:61}
\begin{array}{lll} \alpha x+\beta &= & (k-\alpha h(\tau))^{-1},\\
y&=&(k-\alpha h(\tau))^{-1}[\frac23\tau^3+2D\tau^2+2E\tau+2F]^{-1/2}(\omega-\tau),\\
f&=&(k-\alpha h(\tau))^{5}[\frac23\tau^3+2D\tau^2+2E\tau+2F]^{5/2}\,.
\end{array}
\end{equation}

In equations (\ref{ch3:60}) and (\ref{ch3:61}), $h(\tau)$ is given by (\ref{ch3:56}) and $\omega$ by (\ref{ch3:59}).
Equations (\ref{ch3:60}) and (\ref{ch3:61}) provide a new class of solutions. They also contain existing
solutions as we shall soon point out below.

Now let us investigate when the solutions given by Eqs. (\ref{ch3:60}) and (\ref{ch3:61}) can be written in
terms of elementary functions.  The integrals occuring in Eqs. (\ref{ch3:56}) and (\ref{ch3:59}) are
expressible in terms of elementary functions only if the radicands in (\ref{ch3:56}) and (\ref{ch3:59})
have three repeated factors or two repeated factors.
In the other case it is expressible in terms of elliptic functions. 

{\bf Case (a): Three repeated factors}.

We firstly consider the situation when the radicand in (\ref{ch3:56}) has three repeated factors. There are two
cases (i) and (ii) to look at which we do in turn. If one sets the radicand in (\ref{ch3:56}) as $(\tau+D)^3$, then
$D^2=E$, $D^3=3F$ and  $h(\tau)=\left(\frac32\right)^{3/2}\left(\frac27\right)(D+\tau)^{-7/2}$.  Also
(\ref{ch3:59}) becomes
\begin{equation}
-\int {d\tau\over (D+\tau)^{3/2}}=\int {d\omega\over \left [ (\omega+D)^3
+3C_1\right]^{1/2}}+C_2.\label{ch3:63}
\end{equation}
In order for the right hand side radicand to have three repeated roots, we require $C_1=0$.

For Case (i), (\ref{ch3:60}) gives the one-parameter family of solutions
\begin{equation}\label{ch3:64}
\begin{array}{lll}  
x&=&\beta^2(\frac32)^{3/2}\left(\frac27\right)(D+\tau)^{-7/2}-k,\\
y&=&\beta^{1/7}(\frac32)^{-1/7}(\frac27)^{-3/7}(x+k)^{3/7}\{
[-(x+k)^{1/7}\beta^{-2/7}(\frac32)^{-3/14}(\frac27)^{-1/7}\\
& & +C_2/2]^{-2}\}
-\beta^{5/7}(\frac32)^{2/7}(\frac27)^{-1/7}(x+k)^{1/7},\\
f&=&\beta^{-5/7}(\frac32)^{5/7}(\frac27)^{15/7}(x+k)^{-15/7}\,.
\end{array}
\end{equation}
Although Stephani (1989) considers this case and provides the general solution in terms of 
a line integral, the one-parameter solutions (\ref{ch3:64}) are, to the best of our knowledge,
new and it would be interesting to see how they could 
be obtained from Stephani's. These are the only solutions that are not in 
parametric form for $f\sim x^{-15/7}$.

In the Case (ii), (\ref{ch3:61}) implies the one-parameter family of solutions
\begin{equation}\label{ch3:65}
\begin{array}{lll} \alpha x+\beta &= & [k-\alpha(\frac27)(\frac32)^{3/2}(D+\tau)^{-7/2}]^{-1},\\
y&=&\alpha^{-3/7}(\frac32)^{-1/7}(-\frac27)^{-3/7}(\alpha x+\beta)^{4/7}(1-k\beta-k\alpha x)^{3/7}\times\\
&&[-(1-k\beta-k\alpha x)^{1/7}(\frac27)^{-1/7}\alpha^{-1/7}(\frac32)^{-3/14}(\alpha x+\beta)^{-1/7}+C_2/2]^{-2}\\
&& -\alpha^{-1/7}(\frac32)^{2/7}(-\frac27)^{-1/7}(\alpha x+\beta)^{6/7}(1-k\beta-k\alpha x)^{1/7},\\
f&=&\alpha^{15/7}(\frac32)^{5/7}(-\frac27)^{15/7}(\alpha x+\beta)^{-20/7}(1-k\beta-k\alpha x)^{-15/7}\,.
\end{array}
\end{equation}
This one-parameter family of solutions which are not in parametric form
is, as far as we are aware, new, although the form
$f\sim (x-x_1)^{-15/7}(x-x_2)^{-20/7}$ was quoted before in
Stephani and Wolf (1996).

{\bf Case (b): Two repeated factors}.

We next look at the situation when the radicand in (\ref{ch3:56}) has two repeated factors. There are two
cases (i) and (ii). If we write the radicand in (\ref{ch3:56}) as $(\tau-p)^2(\tau-q)$, then $3D=-(2p+q)$,
$3E=(p^2+2pq)$, $3F=-p^2q$ and 
\begin{equation}\label{ch3:66}
h(\tau)=\left \{  \begin{array}{ll} 
&\displaystyle -\left({-1\over 2(p-q)(\tau-p)^2}+{5\over 4(p-q)^2(\tau-p)}+{15\over 4(p-q)^3}\right){(\frac32)^{3/2}\over
\sqrt{\tau-q}}\\
&- \displaystyle{15(\frac32)^{3/2}\over 8(p-q)^{7/2}}{\rm arctanh} {\sqrt{\tau-q}\over \sqrt{p-q}}, \quad {\rm if}\quad p>q\,,\\
&\displaystyle -\left({-1\over 2(p-q)(\tau-p)^2}+{5\over 4(p-q)^2(\tau-p)}+{15\over 4(p-q)^3}\right){(\frac32)^{3/2}\over
\sqrt{\tau-q}}\\
& +\displaystyle{15(\frac32)^{3/2}\over 8(q-p)^{7/2}}{\rm arctan} {\sqrt{\tau-q}\over \sqrt{q-p}}, \quad {\rm if}\quad p<q\,.
\end{array}\right.
\end{equation}
Equation (\ref{ch3:59}) when $C_1=0$ becomes
\begin{equation}
-\int {d\tau\over (\tau-p)(\tau-q)^{1/2}}=\int\displaystyle {d\omega\over (\omega-p)(\omega-q)^{1/2}}+C_2.\label{ch3:67}
\end{equation}
We deduce from (\ref{ch3:67}) that 
\begin{equation}\label{ch3:68}
\omega=\left \{  \begin{array}{ll}
&q+(p-q)\left[\displaystyle{(\tau-q)^{1/2}+C_3\sqrt{p-q}\over \sqrt{p-q}+C_3(\tau-q)^{1/2}}\right]^2, 
\quad {\rm if}\quad p>q\,,\\
&q+(q-p)\left[\displaystyle{(\tau-q)^{1/2}+C_3\sqrt{q-p}\over \sqrt{q-p}-C_3(\tau-q)^{1/2}}\right]^2, 
\quad {\rm if}\quad p<q\,,
\end{array}\right.
\end{equation}
where $C_3={\rm tanh}\,(\sqrt{p-q}C_2/2)$ in (\ref{ch3:68}a) and
$C_3={\rm tan}\,(\sqrt{q-p}C_2/2)$ in (\ref{ch3:68}b).
Note that as $C_3\rightarrow \infty$, $\omega\rightarrow q+(p-q)^2/(\tau-q)$.
This yields a singular solution to Eq. (\ref{ch3:i2}).

For Cases (i) and (ii), (\ref{ch3:60}) and (\ref{ch3:61}) give the one-parameter family of solutions expressible 
in terms of elementary functions in parametric form
with $h(\tau)$ and $\omega$ given by (\ref{ch3:66}) and (\ref{ch3:68}) respectively. These constitute new 
closed form solutions.

We now provide a summary of the comparison of our results with those previously known.

Previous workers have obtained exact solutions of Eq. (\ref{ch3:i2}) for various 
forms of $f(x)$. In this regard, we can mention for example the paper
by Wolf and Stephani (1996) wherein the two Lie point symmetry and a few
one symmetry cases were stated. In the Noether approach adopted here,
the one Lie point symmetry cases $f(x) =e^x$, $f(x)=x^n$ $(n\ne0,-5,-15/7,-20/7)$ and $f(x)=(x+\alpha)^n
(x+\beta)^{-n-5}$ ($n\ne-5/2$), which turn out to be not integrable,
do not arise. The cases $f(x)=x^{-15/7}$, $f(x)=(x-x_1)^{-15/7}(x-x_2)^{-20/7}$ and the 
Kustaanheimo and Qvist forms do arise here as mentioned before. 
All the two Lie point symmetry cases are equivalent (not
from a physical point of view though) to the case $f(x)=1$
(Conrad {\em et al } 1994, Herlt and Stephani 1992, Stephani and Wolf 1996)
which turns out to be a  Noether
case. The Noether 
approach picks out  existing integrable cases and new ones not treated by
previous workers. In fact we have provided the solutions in parametric form by means of the formulas
(\ref{ch3:60}) and (\ref{ch3:61}). Notwithstanding we have presented new solutions in closed form with $h(\tau)$
and $\omega$ given by (\ref{ch3:66}) and ({\ref{ch3:68}). 

\section{Applications: Non--static spherically
symmetric perfect fluid solutions} 

\subsection{The basic equations}
In a comoving frame of reference, the line element for time-dependent,
spherically symmetric perfect fluid solutions reads
(see for example Kramer {\em et al} 1980)
\begin{equation} ds^2=Y^2(r,t)d\Omega^2+\mbox{e}^{2\lambda
(r,t)}dr^2-\mbox{e}^{2\nu (r,t)}dt^2, \label{ch3:b1} \end{equation} where
\begin{equation} d\Omega \equiv d\theta^2+\sin^2 \theta d\phi^2.
\label{ch3:b2} \end{equation} In this coordinate system the field equations
are
\begin{eqnarray} \kappa_0\mu &=
&\frac{1}{Y^2}-\frac{2}{Y}\mbox{e}^{-2\lambda} \left ( Y''-Y'\lambda'
+\frac{Y'^2}{2Y} \right )+\frac{2}{Y}\mbox{e}^{-2\nu}\left (\dot Y \dot
\lambda +\frac{\dot Y^2}{2Y} \right ), \label{ch3:b3}\\ \kappa_0p &=
&-\frac{1}{Y^2}+\frac{2}{Y}\mbox{e}^{-2\lambda} \left ( Y'\nu'
+\frac{Y'^2}{2Y} \right )-\frac{2}{Y}\mbox{e}^{-2\nu} \left (\ddot Y
-\dot Y\dot \nu +\frac{{\dot Y}^2}{2Y} \right ), \label{ch3:b4}\\ \kappa_0 p Y
&= &\mbox{e}^{-2\lambda}[(\nu ''+\nu^{'2}-\nu '\lambda ')Y+Y''+Y' \nu'
-Y'\lambda '] \nonumber\\
 & &-\mbox{e}^{-2\nu}[(\ddot \lambda+\dot \lambda^2-\dot \lambda  \dot
 \nu)Y+ \ddot Y+ \dot Y \dot \lambda -\dot Y \dot \nu] \label{ch3:b5}\\
0 &=& \dot Y'-\dot Y\nu '-Y'\dot \lambda, \label{ch3:b6}
\end{eqnarray}
where $\mu$ is the energy density, $p$ is the pressure and $\kappa_0$ is
the gravitational constant, the overdot represents differentiation with
respect to $t$ and the prime  differentiation with respect to $r$. The
4--velocity is  \[u^i=(0,0,0,\mbox{e}^{-\nu}),\quad i=1,\ldots,4.\]

The conservation of
the energy--momentum tensor $T_{ab}=(p+\mu)u_au_b+pg_{ab}$
yields the following useful equations
\begin{equation}
p'=-(\mu+p)\nu ',\;\;\dot \mu=-(\mu+p)(\dot \lambda+2\dot Y/Y). \label{ch3:m}
\end{equation} Just as for perfect fluids, the spherically symmetric solutions
can be classified according to their kinematical properties. That is, the
4--velocity's rotation ($\omega_{ab}$), acceleration ($\dot u_i$),
expansion ($\Theta$) and shear ($\sigma_{ab}$).  As a result of spherical symmetry,
$\omega_{ab}=0$. The other quantities  are 
\begin{eqnarray} \dot u_i &= & (0,0,\nu ',0),\;\;  \Theta
=\mbox{e}^{-\nu}(\dot \lambda+2\dot Y/Y), \nonumber\\ &  & \label{ch3:b7} \\
\sigma_1^1 &=&\sigma^2_2=-\frac{1}{2}\sigma_3^3=\frac{1}{3}\mbox{e}^{-\nu}
(\dot Y/Y-\dot \lambda). \nonumber \end{eqnarray} 

\subsection{Expanding solutions without shear} 

These solutions were characterized by
Kustaanheimo and Qvist (1948) as follows. 
\begin{theo}[Kustaanheimo and Qvist 1948]
 \label{ch3:t4}
\begin{em}
Every non--static, expanding and shear--free spherically
symmetric perfect fluid  solution has the form \begin{equation} ds^2
=\frac{1}{F^2}(dr^2+r^2d\theta^2+r^2\sin^2 \theta d\phi^2)-\left [
\frac{\dot F}{FA(t)}\right ]^2dt^2, \label{ch3:b8} \end{equation} where \[
F=F(x,B(t),C(t))\] is the general solution of the equation
\begin{equation} F_{xx}=f(x)F^2, \label{ch3:b9} \end{equation} 
with $x=r^2$ and 
$f(x),\;A(t)$ and $B(t)$ arbitrary functions of integration. \end{em}
\end{theo} 
{\bf Proof.} See Kustaanheimo and Qvist (1948), Kramer {\em et al } (1980).

Eq. (\ref{ch3:b9}) is in fact
Eq. (\ref{ch3:i2}) which we have studied at length in Section 3.2. We can readily
exhibit a non--static spherically symmetric  expanding solution without
shear that is not contained in the Kustaanheimo and Qvist class. For example
Case (b) of Section 3.2 yields this type of solutions:
take $x=r^2$, $F=y$, where now $C_3$ is a function of $t$.

\subsection{Solutions with shear but without acceleration} 
From Eq (\ref{ch3:b7}a) we infer that $\nu$ is a function of  $t$ only and using Eq.
(\ref{ch3:m}a), we find that $p$ is also function of $t$ only. If we assume
further that $Y'\ne 0 $ then we can choose $\nu=0$ and the field equations
reduce  to (Kramer {\em et al} 1980)
\begin{eqnarray} \mbox{e}^{2\lambda} &=&
Y'^2/(1-\epsilon f^2(r)),\; \epsilon=0,\pm 1, \label{ch3:s1}\\ \kappa_0 p(t) Y^2
&=& -2Y\ddot Y-\dot Y^2-\epsilon f^2(r), \label{ch3:s2}\\ \kappa_0\mu &=&
-3p(t)-2\ddot Y'/Y', \label{ch3:s3} \end{eqnarray} where $f(r)$ is an
arbitrary function of integration. In Eq. (\ref{ch3:s2}), perform the change
of variable (cf. Herlt 1996)
\begin{equation} Y=Z^{\alpha} , \label{ch3:s4} \end{equation}
where $\alpha$ is a real constant to be chosen properly later. In the new
variable, Eq. (\ref{ch3:s2}) reads \begin{equation} 2\alpha Z^{2\alpha
-1}\ddot Z+\alpha (3\alpha-2)Z^{2\alpha-2}\dot Z^2
=-\kappa_0p(t)Z^{2\alpha}-\epsilon f^2(r). \label{ch3:s5} \end{equation} In
order to get rid of $\dot Z$, choose $\alpha=2/3$. Hence \begin{equation}
\ddot Z=-\frac{3}{4}\left (\kappa_0p(t)Z+\epsilon f^2(r)Z^{-1/3} \right ).
\label{ch3:s6} \end{equation} Further make the transformation of variables
\footnote{This type
of change of variables is also used in the study of the Ermakov--Pinney
equation (Ermakov 1880).} \begin{equation} W=Z/\rho (t,r),\;\; \tau=\int^t \rho
^{-2}(\xi,r)d\xi, \label{ch3:s7} \end{equation} where $\rho (t,r)$ is a solution
of the linear second-order equation \begin{equation}  \rho_{tt} +\frac{3}{4}\kappa_0 p(t)\rho
=0. \label{ch3:s8} \end{equation} Eq. (\ref{ch3:s6}) becomes
\begin{equation}
W_{\tau\tau} =-\frac{3}{4}\epsilon \rho^{8/3}f^2(r) W^{-1/3}. \label{ch3:s9}
\end{equation}
At this point, two cases arise.

{\bf Case I  \boldmath{$\;\epsilon f(r)=0.$}}\\
We find that
\[ W=a(r)\tau+b(r),\]
where $a(r)$ and $b(r)$ are arbitrary functions of integration.
Thus
\begin{equation}
Y=\rho^{2/3} \left [ a(r)\int^t \rho^{-2} (\xi,r)d\xi +b(r)\right ]^{2/3}.
\label{ch3:s10} \end{equation} In order to avoid zero shear, me must have
$\dot Y'Y\ne \dot YY'$. Hence $a(r)$ and $b(r)$ must be non-zero. 
One need to prescribe $\rho(t,r)$ to get a solution. The pressure $p$ is then given
by Eq. (\ref{ch3:s8}) and the energy density follows from Eq. (\ref{ch3:s3}). This case was also
treated in Herlt (1996).

{\bf Examples}.\\
{\bf (a) Dust solution, $\boldmath {p=0}$}.\\
From Eq. (\ref{ch3:s8}), it follows that $\rho$ will be linear in $t$ (one can set 
$\rho=1$) and from (\ref{ch3:s10})
$$ Y=[c(r)t+d(r)]^{2/3}\,,$$
where $c(r)$ and $d(r)$ are arbitrary functions. By the choices
$c(r)=\pm 3/[2(2m(r))^{1/2}]$ and $d(r)=\mp 3t_0(r)/[2(2m(r))^{1/2}]$,
we re-obtain the solutions of Bondi--Tolman
(see Kramer {\em et al} (1980), p 160).
When the pressure $p$ is constant, Eq. (\ref{ch3:s8}) can invertibly be
mapped to the simplest second-order equation and thus $p$ can be interpreted
without loss of generality as zero pressure with a non-zero cosmological constant.

{\bf (b) Another solution}.\\
Take \[\rho=t^{-1/2},\;\;a(r)=2r,\;\;b(r)=-1.\]
From Eq. (\ref{ch3:s10}), we have \[Y=t^{-1/3}(rt-1)^{2/3}\] and the metric is
given by \begin{equation} \label{ch3:met}
ds^2=t^{-2/3}(rt-1)^{4/3}d\Omega^2+t^{4/3}(rt-1)^{-2/3}dr^2-dt^2.
\end{equation} Straightforward calculations show that $\dot Y'Y\ne \dot
YY'$. The pressure $p$ is given by $ p(t)=-1/[\kappa_0 t^2] $ and
the energy density  is
\[\mu=\frac{1}{\kappa_0}\left (\frac{3}{\kappa_0}+\frac{4}{9}\right )
\frac{1}{t^2}+\frac{8r}{9\kappa_0(rt^2-t)}\left (1-\frac{r}{rt^2-t}\right).\]
It can be observed that for large $r$, $\mu$ depends on $t$ only. Hence,
this model can be considered as being asymptotically barotropic.

One can obtain more solutions by judiciously choosing one or the other
of $\rho$ or $p$.

{\bf Case II  \boldmath{$\;\epsilon f(r)\ne 0.$}}\\
Make the change of variable
\[ W =\delta K,\]
where
\[\delta=\left [\frac{3}{4} f^2(r)\right ]^{3/4}.\]
Eq. (\ref{ch3:s9}) becomes
\begin{equation}
K_{\tau\tau}=-\epsilon \rho^{8/3}K^{-1/3},\quad \epsilon=\pm 1. \label{ch3:s11}
\end{equation}
In Section 3.2, we have seen that Eq. (\ref{ch3:s11}) is solvable by quadratures
for $\rho^{8/3}=-\epsilon f_0(a(r)\tau^2+2b(r)\tau+c(r))^{-4/3}$,
i.e. 
\begin{equation}
\rho=(a(r)\tau^2+2b(r)\tau+c(r))^{-1/2},\label{ch3:s12}
\end{equation}
after we set $f_0=-\epsilon$. Since the pressure depends on time only, Eq. (\ref{ch3:s8})  
imposes certain  constraints on  $a,\;b$ and $c$. 
The solution to Eq. (\ref{ch3:s11}) is
then given by  (see Section 3.2) 
\[K=(a(r)\tau^2+2b(r)\tau+c(r))^{1/2}u,\]
where $u$ is defined by 
\begin{equation}
\int  \frac{du}{\left
[(b^2(r)-a(r)c(r))u^2-3\epsilon u^{2/3}+A(r)\right]^{1/2}} =\int
\frac{d\tau}{a(r)\tau^2+2b(r)\tau+c(r)}+B(r),\label{ch3:sn10}
\end{equation} 
and $A(r)$, $B(r)$ are arbitrary
functions of integration and $\epsilon=\pm1$. Finally the solution to Eq. (\ref{ch3:s2}) with $p$
defined by Eq. (\ref{ch3:s8}) and $\rho$ given by Eq. (\ref{ch3:s12} ) is
\begin{equation} \label{ch3:s13} \left \{ \begin{array}{lll}
d\tau=(a(r)\tau^2+2b(r)\tau+c(r)) dt,\\ Y=\left (\frac{3}{4}
f^2(r)\right )^{1/2}u^{2/3} ,\end{array} \right.
\end{equation} where $u$ is defined by (\ref{ch3:sn10}).

Now we show that the pressure and the energy
density are well-defined. From Eq. (\ref{ch3:s8}), we deduce that
\[p=-\frac{4}{3\kappa_0}\rho^{-1}\rho_{tt}.\] Using the chain
rule, we find that \begin{equation} \label{ch3:s14}
p=-\frac{4}{3\kappa_0}\rho^{-6}\left
[\rho\rho_{\tau\tau}-2\rho_\tau^2\right ], \end{equation} where
$\rho$ is provided by Eq. (\ref{ch3:s12}). Finally the energy density is
computed from Eq. (\ref{ch3:s3}). 

{\bf Remark}. Note that if the pressure is known before hand, (\ref{ch3:s14}) enables us
to properly choose $\rho$. For instance, if $p=0$ (dust solutions), the choice $\rho=1$
is appropriate. On the other hand, we can choose $\rho$ (i.e. choose $a(r)$, $b(r)$ and $c(r)$)
and in this case $p$ is given by (\ref{ch3:s14}). The possible forms of $\rho$ are $1$, $\tau^{-1}$,
$(\tau^2-1)^{-1/2}$ and $(\tau^2+1)^{-1/2}$ which can be deduced from
the factorization of the quadratic in $\rho$. All these forms lead to $p=0$ or $p=$constant.
The solutions that result here are precisely those obtained
by Ruban (1969) (see also Herlt 1996).
These turn out to be the only known solutions.

{\bf Example}.\\
{\bf (a) Dust solution, ${\boldmath p=0}$.}\\
We select $\rho=1$. This corresponds to the choices $a=0$, $b=0$ and $c=1$.  Thus,
$$
\left \{ \begin{array}{lll}
d\tau= dt,\\ 
Y=\left (\frac{3}{4}f^2(r)\right )^{1/2}u^{2/3}, \end{array} \right.
$$ 
where $u$ is defined by
$$
{du\over dt}=(-3\epsilon u^{2/3}+A(r))^{1/2}\,.
$$
This in fact will yield the solutions of Bondi--Tolman. Indeed, we easily obtain
$$
{\dot Y}^2=-\epsilon f^2(r)+F(r)/Y\,,
$$
where $F(r)=f^3(r)A(r)/\sqrt{12}$ is arbitrary. From the last equation, after the introduction
of $d\eta=f dt/Y$,  follow the solutions of Bondi-Tolman
(Kramer {\em et al} 1980).

\section{Conclusion} 
The standard Lie method applied to Eq. (\ref{ch3:i1}) leads to complications as it gives rise
to a third--order nonlinear integro--differential equation which is nontrivial to solve and
hence the difficulty to provide a complete Lie symmetry classification of the equation. 

In general if one
knows a single Lie point symmetry of a second--order ordinary differential equation, 
this fact does not guarantee its complete integrability in a straightforward
manner whereas in the case of a single Noether point
symmetry associated with a  given Lagrangian of the equation, complete integrability
is assured. In this chapter,  we have
studied Noether point symmetries corresponding to a natural Lagrangian of Eq. (\ref{ch3:i1})  in search of new integrable cases. 
As a result we have found new integrable
cases of Eq. (\ref{ch3:i2}) given by the class of solutions
(\ref{ch3:60}) and (\ref{ch3:61}). This has been applied to the construction of
non--static spherically symmetic perfect fluid solutions. Solutions not
reducible to those of Kustaanheimo and Qvist (1948) have been obtained.
Notwithstanding,
existing solutions with shear have been systematically constructed in a unified manner
via the Noether approach. This is contained in the formula (\ref{ch3:n22}).

%\end{document}

%\newpage
%\begin{thebibliography}{99} \bibitem{qvis}
%Kustaanheimo P and Qvist B 1948  A note on some general solutions of the
%Einstein field equations in a spherically symmetric world {\em Soc. Sci.
%Fennica, Comm. Phys.--Math.} {\bf XIII} 12 \bibitem{kram} Kramer D,
%Stephani H, MacCallum M and Herlt E 1980 {\em Exact Solutions of
%Einstein's Field Equations} (Cambridge: Cambridge University Press)
%\bibitem{ovsi} Ovsiannikov L V  1982  {\em Group  Analysis of Differential
%Equations} (New York: Academic Press) 
%\bibitem{ibra} Ibragimov N H 1985 {\em Transformation Groups Applied to Mathematical 
%Physics} (Dordrecht: D Reidel)
%\bibitem{olve}  Olver P J 1986 {\em
%Applications of Lie Groups to Differential Equations} (New York:
%Springer--Verlag) 
%\bibitem{blum} Bluman G W and Kumei S 1989 {\em
%Symmetries and Differential Equations} (New York: Springer--Verlag)
%\bibitem{step} Stephani H 1989 {\em Differential Equations--Their Solutions
%Using Symmetries} (Cambridge: Cambridge University Press) 
%\bibitem{lie}
%Lie S 1912   {\em Vorlesungen \"uber Differentialgleichungen mit bekanten
%infinitesimalen Transformation} (Berlin: Leipzig) 
%\bibitem{leac} Leach
%P G L, Maartens R and Maharaj S D 1992 Self--similar solutions of the
%generalized Emden--Fowler equation {\em Int. J. Non--Linear Mech.}, Vol.
%{\bf27} No. 4, 575--582 
%\bibitem{herl} Herlt H and
% Stephani H 1992  Invariant transformations of the class $y''=F(x)y^n$ of
%differential equations {\em J. Math. Phys.} {\bf 28}(11) 
%\bibitem{wolf} Stephani H and Wolf T 1996 Spherically
%symmetric perfect fluids in shear--free motion---the symmetry approach
%{\em Class. Quantum Grav.} {\bf 13}, 1261--1271 
%\bibitem{noet}
%Noether E 1918 Invariante Variationprobleme {\em Nachr. K\"onig. Gessell.
%Wissen. G\"ottigen, Math--Phys.} {\bf Kl.}, 235--257 
%\bibitem{wyma} Wyman M 1976
% Jeffrey--Williams Lecture: Nonstatic radially symmetric distributions of
%matter {\em Can. Math. Bull.} {\bf 19}, 343 
%\bibitem{maha} Maharaj S D, Leach P G L and Maartens R 1996 Expanding 
%Spherically Symmetric Models without Shear {\em Gen. Relativity and Grav.} {\bf28}(1), 35
%\bibitem{catr} Sarlet W and Cantrijn F 1981 {\em SIAM Rev.} {\bf 23}, 467--494 
%\bibitem{maho}  Kara A H Vawda F E and Mahomed F M    1994 Symmetries of first integrals and
%solutions to differential equations {\em Lie Group and their Applications} Vol. {\bf1} No. 2, 27--48  
%\bibitem{kara} Ibragimov N H, Kara A H and Mahomed F M 1998 Lie-B\"acklund and
%Noether Symmetries with Applications {\em Nonlinear Dynamics} {\bf 15}, 115
%\bibitem{kamk} Kamke E 1983 {\em Differentialgleichungen -- L\"osungsmethoden und L\"osungen I.}
%(Stuttgart: B G Teubner)
%\bibitem{poly} Polyanin A D and Zaitsev V F  1995 {\em Handbook of Exact Solutions for Ordinary
%Differential Equations} (Boca Raton: CRC Press)
%\bibitem{mell} Conrad M Mellin,  Mahomed F M and Leach P G L 1994 Solution of generalised
%Emden-Fowler equations with two symmetries {\em Int. J. Non-Linear Mechanics} {\bf 29}, 529
%\bibitem{herlt} Herlt E 1996 Spherically Symmetric Nonstatic Perfect Fluid Solutions with 
%Shear {\em Gen. Relativity and Grav.} {\bf28}(8), 919
%\bibitem{ruba} Ruban V A 1969 {\em ZhETF} {\bf5b}, 1914 [{\em Sov. Phys. JETP} {\bf29},1027]
%\bibitem{erma} Ermakov V 1880 {\em Univ. Isz. Kiev Series} III {\bf 9}, 1 (trans. by O Harin) 
%\end{thebibliography}
%\end{document}

