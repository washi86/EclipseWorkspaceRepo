%\documentstyle[12pt]{report}
%\setlength{\parindent}{0mm}
%\setlength{\parskip}{14pt}
%\renewcommand{\baselinestretch}{1.5}
%\setlength{\topmargin}{0pt}
%\setlength{\headheight}{0pt}
%\setlength{\headsep}{0pt}
%\setlength{\footskip}{45pt}
%\setlength{\footheight}{0pt}
%\setlength{\textwidth}{430pt}
%\setlength{\textheight}{660pt}
%\setlength{\oddsidemargin}{10pt}
%\begin{document}
%\newcommand{\RR}{\mathrm{I\!R\!}}
%\newcommand{\dt}{\frac{\partial}{\partial t}}
%\newcommand{\dx}{\frac{\partial}{\partial x}}
%\newcommand{\dy}{\frac{\partial}{\partial y}}
%\newtheorem{theo}{Theorem}[chapter]

\chapter{Symmetry breaking for a system of two linear second--order ordinary
differential equations}

A new canonical form for systems of two second--order linear ordinary
differential equations (odes) is obtained. The latter is decisive in
unravelling the symmetry structure of systems of two second--order linear odes.
Namely, we establish that the point symmetry Lie algebras of systems of two
second--order linear odes can be $5$--, $6$--, $7$--, $8$-- or
$15$--dimensional. This result enhances both the richness and the complexity
of the symmetry structure of linear systems. The work presented here
is reported in Wafo Soh and Mahomed (1999e).

\section{Introduction}
Symmetry properties of scalar odes is well--documented in the literature.
Many results about scalar odes were obtained by the founder of symmetry
analysis of differential equations (des) Sophus Lie (1891) himself.
For scalar  second--order odes, Lie showed that the maximum number
of point symmetries is $8$ and is achieved by linear odes.
Also, he proved that a scalar second--order ode is linearizable
if and only if it has $8$ point symmetries.

In the case of scalar linear odes of oder $n$, $n\ge 3$, Mahomed and Leach
(1990) recently showed that the number of point symmetries can be $n+1$,
$n+2$ or $n+4$. This clearly shows that for $n\ge 3$, scalar linear odes are
not equivalent to each other. Further, for $n\ge 3$, there are nonlinear
and nonlinearizable equations with $n+1$ and $n+2$ symmetries
(Ibragimov and Mahomed 1996).
Hence linearization criteria for scalar $n$--order odes, $n\ge 3$, do not
rest solely on the dimension of their Lie point symmetry  algebras
(Mahomed and Leach 1990).

The purpose of this chapter is to investigate symmetry properties of systems
of second--oder linear odes. We show that a system of two second--order linear
odes can admit $5$, $6$, $7$, $8$ or $15$  point symmetries. We also provide
a representative element for each class.

The outline of this work is as follows. In Section 5.2 we investigate
canonical forms for systems of linear odes. In Section 5.3 we discuss
group classification for systems of two second--order linear odes. Finally,
in Section 5.4 we discuss some implications of the results obtained in the
previous section.

We would like to refer the
reader to a recent paper by Gorringe and Leach (1988). These authors
treated the case of systems of two second--oder linear odes
with constant coefficents in the complex
domain and they proved that these systems can have $7$, $8$ or $15$
point symmetries.


\section{Canonical forms of systems of second--order linear odes}
The major difficulties one faces when trying to classify the general linear
equation is the huge number of arbitrary elements. For a system of $n$
second--order nonhomogeneous linear odes, there are $2n^2+n$ arbitrary
elements. Since invertible transformations do not affect the number of
symmetries, we shall get a simple form of the system before performing
group classification. Let us start with a theorem which will be crucial for
reducing the number of arbitrary elements.
\begin{theo}
\begin{em}
\label{ch5:t1}
Any system of $n$ second--order nonhomogeneous linear odes
\begin{equation}
\label{ch5:c1}
\b x''=A\b x'+B\b x +\b c
\end{equation}
can be mapped invertibly to one of  the following forms
\begin{eqnarray}
\b y'' &=&\bar A \b y',  \label{ch5:c2} \\
\b z'' &= & \bar B \b z,  \label{ch5:c3} 
\end{eqnarray}
where $A$, $B$, $\bar A$, $\bar B$ are $n \times n$ matrices
and $\b x$, $\b y$, $\b z$ and $\b c$ are vectors.

We shall refer to (\ref{ch5:c2}) as
the {\em first canonical form} of the general system of second--order linear 
odes (\ref{ch5:c1})  and to (\ref{ch5:c3}) as the {\em second canonical form}. 
Note that they are the counterpart of the
Laguerre--Forsyth canonical forms (Forsyth 1921) for scalar equations.
\end{em}
\end{theo}

{\bf Proof.}
Let $\b x^{*}=(x^{*}_{1},\ldots,x^{*}_{n})^T$ be a
particular solution of Eq. (\ref{ch5:c1}). Under the change of variables 
\begin{equation}
\b x=M\b y+\b x^{*}, \label{ch5:eq2} 
\end{equation}
Equation (\ref{ch5:c1}) becomes
\begin{equation}
M\b y^{''}+\left (2M^{'}-AM \right )\b y^{'}
+\left (M^{''}-AM^{'}-BM \right )\b y=0. \label{ch5:eq3} 
\end{equation}
Now let 
\[ \b m_j=(m_{ij})^T,\;\; i,j=1,\ldots ,n\]
be $n$ linearly independent solutions of the homogeneous equation associated
with (\ref{ch5:c1}).  Choose $M=[m_{ij}]$. Thus M is nonsingular and satisfies
the equation
\[M^{''}-AM^{'}-BM=0.\]
Whence with this choice, Eq. (\ref{ch5:eq3}) reduces to
\begin{equation}
\b y^{''}=\bar A \b y^{'}, \label{ch5:eq4}
\end{equation}
where $\bar A=M^{-1}(AM-2M^{'})$. By taking $\b m_j=(m_{ij})^T$ to be $n$ 
linearly independent solutions of
\[2\b x^{'}-A\b x=0\]
and $M=[m_{ij}]$, we arrive at the equation
\begin{equation}
\b y^{''}=\bar B \b y, \label{ch5:eq5}
\end{equation}
where $\bar B= M^{-1}(AM^{'}+BM-M^{''})$. It should be stressed that in general,
$\bar A $ and $\bar B$
are nonzero. This completes the proof of the theorem.

When we specialize Theorem \ref{ch5:t1} to the case $n=2$ we see that the
number of arbitrary elements reduces from $2\times 2^2+2=10$ to
$4$. We aim at obtaining further reduction in the case $n=2$. Consider a
system of two second-order linear odes written in the second canonical form
\begin{equation}
\label{ch5:c4}
\left \{ \begin{array}{lll}
          x'' &=& a(t)x+b(t)y, \\
          y'' &= & c(t)x+d(t)y.
          \end{array}
          \right.
\end{equation}
Perform the following change of variables
\begin{equation}
\label{ch5:c5}
\bar x=x/\rho (t),\;\; \bar y=y/\rho (t),\;\; \bar t= \int^t \rho^{-2}(s)ds,
\end{equation}
where $\rho$ satisfies the following equation
\begin{equation}
\label{ch5:c6}
\rho ''-\frac{a+d}{2}\rho =0.
\end{equation}
Simple calculations show that
\begin{equation}
\label{ch5:c6}
\left \{ \begin{array}{lll}
        \bar x'' &=&\alpha (\bar t)\bar x+ \beta (\bar t) \bar y, \\
        \bar  y'' &= & \gamma (\bar t)\bar x -\alpha (\bar t) \bar y,
          \end{array}
          \right.
\end{equation}
where
\[ \alpha=\frac{\rho^3 (a-d)}{2},\;\;\beta=\rho^3 b,\;\;\gamma=\rho^3 c.\]
Hence we have proved the following theorem.
\begin{theo}
\label{ch5:t2}
\begin{em}
Any system of two second-order linear odes can be mapped invertibly to
the linear system
\begin{equation}
\label{ch5:c7}
\left \{ \begin{array}{lll}
          x'' &=& a(t)x+b(t)y, \\
          y'' &= & c(t)x-a(t)y.
          \end{array}
          \right.
\end{equation}
\end{em}
\end{theo}
Theorem \ref{ch5:t2} implies that symmetry properties of a system of two
second--order linear odes depend on three arbitrary elements only. Thus
the problem of group classification of systems of two second--order linear
odes is reduced to that of (\ref{ch5:c7}). In the sequel, we shall perform
group classification of (\ref{ch5:c7}).

\section{Group classification of systems of two second--order linear odes}

Models occuring in different fields of applied science
(Fluid Mechanics, Elasticity, \dots)  sometimes contains
some parameters (functions)  which cannot be determined by any known law.
Quite often, empirical results may help in determining the form of these
unknown parameters. But Lie--Ovsiannikov's theory provides an elegant
and simple way
of finding these arbitrary parameters: its seems that nature prefers models
which are maximally symmetric. Hence finding such models leads to explicit
forms for the unknown parameters. This process is termed
{\em group classification}.
The classification is done up to form preserving transformations (equivalence
transformations). For the methodology of group classification, the excellent
book by Ovsiannikov (1982) is recommended.

Consider the operator
\[X=\xi (t,x,y)\dt +\eta (t,x,y)\dx+\mu (t,x,y)\dy \; \cdot\]
According to the Lie algorithm described in Chapter 1, $X$ is a symmetry of
(\ref{ch5:c7}) if and only if it satisfies the system
\begin{eqnarray}
\xi_{xx}  =  0,\;\; \xi_{yy}=0,\;\;\xi_{xy}=0,\;\;\mu_{xx} &=& 0, \label{ch5:d1} \\
\eta_{xy}-\xi_{ty} = 0,\;\;\eta_{xx}-2\xi_{tx} &= & 0,\label{ch5:d2}\\
\eta_{ty}-(ax+by)\xi_{y} &= & 0, \label{ch5:d3}\\
2\eta_{tx}-\xi_{tt}-3(ax+by)\xi_{x}-(cx-ay)\xi_{y} &=& 0, \label{ch5:d4}\\
\eta_{tt}+(ax+by)(\eta_x-2\xi_t)+(cx-ay)\eta_y &=& a'x\xi +b'y\eta +
a\eta + b\mu , \label{ch5:d5}\\
\mu_{xy}-\xi_{tx} =  0,\;\;\mu_{yy}-2\xi_{ty} &= & 0, \label{ch5:d6}\\
\mu_{tx}-(cx-ay)\xi_x &= & 0, \label{ch5:d7}\\
\mu_{tt}+ (cx-ay)(\mu_y-2\xi_t)+(ax+by)\mu_x &=&
c'x\xi -a'y\xi + c\eta -a\mu , \label{ch5:d8}\\
2\mu_{ty}-\xi_{tt}-3(cx-ay)\xi_y-(ax+by)\xi_x &=& 0. \label{ch5:d9}
\end{eqnarray}
Using (\ref{ch5:d1}), (\ref{ch5:d2}) and (\ref{ch5:d6}), we get
\begin{eqnarray}
\xi &= & \alpha (t) x+\beta (t) y+\lambda (t), \label{ch5:d10}\\
\eta  &=& \alpha '(t)x^2+\beta '(t)xy+r(t)x+s(t)y+u(t), \label{ch5:d11}\\
\mu &=& \beta '(t)y^2+\alpha '(t)xy+p(t)x+q(t)y+v(t). \label{ch5:d12}
\end{eqnarray}
Now substitute (\ref{ch5:d10})--(\ref{ch5:d12}) into (\ref{ch5:d3}). After some
simple calculations, we obtain:
\begin{eqnarray}
s &=&s_0=\mbox{const.}, \label{ch5:d13}\\
\beta b &= & 0,\label{ch5:d14}\\
\beta ''-a\beta &= & 0. \label{ch5:d15}
\end{eqnarray}
Equation (\ref{ch5:d14}) prompts the consideration of the following cases:

{\bf Case I  \boldmath{ $b\ne 0.$} }

In this case, simple manipulations leads to the following:
\begin{eqnarray}
\xi &=& \lambda (t),\label{ch5:d16}\\
\eta &=& (\frac{\lambda '}{2}+r_0)x+s_0 y+u(t), \label{ch5:d17}\\
\mu & =& p_0 x+(\frac{\lambda '}{2}+q_0)y+v(t), \label{ch5:d18}
\end{eqnarray}
where $(u,v)$ solves (\ref{ch5:c7}), $r_0,\;s_0,\;p_0,\; q_0$ are arbitrary
constants and  $\lambda$ satisfies the equations
\begin{eqnarray}
\frac{\lambda '''}{2}-2a\lambda '-a'\lambda+s_0 c-bp_0 &=&0,
\label{ch5:d19}\\
\frac{\lambda '''}{2}+2a\lambda '+a'\lambda-s_0 c+bp_0 &=& 0,
\label{ch5:d20}\\
2b\lambda '+b'\lambda+2as_0+(q_0-r_0)b &= & 0 ,\label{ch5:d21}\\
2c\lambda '+c'\lambda-2ap_0+(r_0-q_0)c & =& 0. \label{ch5:d22}
\end{eqnarray}
By adding (\ref{ch5:d19}) to (\ref{ch5:d20}), we obtain:
\[\lambda '''= 0, \]
i.e.
\begin{equation}
\lambda=\frac{c_1}{2}t^2+c_2t+c_3. \label{ch5:d23}
\end{equation}
Substituting (\ref{ch5:d23}) into (\ref{ch5:d20})--(\ref{ch5:d22}), we get
\begin{eqnarray}
2a(c_1t+c_2)+a'(\frac{c_1}{2}t^2+c_2t+c_3)-s_0c+p_0b &=& 0,
\label{ch5:d24}\\
2b(c_1t+c_2)+b'(\frac{c_1}{2}t^2+c_2t+c_3)+2as_0+(q_0-r_0)b &=&0,
\label{ch5:d25}\\
2c(c_1t+c_2)+c'(\frac{c_1}{2}t^2+c_2t+c_3)-2ap_0+(r_0-q_0)c &=&0,
\label{ch5:d26}
\end{eqnarray}
At this point, we can start group classification.

{\bf Case I.1  \boldmath{$\;a,\; b$} and \boldmath{$c$} are arbitrary.}

In this case we find that
\[c_1=c_2=c_3=s_0=p_0=0,\;\;r_0=q_0.\]
Hence
\begin{eqnarray}
\xi &= & 0, \label{ch5:d27}\\
\eta & = & r_0 x+u(t), \label{ch5:d28}\\
\mu & =& r_0 y+v(t), \label{ch5:d29}
\end{eqnarray}
where $(u,v)$ solves (\ref{ch5:c7}). Thus the {\em principal Lie algebra}
(see Ovsiannikov 1982 for terminology) is spanned by the following operators:
\begin{eqnarray}
X_i &=& u_i \dx+ v_i \dy,\;\;i=1,\ldots,4, \label{ch5:d30}\\
X_5 & =& x \dx+ y \dy, \label{ch5:d31}
\end{eqnarray}
where $(u_i,v_i)$ are linearly independent solutions of (\ref{ch5:c7}).\\
Rewrite (\ref{ch5:d24})--(\ref{ch5:d26}) in matrix form.
\[ \left [ \begin{array}{l}
            a'\\
            b'\\
            c'
            \end{array}
   \right ]  +
\left [ \begin{array}{lll}

\displaystyle{\frac{2(c_1t+c_2)}{\frac{c_1}{2}t^2+c_2t+c_3 }} &
\displaystyle{\frac{p_0}{\frac{c_1}{2}t^2+c_2t+c_3 }} &
\displaystyle{\frac{-s_0}{\frac{c_1}{2}t^2+c_2t+c_3 }}\\
\displaystyle{\frac{2s_0}{\frac{c_1}{2}t^2+c_2t+c_3 }} &
\displaystyle{\frac{-2(c_1t+c_2)+q_0-r_0}{\frac{c_1}{2}t^2+c_2t+c_3 }} &    0\\
\displaystyle{\frac{-2p_0}{\frac{c_1}{2}t^2+c_2t+c_3 }} & 0 &
\displaystyle{\frac{2(c_1t+c_2)+r_0-q_0}{\frac{c_1}{2}t^2+c_2t+c_3 }}

\end{array}
\right ]
\left [ \begin{array}{l}
           a\\
           b\\
           c
           \end{array}
 \right ] =0.
\]
Whence the following cases should be considered.

{\bf Case I.2 \boldmath{$\;\;a=a_0,\; b=b_0,\; c=c_0.$} } 

Substitution of $a$, $b$ and $c$ into (\ref{ch5:d24})--(\ref{ch5:d26}) yields
\begin{eqnarray*}
c_1 &=& 0,\\
c_2 &=& \frac{-a_0}{b_0}s_0+\frac{r_0-q_0}{2},\\
p_0 & =& \frac{2a_0^2+b_0c_0}{b_0^2}s_0+\frac{a_0}{b_0}(q_0-r_0),\\
2a_0(a_0^2+b_0c_0)s_0 &= & b_0(a_0^2+b_0c_0)(r_0-q_0).
\end{eqnarray*}
At this point the following subcases arise. 

{\bf Case I.2.1  \boldmath{$\;\; \Delta =a_0^2+b_0c_0=0. $}}
\begin{eqnarray*}
X_6 &= & \dt , \\
X_7 & = & -\frac{a_0}{b_0}t\dt +(-\frac{a_0}{2b_0}x+y)\dx +
(\frac{2a_0^2+b_0c_0}{b_0^2}x-\frac{a_0}{2b_0}y)\dy,\\
X_8 &= & \frac{1}{2}t\dt+\frac{5}{4}x\dx +
(-\frac{a_0}{b_0}x+\frac{1}{4}y)\dy \;\cdot
\end{eqnarray*}
{\bf Case I.2.2  \boldmath{$\;\;\Delta =a_0^2+b_0c_0 \ne 0.$} }
\begin{eqnarray*}
X_6 &= & \dt , \\
X_7 &= &\left \{ \begin{array}{ll}
                 \left ( x+\displaystyle{\frac{b_0}{2a_0}}y \right )\dx & \mbox{if }
                 a_0\ne 0, \\
                 &   \\
                 y\dx +\frac{c_0}{b_0}x\dy & \mbox{if } a_0=0.
                 \end{array}
                 \right.
\end{eqnarray*}

{\bf Case I.3 \boldmath{$\;\;a=a_0,\;b=b_0,\;c\ne\mbox{const.}$}}

In this case, the equations (\ref{ch5:d24}) to (\ref{ch5:d26}) become
\begin{eqnarray}
2a_0(c_1t+c_2)-s_0c+p_0b_0 &=& 0, \label{ch5:n1}\\
2b_0(c_1t+c_2)+2s_0a_0+(q_0-r_0)b_0 &=& 0, \label{ch5:n2}\\
2c(c_1t+c_2)+c'(\frac{c_1}{2}t^2+c_2t+c_3)-2p_0a_0+(r_0-q_0)c &=& 0. \label{ch5:n3}
\end{eqnarray}
Since $b_0\ne0$, (\ref{ch5:n2}) implies that $c_1=0$ and (\ref{ch5:n1}) gives
$s_0=0$. Note that if $q_0=r_0$, there is no extension of the principal Lie
algebra. Hence by assuming that $q_0\ne r_0$, we find up to equivalence
transformation that
\[c=c_0t^{-4}-\frac{a_0^2}{b_0},\; c_0\ne 0\]
and the extending operator is
\[X_6=\frac{1}{2}t\dt +\frac{5}{4}x\dx+ \left (-\frac{a_0}{b_0}x+\frac{1}{4}y
\right )\dy \cdot\]

{\bf Case I.4 \boldmath{$a=a_0,\;b\ne \mbox{const.},\;c=0.$}}

Simple calculations lead to the following subcases.

{\bf Case I.4.1 \boldmath{$\;\;a=a_0\ne 0,\;b=b_0\mbox{e}^{mt}+ka_0,\;c=0.$}}

Straightfoward manipulations yield the extra operator
\[X_6=\frac{1}{m}t\dt+\left ( x+\frac{k}{2}y \right ) \dx \;\cdot \]

{\bf Case I.4.2 \boldmath{$\;\;a=0=c,\;b\ne \mbox{const.}$}}

Eqs. (\ref{ch5:d24})--(\ref{ch5:d26}) become
\[
p_0 =0 ,\quad
(\frac{c_1}{2}t^2+c_2t+c_3)b'+(2c_1t+2c_2+q_0-r_0)b = 0 .
\]
Whence the subcases:

{\bf Case I.4.2.1  \boldmath{$\;\;b$} arbitrary.}

In this case, $p_0=0$, $r_0=q_0$, $c_1=c_2=c_3=0$.
\[X_6=y\dx\cdot \]

{\bf Case I.4.2.2 \boldmath{$\;b=b_0\ne 0.$}}

This case falls into Case I.2.1.

{\bf Case I.4.2.3 \boldmath{$\;\;b\ne \mbox{const.}$} }

\[b=b_0(\frac{c_1}{2}t^2+c_2t+c_3)^{-2}\exp \left [
\int \frac{1}{\frac{c_1}{2}t^2+c_2t+c_3} \right ].  \]
Using time translation and scaling which are obviously
equivalence transformations, we arrive at the following subcases:

{\bf Case I.4.2.3.1  \boldmath{$\;\;b=b_0t^m,\; m\ne 0,\;-4.$}}

Simple calculations show that
\[ c_1= 0,\quad c_3=0,\quad c_2 = \frac{r_0-q_0}{m+2}\cdot \]
Thus the extending operators are:
\[ X_6  = y\dx,\quad X_7  =  2t\dt +(2m+5)x\dx + y\dy \cdot \]
Note that the case $m=-2$ is not special.

{\bf Case I.4.2.3.2 \boldmath{$\;\; b=b_0t^{-4}.$}}

The change of variables
\[T=-1/t,\;X=x/t,\;Y=y/t,\]
brings this case to Case I.2.1. 

{\bf Case I.4.2.3.3 \boldmath{$\;\;b=b_0 \exp (mt)$, $m\ne 0.$} }

Straightfoward computations give:
\[ c_1=c_2 = 0,\quad c_3 =\frac{r_0-q_0}{m}\; \cdot  \]
The extending operators are
\[X_6 = y\dx,\quad X_7 = \dt-my\dy \cdot \]

{\bf Case I.4.2.3.4  \boldmath{$\;\;b=b_0 (t^2+1)^{-2}
\exp (m \mbox{ Arctg } t).$}}

After simple manipulations, we find that the added operators are
\[ X_6 = y\dx,\quad X_7 = (t^2+1)\dt + (t+m)x\dx+ ty\dy \;\cdot \]
{\bf Case I.4.2.3.5  \boldmath{$\;\; b=b_0
\displaystyle{\frac{(1+t)^{m/2-2}}{(1-t)^{m/2+2}}}\;\cdot$}}

The following subcases arise:

(i) If $m\ne \pm 4$, the extending operators are:
\[ X_6 =y\dx,\quad X_7 = (t^2-1)\dt +(t+m)x\dx +ty \dy \;\cdot \]
(ii) If $m=\epsilon 4$, $\epsilon=\pm 1$

The extra operators are
\begin{eqnarray*}
X_6 &=& y\dx,\\
X_7 &=& (t-\epsilon)^2\dt +(t-\epsilon)x\dx + (t-\epsilon)y\dy,\\
X_8 &=& 2(t-\epsilon)\dt +x\dx + 5y \dy \;\cdot
\end{eqnarray*}
{\bf Case I.5 \boldmath{$\;\;a\ne\mbox{const.},\;b=b_0,\;c=c_0.$}}

Routine calculations show that there is no extension of the principal Lie
algebra in this case.

{\bf Case I.6 \boldmath{$\;\;a=a_0,\; b,\;c\ne \mbox{const.}$}}

After some simple calculations, the following subcases arise:

{\bf Case I.6.1  \boldmath{$\;\; a=a_0\ne 0,\; b=b_0\mbox{e}^{mt},\; c=c_0\mbox{e}^{-mt},\;
m\ne 0$.}}

The principal Lie algebra is extended by
\[X_6=\dt+mx\dx \cdot \]

{\bf Case I.6.2  \boldmath {$\; a=0,\;b=b_0(t^2+1)^{-2}\exp (m\mbox{ Arctg } t),\\
c=c_0(t^2+1)^{-2}\exp (-m\mbox{ Arctg } t),\; m\ne 0.$}}

The extra operator is
\[X_6=(t^2+1)\dt+(t+m)x\dx+ty\dy \;\cdot\]

{\bf Case I.6.3  \boldmath{$\; a=0,\; b=b_0 \displaystyle{\frac{(1+t)^{m/2-2}}{(1-t)^{m/2+2}}},\;
c=c_0 \displaystyle{\frac{(1+t)^{-m/2-2}}{(1-t)^{-m/2+2}}},\;m\ne 0.$}}

The additional operator is
\[X_6=(t^2-1)\dt+(t+m)x\dx+ty\dy \; \cdot\]

{\bf Case I.6.4  \boldmath{$\;a=0,\; b=b_0t^{m-2},\;c=c_0t^{-m-2},\;m\ne 0,-2,2.$}}

The extra operator is
\[X_6=2t\dt+(2m+1)x\dx+y\dy \;\cdot\]

{\bf Case I.6.5 \boldmath{$\;a=0,\;b=b_0(\frac{c_1}{2}t^2+c_2t+c_3)^{-2},\;
                       c=\alpha b,\; \alpha=\mbox{const.}$} }

The extending operators are
\begin{eqnarray*}
X_6 &=& y\dx+\alpha x\dy ,\\
X_7 & = & (\frac{c_1}{2}t^2+c_2t+c_3)\dt +(\frac{c_1}{2}t+\frac{c_2}{2})x\dx
+(\frac{c_1}{2}t+\frac{c_2}{2})y\dy \cdot
\end{eqnarray*}
{\bf Case I.6.6  \boldmath{$\; a=a_0,\; b=b_0t,\;c=\frac{p_0b_0}{s_0}t.$}}

\[X_6=-\frac{2a_0s_0}{b_0}\dt+s_0y\dx+p_0x\dy \;\cdot\]

{\bf Case I.6.7 \boldmath{$a=0,\; b=b_0(t^2+\epsilon),\;c=\frac{p_0b_0}{s_0}(t^2+
\epsilon),\; \epsilon =0,\pm 1.$}} 

\[X_6=s_0y\dx+p_0x\dy\;\cdot\]

{\bf Case I.7 \boldmath{$\; a\ne \mbox{const.},\; b=b_0,\; c\ne \mbox{const.}$}}

The only choice of the arbitrary elements leading to the extension of the
principal Lie algebra is up to equivalence transformations:
\[a=a_0t^{-2}+kb_0,\; b=b_0,\; c=c_0t^{-4}-k(2a_0t^{-2}+kb_0),\]
where $a_0,\;c_0,\;k$ are constants, $a_0\ne 0$, $(c_0,k)\ne (0,0)$. The extra
operator is:
\[X_6=2t\dt+5x\dx+(y-4kx)\dy \;\cdot \]

{\bf Case I.8 \boldmath{$\; a\ne \mbox{const.},\;b \ne \mbox{const.},\; c=0.$}}

In this case we have
\[a=a_0\left ( \frac{c_1}{2}t^2+c_2t+c_3\right )^{-2}.\]
Up to time translation and scaling, the following subcases arise:

{\bf Case I.8.1  \boldmath{ $\;a=a_0t^{-2},\;\;b=\frac{n}{m-2}t^{-2}+b_0t^{-m},\;c=0
,\;m\ne 2,\;b_0\ne 0.$}}

The principal Lie algebra is extended by
\[X_6=2t\dt +\left [ (5-2m)x-\frac{n}{a_0}y \right ]\dx +y\dy \;\cdot \]

{\bf Case I.8.2  \boldmath{$\;a=a_0t^{-2},\;b=\frac{n}{m-2}t^{-2},\;c=0,\;n\ne 0,\; m\ne 2.$}}

The principal Lie algebra is extended by the following operators:
\[
X_6 = t\dt ,\quad
X_7 = \left (x+\frac{n}{2a_0(m-2)}y \right )\dy \;\cdot
\]

{\bf Case I.8.3  \boldmath{$\;a=a_0t^{-2},\;b=b_0t^{-2}+nt^{-2}\ln t,\; c=0,\;n\ne 0.$} }

The extra operator is
\[X_6=2t\dt +\left ( x-\frac{n}{a_0}y \right ) \dx +y\dy \;\cdot\]

{\bf Case I.8.4 \boldmath{ $\; a=a_0 (t^2+\epsilon)^{-2},\;
bt^4\mbox{e}^{-m/t}=n\int t^2(t^2+\epsilon)^{-2}\mbox{e}^{-m/t} dt,\\
c=0,\; \epsilon=o,\pm 1,\; (m,n)\ne (0,0) \mbox{ if } \epsilon=0.$}}

The principal Lie algebra is extended by the single operator
\[X_6=t^2\dt +\left ( tx-\frac{n}{2a_0}y \right )\dx+(t+m)y\dy\;\cdot \]

{\bf Case I.8.5 \boldmath{$\;a=a_0t^{-4},\;b=b_0t^{-4},\;c=0.$}}

The change of variables
\[T=-1/t,\;X=x/t,\;Y=y/t,\]
brings us back to Case I.2.2. 

{\bf Case I.9 \boldmath{$\;a,\;b,\;c \ne \mbox{const.}$}}

In general the system (\ref{ch5:d24})--(\ref{ch5:d26}) does not have closed form
solution although a power series solution can be obtained. In order to find
specifications of the arbitrary elements leading to the extension of
the principal Lie algebra, let us rewrite (\ref{ch5:d24})--(\ref{ch5:d26}) in the
following form:
\begin{eqnarray}
\left ( 2at+\frac{a'}{2}t^2 \right )c_1+(2at+a't)c_2-s_0c+p_0b &=&0,
\label{ch5:a1}\\
\left ( 2bt+\frac{b'}{2}t^2 \right )c_1+(2bt+b't)c_2 +b'c_3+2s_0a
+(q_0-r_0)b &=&0, \label{ch5:a2}\\
\left ( 2ct+\frac{c'}{2}t^2 \right )c_1+(2ct+c't)c_2
+c'c_3-2ap_0 +(r_0-b_0)c &=&0 . \label{ch5:a3}
\end{eqnarray}
Note that for extension to be brought by a given constant, the choice  of the
arbitrary elements must be such that that constant does not appear
explicitly in (\ref{ch5:a1})--(\ref{ch5:a3}).
Hence the relevant cases are the following ones.
 
{\bf Case I.9.1 \boldmath{$\;a=a_0t^{-4},\; b=b_0t^{-4}\; ,  c=c_0t^{-4}.$} }

The change of variables
\[T=-1/t,\;X=x/t,\;Y=y/t,\]
brings us back to Case I.2. \\

{\bf Case I.9.2 \boldmath{$ \;a=a_0t^{-2},\; b=b_0t^{-2},\;c=c_0t^{-2},\; b_0\ne 0.$} }

Substitution of these expressions into (\ref{ch5:d24}) to (\ref{ch5:d26}) leads
to the following equations:
\[c_1 = 0,\quad c_2=0,\quad p_0 = \frac{c_0}{b_0}s_0,\quad
r_0 = q_0+\frac{2a_0}{b_0}s_0.
\]
The following extensions result:
\[
X_6 = t\dt , \quad
X_7 = \left (\frac{2a_0}{b_0}x+y \right )\dx +\frac{c_0}{b_0}x\dy \;\cdot
\]
{\bf Case II \boldmath{$\;\;b=0.$}}

Without loss of generality, we can assume that $c=0$. Otherwise the change
of variables (which is indeed an equivalence transformation)
\[\bar t= t,\;\;\bar x=y, \;\; \bar y= x, \]
will bring us back to the case $b\ne 0$. Thus the relevant cases are:


{\bf Case II.1  \boldmath{$\;\;b=0=c,\; a$} arbitrary.}
\[X_6 =x\dx\cdot \]

{\bf Case II.2  \boldmath{$\;\;b=0=c,\; a=a_0\ne 0.$} }

We find that:
\[X_6 = x\dx,\;\;X_7 = \dt \cdot \]

{\bf Case II.3  \boldmath {$\;\;b=0=c,\; a\ne \mbox{cont.}$} }

\[ a=a_0(\frac{c_1}{2}t^2+c_2t+c_3)^{-2}. \]
Thus the following subcases up to equivalence transformations
(time translation and scaling) arise:

{\bf Case II.3.1 \boldmath{$ \;\;a=a_0t^{-2}.$}}
\[ X_6 = x\dx ,\quad X_7  = t\dt \cdot \]

{\bf Case II.3.2 \boldmath{$\;a=a_0t^{-4}.$} }

The change transformation
\[T=-1/t,\;X=x/t,\;Y=y/t,\]
brings us back to the Case I.2.2. 

{\bf Case II.3.3 \boldmath{$\;\; a=a_0(t^2\pm 1)^{-2}.$} }
\[ X_6 = x\dx ,\quad  X_7 = (t^2\pm 1)\dt +tx \dx +ty \dy \cdot \]
{\bf Case II.4 \boldmath {$\;\;a=b=c=0.$}}

We obtained in this case the well--known free particle equations. The
additional operators are
\[X_6 = \dt,\quad X_7=x\dx ,\quad X_8 =x\dt, \quad X_9 =y\dt,\]
\[X_{10}=t\dt,\quad X_{11}=y\dx,\quad X_{12}=x\dy ,\quad
X_{13}= t^2\dt +tx \dx +ty \dy,\]
\[X_{14}=tx\dt +x^2 \dx +xy\dy,\quad X_{15}=ty\dt +xy \dx +y^2\dy\; \cdot\]
\section{Discussion and  applications}
From the analysis we have perfomed, the following theorem emerges.
\begin{theo}
\label{ch5:t3}
\begin{em}
A linear system of two second--order odes is reducible to the free particle
equation if and only if it has $15$ symmetries.
\end{em}
\end{theo}
An application of this theorem is provided by the well-known
time--dependent oscillator equation
\[x''=-\omega (t)x,\;\;y''=-\omega (t) y,\]
which appears in numerous applications. Straightforward calculations show
that it has $15$  point symmetries. Hence it can be mapped to the free particle
equation.
The following transformation (cf. Gorringe and Leach 1988):

\begin{equation}
 \bar t=\int^t \rho^{-2}(t) ds,\;\;\bar x=x/\rho (t) ,
\;\;\bar y=y/\rho (t), \label{ch5:os} 
\end{equation}
where $\rho ''+\omega (t) \rho =0$, does the job. 

Furthermore, if a system of two second--order linear odes has $5$ or $6$
point symmetries, it cannot be reduced to a system with constant coefficients.
Also constant coefficients as well as variable coefficients systems can
have $7$ or $8$ symmetries.

It is worth noting that the classification done in Gorringe and Leach (1988)
for the complex constant
coefficients case does not take into account all real cases.
Indeed for a given dimension, there are more nonsimilar real Lie algebras
than there are nonsimilar complex Lie algebras. Our approach is general
(constant as well as variable coefficients are dealt with) and is done in the
real domain.

\section{Conclusion}
We have proved that a system of two second--order odes can have $5$, $6$,
$7$, $8$ or $15$ point symmetries. The maximum number of symmetries being
attained by equations reducible to the free particle equation.
Further, $5$ and $6$ point 
symmetries are the preserve of equations with variable coefficients.

The study of symmetry breaking for systems of $m$ $n$th order linear odes, $m\ge 3$,
$n\ge 2$, remains open.  Note that the difficulty here is the absence of a sort
of optimal canonical form comparable to (\ref{ch5:c7})
. This fact renders the analysis of the determining
equation quite tedious. For $n=2$, only the lower bound, $2m+1$,
and the upper bound, $(m+2)^2-1$, for the number of
symmetries are known (Gonz\'alez-Gasc\'on and Gonz\'alez--L\'opez 1983).

%\end{document}




