%\documentclass[12pt]{report}
%\setlength{\parindent}{0mm}
%\setlength{\parskip}{14pt}
%\renewcommand{\baselinestretch}{1.5}
%\setlength{\topmargin}{0pt}
%\setlength{\headheight}{0pt}
%\setlength{\headsep}{0pt}
%\setlength{\footskip}{45pt}
%\setlength{\textwidth}{465pt}
%\setlength{\textheight}{660pt}
%\setlength{\oddsidemargin}{10pt}
%\newcommand{\RR}{\mathrm{I\!R\!}}
%\newcommand{\FF}{\mathrm{I\!F\!}}
%\newcommand{\dt}{\frac{\partial}{\partial t}}
%\newcommand{\dq}{\frac{\partial}{\partial q }}
%\newcommand{\dr}{\frac{\partial}{\partial p }}
%\newtheorem{defi}{Definition}[chapter]
%\newtheorem{theo}{Theorem}[chapter]
%
%\begin{document}

\chapter{Introduction}
Navier-Stokes Equations, named after Claude-Louis Navier and George Gabriel Stokes,  are nonlinear partial differential equations governing the dynamics of fluids. They arise in several areas of practical importance such as meteorology, aerodynamics, geophysics, and animation in movies or video games. They comprise a set of equations encoding conservation of mass and linear momentum. These equations must be supplemented with appropriate constitutive relations and boundary conditions. The former provides a relation between stress and rate of shear (strain). They allow the classification of fluids into Newtonian and non-Newtonian fluids. Newtonian fluids are characterized by a linear relation between stress and strain whereas non-Newtonian fluids violate such a relation.  The existence, uniqueness, and regularity of solutions of the Navier-Stokes equations under appropriate boundary conditions and data are well-known in 2-D.  In 3-D there are still unsolved problems related to existence, uniqueness, and regularity of solutions of Navier-Stokes Equations. Arriving at the proper set of equations to model the behavior of different fluids required get diligence and hard work from individuals such as Leonhard Euler, Augustin-Louis Cauchy, and Sim�on Denis Poisson to name a few. 

%\end{document}