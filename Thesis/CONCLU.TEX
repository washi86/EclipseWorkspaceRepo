%\documentstyle[12pt]{report}
%\setlength{\parindent}{0mm}
%\setlength{\parskip}{6pt}
%\renewcommand{\baselinestretch}{1.5}
%\setlength{\topmargin}{0pt}
%\setlength{\headheight}{0pt}
%\setlength{\headsep}{0pt}
%\setlength{\footskip}{45pt}
%\setlength{\footheight}{0pt}
%\setlength{\textwidth}{465pt}
%\setlength{\textheight}{660pt}
%\setlength{\oddsidemargin}{10pt}
%\begin{document}

\chapter*{Conclusions and Further Works}
\addcontentsline{toc}{chapter}{\protect\numberline{}{Conclusions and Further Works}}

In this thesis, we have studied the symmetry properties of
ordinary differential equations (odes) with the ultimate
goal of providing some insights on their systematic integration. 

Using Lie's enumeration of irreducible contact transformations of the complex
plane, we were able to completely classify scalar odes with irreducible
contact symmetry algebras. From this classification, we were able to
deduce that up to
contact transformations, $q^{(3)}=0$ is the only third--order ode admitting
a nontrivial contact symmetry algebra. Further, we have shown except for
linear third--order odes with four-- and  five--dimensional point  symmetry
algebras, contact symmetry algebras of third--order odes are subalgebras of
the
ten--dimensional contact symmetry algebra of $q^{(3)}=0$. Moreover,
fourth--order odes were shown to be devoid of nontrivial contact symmetries
and nontrivial odes with irreducible contact symmetry algebras occured
from order $n \ge 5$. 

We have also advocated the use of Noether point symmetries as an alternative
approach for integration. Indeed the invariance of the Noether integral under
the Noether point symmetry associated with it provided an effective
integration procedure. This important property was used for
investigating integrable cases of the equation $y''=f(x)y^n$ which plays
a fundamental role in General Relativity for $n=-1/3,\;2$.
As a result, we have obtained in the case $n=2$ known solutions and a class
of new solutions given parametrically. In the case $n=-1/3$ we re--derived
in a unified way all existing solutions. 
 
A fair portion of this work has been devoted to the investigation of
symmetry properties of systems of odes. We give below an overview of our
findings. 
 
Through symmetry analysis, we have generalized the celebrated Abel--Forsyth
formula to systems of linear homogeneous odes of arbitrary order.
This generalization was applied to projective Riccati equations. As
a result we proved that if $n$ solutions of an $n$--dimensional projective
Riccati equation are known, the general solution can be obtained by quadratures.
Also, we gave a symmetry approach to variation of parameters for
systems.

In studying symmetry properties of systems of two linear second--order odes,
we derived a novel canonical form. The latter was decisive for unravelling
the symmetry structure of these equations. Precisely, we showed that a system
of two second--order linear odes  may admit a $5$--, $6$--, $7$--, $8$-- or
$15$--dimensional point symmetry algebra. This clearly shows that systems
of two second--order linear odes are not equivalent to each other. Hence the
necessity of investigating the linearizability of systems of two
second--order odes arose.

We proved  two linearization criteria. The first one states that for a system
of two second--order odes to be linearizable by means of point transformations,
it is necessary and sufficient that it admits the four--dimensional abelian
Lie algebra. The second one constrained the system to admit
the four--dimensional  Lie algebra with commutator $[X_i,X_j]=0,\;
[X_i, X_4]=X_i,\; i,\;j=1,\;2,\;3$ for linearization. The two criteria
enable us to explicitly construct the linearizing transformations by solving
mere first--order linear pdes. In addition, we extended our criteria
to systems of $n > 2$ second--order odes.

Furthermore, we gave nonsimilar realizations  of three--
and four--dimensional real Lie algebras in $(1+2)$--space. We used this
result to completely classify 
systems of two second--order odes having four--dimensional point symmetry
algebras. Our classification sheded some light on the uncoupling problem and
the integration
of systems of two second--order odes. Concerning the integration, we offered
two different routes as the classical procedure failed.


The above is a basic summary of our study. There still remain open problems.
We discuss a few below. 

It would be interesting to find the role of contact symmetries in the
integration  of scalar odes. We note that integrable scalar odes
admitting contact symmetries can be integrated without using the
contact symmetries. Maybe, contact symmetries for such equations 
express some geometrical properties of the solution curves.

In dealing with the equation $y''=f(x)y^n$, we studied its Noether point
symmetries with respect to its natural Lagrangian. But there may exist
Lagrangians not equivalent to the natural Lagrangian and  which yield richer
symmetry and possibly new integrable cases. We do not expect the search of
such Lagrangians to be trivial.

It would be worthwhile investigating symmetry breaking for general  system of
linear odes. Note that the difficulty here will be to find an appropriate
canonical form comparable to the one we obtained in Chapter 3. Also, the
linearization problem for more general odes needs attention.

Finally  the pertinent problem of complete group classification of systems of two second--order
odes need to be addressed. One angle of attack of this question could be the
group classification of each canonical form we obtained. But this is a brute
force approach. We believe that the most appropriate way to deal with this
issue  is to study real Lie algebras of dimension greater than or equal to
five. Obviously Lie algebras containing  the four--dimensional algebras
leading to linearization should be excluded.


%\end{document}
