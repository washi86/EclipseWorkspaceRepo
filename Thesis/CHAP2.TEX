\documentstyle[12pt]{report}
\setlength{\parindent}{0mm}
\setlength{\parskip}{14pt}
\renewcommand{\baselinestretch}{1.5}
\setlength{\topmargin}{0pt}
\setlength{\headheight}{0pt}
\setlength{\headsep}{0pt}
\setlength{\footskip}{45pt}
\setlength{\footheight}{0pt}
\setlength{\textwidth}{465pt}
\setlength{\textheight}{660pt}
\setlength{\oddsidemargin}{5pt}
\newcommand{\RR}{\mathrm{I\!R\!}}
\newcommand{\FF}{\mathrm{I\!F\!}}
\newcommand{\LL}{{\cal L}}
\newcommand{\dt}{\displaystyle{\frac{\partial}{\partial t}}}
\newcommand{\dq}{\displaystyle{\frac{\partial}{\partial q }}}
\newcommand{\dr}{\displaystyle{\frac{\partial}{\partial p }}}
\newtheorem{defi}{Definition}[chapter]
\newtheorem{theo}{Theorem}[chapter]
\newtheorem{lem}{Lemma}[chapter]
\begin{document}

\chapter{Contact Symmetries of Scalar Ordinary Differential Equations}
Using Lie's classification of irreducible contact transformations of the
complex plane (Lie 1874, Lie and Engel 1890, Campbell 1903, Olver 1994,
Ibragimov 1999),
we show that
a third--order scalar ordinary differential equation (ode)
admits an irreducible contact symmetry algebra if and only if it is reducible
to $q^{(3)}=0$ via a contact transformation. This result coupled with the
classification of third--order odes with point symmetries
(Mahomed and Leach 1988, 1990, Gat 1992, Ibragimov and Mahomed 1996)
provide an explanation of symmetry breaking for third--order odes. Indeed, in
general, the  point symmetry algebra of a third--order ode is not a
subalgebra of the seven--dimensional point symmetry algebra of $q^{(3)}=0$.
However, the contact symmetry algebra of any third--order ode, except for
third--order linear odes with four-- and five--dimensional point
symmetry algebras,  is a subalgebra of the ten--dimensional contact symmetry
algebra of $q^{(3)}=0$.

We also prove that a fourth--order
scalar ode cannot admit an irreducible contact symmetry algebra.
Moreover, we completely classify scalar $n$th--order ($n\ge 5$) odes
which admit nontrivial contact symmetry algebras.
The content of
this chapter is also to be found in Wafo Soh {\em et al} (1999a).


\section{Introduction}
The notion of tangential transfomations was already present in Lie's doctoral
thesis (Lie 1871) in which he found a contact transformation having
the property of mapping straight lines into spheres in space. But the
theoretical foundation of the theory of contact transformations was laid in
Lie (1874) and Lie and Engel (1888, 1890, 1893). Since then, there have been
numerous works on
its applications to geometry (geodesics of a Riemannian space) and
dynamics (Lie 1889, 1896, Vessiot 1906, Levi--Civita and Amaldi 1927,
Whittaker 1927, Eisenhart 1933). Applications to differential equations were
inaugurated by Lie himself: he proved that
for ordinary differential equations (odes) of order $n$, $n \ge 3$, the
contact symmetry algebra is finite--dimensional. He further classified
partial differential equations with two independent variables admitting
contact symmetries (Lie 1881, Ibragimov 1995). More importantly, he gave
the complete classification of finite--dimensional irreducible contact
Lie algebras in two complex variables
(see Lie 1874, Lie and Engel 1890, Campbell 1903, Olver 1994, Ibragimov 1999).
In this chapter, we exploit this result
in order to  classify up to contact transformations, all $n$th--order odes,
$n \ge 3$, admitting nontrivial contact symmetry algebras.

\section{Contact symmetries of second--order odes}
For the sake of completeness, we start by studying second--order odes.
Hereafter, unless otherwise stated, all considerations will be local.

Consider the second--order ode
\begin{equation}
\label{ch2:cs1}
\ddot q=\omega (t,q,\dot q)
\end{equation}
which is equivalent to the system
\begin{equation}
\label{ch2:cs2}
d\dot q-\omega (t,q,\dot q)dt =0,\;\; dq-\dot q dt=0.
\end{equation}
Let $I(t,q,\dot q)$ be a first integral\footnote{Since the equation defining
a first integral of a second--order ode
is a linear partial differential equation, this guarantees the local
existence of a  first integral.}   of Eq. (\ref{ch2:cs1}) and
$J(t,q,\dot q)$ a function  such that
\begin{equation}
\label{ch2:cs3}
J_t+\dot q J_q+\omega (t,q,\dot q)J_{\dot q}=I(t,q,\dot q),
\end{equation}
where the subscripts refer to partial differentiations. Then, the change  of
variables
\begin{equation}
\label{ch2:cs4}
\bar t =t,\;\; \bar q= J(t,q,p),\;\;\bar p=I(t,q,p),
\end{equation}
where $p=\dot q$, is a contact transformation. Indeed
\begin{eqnarray*}
d\bar q-\bar p d\bar t &= & ( J_t dt+J_q dq+ J_pdp)-Idt \\
&=& (J_t dt+pJ_qdt+\omega J_pdt )-Idt\;\;(\mbox{using Eqs. (\ref{ch2:cs2})} )\\
&=& I dt-Idt\;\; (\mbox{by Eq. (\ref{ch2:cs3})})\\
&=& 0.
\end{eqnarray*}
In the new variables, Eq. (\ref{ch2:cs1}) reads
\[ \frac{d^2\bar q}{d {\bar t}^2}=0.\]
Thus we have proved the following theorem.
\begin{theo}
\label{ch2:tc1}
\begin{em}
Up to contact transformations, all scalar second--order odes are equivalent
to the free particle equation.
\end{em}
\end{theo}
{\bf Example.} In order to illustrate Theorem \ref{ch2:tc1}, we consider the
nonlinear equation
\begin{equation}
\label{ch2:cs5}
\ddot q={\dot q}^n,\;\; n\ge 4 .
\end{equation}
Note that Eq. (\ref{ch2:cs5}) cannot be linearized by means of point
transformations (Lie 1883, Tresse 1894, 1896, Mahomed 1989). A first
integral of Eq. (\ref{ch2:cs5}) is
\[I=\frac{p^{1-n}}{1-n}-t. \]
Now Eq. (\ref{ch2:cs3}) becomes
\[J_t+p J_q+p^n J_p=\frac{p^{1-n}}{1-n}-t.\]
This equation has the general solution
\[ J=  tp^{1-n}/(1-n)-t^2 +F\left (p^{1-n}/(1-n)-t,q-p^{2-n}/(2-n)\right ),\]
where $F$ is an arbitrary function. Thus we can take for instance
\[ \bar t=t,\;\; \bar q =q-p^{2-n}/(2-n)+tp^{1-n}/(1-n)-t^2,\;\;
\bar p=p^{1-n}/(1-n)-t.\]


An immediate consequence of Theorem \ref{ch2:tc1} is that the study of contact
symmetries of second--order odes reduces to that of the free particle
equation. This study has been undertaken by numerous authors (Campbell 1903,
Schwarz 1983, Mahomed and Leach 1991) and is summarised in the following
theorem.
\begin{theo}
\label{ch2:tc2}
\begin{em}
The contact symmetry Lie algebra of the free particle equation $\ddot q=0$ is
the infinite--dimensional algebra $sl(3,\RR\;)\oplus_s L^{\infty}$, where
$sl(3,\RR\;)$ is the Lie algebra of the projective group of the plane,
$\oplus_s$ is the semi--direct sum and
$L^{\infty}$ is the infinite--dimensional irreducible Lie algebra of contact
vector fields generated by $W=tU(p,q-pt)+V(p,q-pt)$, where $U$ and $V$ are
functions such that $W$ is not linear in $p$.
\end{em}
\end{theo}
{\bf Proof.} See for instance Campbell (1903), Schwarz (1983),
Mahomed and Leach (1991).

Theorem \ref{ch2:tc1} together with Theorem \ref{ch2:tc2} yield the following one.
\begin{theo}
\begin{em}
The contact symmetry Lie algebra of a nonlinear second--order scalar ode is
isomorphic to the Lie algebra $sl(3,\RR\;)\oplus_s L^{\infty}$
of Theorem \ref{ch2:tc2}.
\end{em}
\end{theo}

\section{Contact symmetries of $n$th--order odes, $n\ge 3$}
From the investigations of Lie and Scheffers (1891), Svirshchevskii (1993, 1995),
Ibragimov {\em et al} (1997) and Yumaguzhin (1997), it appears that
third--order linear
odes reducible to $q^{(3)}=0$ via point transformations  are the only
third--order linear odes admitting irreducible contact symmetry algebras.
So the existence of
nonlinear $n$th--order odes, $n\ge 3$ possessing irreducible contact symmetry
algebras may be questioned. We shall completely tackle this problem in this
section.
\subsection{Some important results}
Let us begin with some well--known fundamental results due to Lie.
\begin{theo}[Lie--Scheffers 1891]
\begin{em}
Let
\begin{equation}
\label{ch2:cs6}
q^{(n)}=\omega (t,q,\dot q,\ldots, q^{(n-1)})
\end{equation}
 be an  $n$th--order scalar
ordinary differential equation. Then the following hold.

(i) If $n=3$, then Eq. (\ref{ch2:cs6}) admits at most a ten--dimensional symmetry
group of contact transformations. Moreover, the symmetry group is
ten--dimensional if and only if Eq. (\ref{ch2:cs6}) is equivalent
(up to a contact transformation) to $q^{(3)}=0.$

(ii) If $n\ge 4$, then Eq. (\ref{ch2:cs6}) admits at most an $(n+4)$--parameter
symmetry group of contact transformations. In addition, the symmetry group
is $(n+4 )$--dimensional if and only if Eq. (\ref{ch2:cs6}) is equivalent to
$q^{(n)}=0$.
\end{em}
\end{theo}
{\bf Proof.}  Consult Lie and Scheffers (1891), Gonz\'alez--Gasc\'on and  Gonz\'alez--L\'opez
(1983).

The  next theorem first appeared in Lie (1874) and was reported in
Lie and Engel (1890).
\begin{theo}[Lie 1874]
\begin{em}
\label{ch2:liec}
Finite--dimensional irreducible Lie algebras of contact transformations of the complex plane
belong to one of the following  three classes modulo contact transformations.\\
(i) $\LL_6$ with generators
\[ X_1=\dt,\;X_2=t\dq+\dr,\;X_3=\dq,\]
\[X_4=\frac{1}{2}t^2\dq+t\dr,\;
X_5=t\dt-p\dr,\;X_6=p\dt+\frac{1}{2}p^2\dq,\]
(ii) $\LL_7$ generated by  $X_1,\ldots,X_6$ and
\[X_7=t\dt+2q\dq+p\dr ,\]
(iii) $\LL_{10}$  with basis formed by $X_1,\ldots,X_7$ and
\[X_8=(q-tp)\dt-\frac{1}{2}tp^2\dq-\frac{1}{2}p^2\dr,\;
  X_9 = \frac{1}{2}t^2\dt+tq\dq+q\dr,\]
\[  X_{10}= (tq-\frac{1}{2}t^2p)\dt+(q^2-\frac{1}{4}t^2p^2)\dq
  +(qp-\frac{1}{2}tp^2)\dr \; \cdot \]
\end{em}
\end{theo}

{\bf Proof.} See Lie (1874), Lie and Engel (1890), Campbell (1903).\\ 

\subsection{Third--order odes}
{\bf (a) Contact symmetries} 

Consider a third--order ode
\begin{equation}
\label{ch2:cs8}
q^{(3)}=\omega (t,q,\dot q,\ddot q)
\end{equation}
invariant under the Lie algebra $\LL_6$ of Theorem \ref{ch2:liec}. Invariance under
$X_1$ and $X_3$ imply that $\omega$ is independent of $t$ and $q$. Hence Eq.
(\ref{ch2:cs8}) reduces to
\begin{equation}
\label{ch2:cs9}
q^{(3)}=\omega (\dot q,\ddot q).
\end{equation}
Imposing the symmetry $X_6$ on Eq. (\ref{ch2:cs9}), we find that $\omega$
must satisfy
\[-3\ddot q \omega =-\ddot q^2 \omega_{\ddot q},\]
i.e.
\[\omega=\ddot q^3J(\dot q),\]
where $J$ is an arbitrary function. Now Eq. (\ref{ch2:cs9}) becomes
\begin{equation}
\label{ch2:cs10}
q^{(3)}=\ddot q^3J(\dot q).
\end{equation}
The invariance of  Eq. (\ref{ch2:cs10}) under $X_2$  forces $J$ to satisfy
\[\dot q J'+3J=0,\]
where the prime stands for differentiation with respect to $\dot q$. Thus
\[J=\frac{C}{\dot q^{3}},\]
where $C$  is an arbitrary constant of integration. Finally, by imposing the
symmetry $X_4$ on the resulting equation, we find that $C=0$. In summary,
the third--order ode admitting $\LL_6$ is $q^{(3)}=0$. Note that $\LL_6$ is a
subalgebra of $\LL_7$ and $\LL_{10}$. Furthermore, $X_7,\ldots ,X_{10}$ are also admitted
by $q^{(3)}=0$. We have thus proved the theorem:
\begin{theo}
\label{ch2:3cont}
\begin{em}
A third--order ode admits an irreducible contact symmetry algebra if and only
if it can be mapped to the simplest equation $q^{(3)}=0$ by means of contact
transformations.
\end{em}
\end{theo}
{\bf Example.} (Lie and Scheffers 1896, Campbell 1903, Ibragimov and
Mahomed 1996). 

(i) The  differential equation satisfied by all circles of the plane is
\begin{equation}
\label{ch2:cs11}
q^{(3)}=\frac{3\dot q \ddot q^2}{1+\dot q^2}\;\cdot
\end{equation}
Under the contact transformation
\begin{equation}
\label{ch2:cs12}
t= \frac{-\bar t}{2\bar p},\;\; \bar q=q-\frac{1}{2}\bar t\bar p,\;\;
p=-{\bar p}^2,
\end{equation}
Eq. (\ref{ch2:cs11}) reduces to $\bar q^{(3)}=0$. Hence Eq. (\ref{ch2:cs11}) admits a
ten--dimensional contact symmetry algebra. 

(ii) The equation describing  hyperbolas of the plane is
\begin{equation}
\label{ch2:cs13}
q^{(3)}=\frac{3\ddot q^2}{2\dot q}.
\end{equation}
The contact transformation\footnote{The contact transformation
$\bar t= p,\;\; \bar q=tp-q,\;\; \bar p=t$ is the well-known Legendre
transformation.}
\begin{equation}
\label{ch2:cs14}
\bar t=\sqrt p,\;\; \bar q=tp-q,\;\; \bar p=t,
\end{equation}
maps Eq. (\ref{ch2:cs13}) to $\bar q^{(3)}=0$.\\
Note that the complex  transformation
\begin{equation}
\label{ch2:cs15}
t=\bar q+i\bar t,\;\; q=\bar t+i\bar q,
\end{equation}
maps Eq. (\ref{ch2:cs11}) to Eq. (\ref{ch2:cs13}) (see Ibragimov and
Mahomed 1996).

{\bf (b) Symmetry breaking}

Its well-known that point symmetry Lie algebras of second--order odes
are subalgebras of the eight--dimensional point symmetry algebra
$sl(3,\RR\;)$, of the free particle equation $\ddot q=0$. For third--order
odes  the classification of point symmetry Lie algebras given
in Table 1.1  shows that this result does not apply.
The main difference between scalar second--order linear odes and
higher--order linear odes is that  for $n\ge 3$ there are three
classes of scalar $n$th--order linear
odes (see Mahomed 1989 and Mahomed and Leach 1990). This fact considerably
affects the symmetry breaking problem as we shall see.

Note that in Table 1.1 we have listed only realizations in the complex domain
that are admitted by third--order odes (Mahomed 1989, Gat 1992, Ibragimov
and  Mahomed 1996).  A close analysis of this table shows that all the
equations listed except for those admitting $L_6$ and $L_7$ admit
reducible contact symmetry algebras. Precisely, their contact symmetry Lie
algebras are mere first prolongations of their point symmetry algebras. The
equation  admitting $L_6$ admits an irreducible
ten--dimensional contact symmetry algebra (cf. Lie and Scheffers 1896,
Campbell 1903). From
Theorem \ref{ch2:3cont}, we deduce that the equation is reducible via contact
transformations to
$q^{(3)}=0$. Further, except for equations admitting $L_{4,1}$ and $L_5$, the
symmetries of the other equations are also symmetries of $q^{(3)}=0$.
To see this
we need to examine only the nontrivial cases $L_{3,6}^{II}$, $L_{3,6}^{III}$,
$L_{4,4}$, $L_{4,5}^I$ and $L_6$. The last case has already been dealt with. 

The Lie algebra $L_{4,5}^I$
is a subalgebra of $L_6$ and we have seen that the equation admitting $L_6$ is
reducible to $q^{(3)}=0$. Thus  the operators of $L_{4,5}^I$ written
in appropriate coordinates  are symmetries of $q^{(3)}=0$.

The equation admitting $L_{3,6}^{II}$, $L_{3,6}^{III}$ or $L_{4,4}$  can be
considered as a generalization of Eq. (\ref{ch2:cs11}) and we have seen that this
equation is reducible via contact transformations to $q^{(3)}=0$. Hence  
the generators of $L_{3,6}^{II}$, $L_{3,6}^{III}$ or $L_{4,4}$ written in
appropriate variables are symmetries of $q^{(3)}=0$. 

In summary, we have proved the following theorem.
\begin{theo}
\label{ch2:3break}
\begin{em}
Except for linear equations admitting a maximal four-- or
five--dimensional point
symmetry Lie algebra, the contact symmetry algebra of any third--order ode is
a subalgebra of the ten--dimensional contact symmetry algebra of the simplest
equation $q^{(3)}=0$.
\end{em}
\end{theo}

\subsection{Fourth--order odes}
Here, we explore the existence of irreducible contact symmetry algebras
for fourth--order odes.

Consider a fourth-order ode
\begin{equation}
\label{ch2:cs16}
q^{(4)}=\omega (t,q,\dot q,\ddot q, q^{(3)}).
\end{equation}
By imposing the symmetries $X_1,\ldots , X_5$ of Theorem \ref{ch2:liec}
on Eq. (\ref{ch2:cs16}), we are led to
\begin{equation}
\label{ch2:cs17}
q^{(4)}=C \left [ q^{(3)} \right ]^{4/3},
\end{equation}
where $C$ is an arbitrary constant. Straightfoward calculations show that
Eq. (\ref{ch2:cs17}) does not admit $X_6$. Indeed, it is easy to verify that

\[X_6^{[4]}\left ( q^{(4)}-C \left [ q^{(3)} \right ]^{4/3}\right )
|_{(\ref{ch2:cs17})}=6\left [ q^{(3)}\right ]^2 \ne 0.\]

Thus there is no fourth--order ode
invariant under $\LL_6$. Since $\LL_6$ is a subalgebra of $\LL_7$ and
$\LL_{10}$, we
deduce the theorem:
\begin{theo}
\label{ch2:4cont}
\begin{em}
Fourth-order scalar odes do not  admit irreducible contact symmetry
algebras.
\end{em}
\end{theo}
\subsection{$n$th--order odes, $n \ge 5$} 
In this section, we are concerned with the existence of $n$th--order odes,
$n\ge 5$, admitting irreducible contact symmetry algebras. Contrary to
the cases previously treated (third-- and fourth--order), we shall see that
there are genuine nonlinear $n$th--order odes, $n\ge 5$, that admit
irreducible contact symmetry algebras. 

Let us start with the analysis of fifth--order odes. Consider the general
fifth--order ode
\begin{equation}
\label{ch2:cs18}
q^{(5)}=\omega (t,q,\dot q,\ddot q, q^{(3)},q^{(4)}).
\end{equation}
The invariance of Eq. (\ref{ch2:cs18}) under the operators $X_1,\;X_2,\;X_3$ and
$X_4$ implies that $\omega =\omega (q^{(3)},q^{(4)})$. Further, by requiring
the invariance of Eq. (\ref{ch2:cs18}) with respect to $X_5$ and $X_6$, we obtain
the ode

\begin{equation}
\label{ch2:cs19}
q^{(5)}=\frac{5}{3}\frac{\left [q^{(4)}\right ]^2 }{q^{(3)}}
+C\left [q^{(3)}\right ]^{5/3},
\end{equation}
where $C$ is an arbitrary constant. If we  impose $X_7$  on Eq. (\ref{ch2:cs19}),
we find that $C=0$. Thus  the equation
\begin{equation}
\label{ch2:cs20}
q^{(5)}=\frac{5}{3}\frac{\left [q^{(4)}\right ]^2 }{q^{(3)}} ,
\end{equation}
admits $\LL_7$. It is easy to check that Eq. (\ref{ch2:cs20}) does not admit
$\LL_{10}$. Indeed, after cumbersome albeit simple
calculations we find that

\[ X_9^{[5]}\left (
q^{(5)}-\frac{5}{3}\frac{\left [q^{(4)}\right ]^2 }{q^{(3)}} \right )
|_{(\ref{ch2:cs20})}=-\frac{10}{3}q^{(4)} \ne 0. \]
In summary we have proved the following:
\begin{theo}
\label{ch2:5cont}
\begin{em}
There are only two classes of fifth--order odes admitting irreducible contact
symmetry algebras. A representative of the first class is Eq. (\ref{ch2:cs19}) which
admits the six--dimensional Lie algebra $\LL_6$ and a representative of the
second is Eq. (\ref{ch2:cs20}) which admits $\LL_7$.
\end{em}
\end{theo}
In order to simplify the notation, from now on, the $n$th derivative of $q$
with respect to $t$, viz. $q^{(n)}$, will be denoted simply by $q_n$. 

Consider the most general sixth--order ode
\begin{equation}
q_6=F(t,q,q_1,q_2,q_3,q_4,q_5).
\label{ch2:cs21}
\end{equation}
The invariance of Eq. (\ref{ch2:cs21}) under $X_1,\; X_2,\; X_3$ and $X_4$
implies $F$ does not depend on $t$, $q$, $q_1$ and $q_2$, so 
\begin{equation}
F=G(q_3,q_4,q_5).
\label{ch2:cs22}
\end{equation}
Moreover, the invariance of Eq. (\ref{ch2:cs21}) under $X_5$ restricts
$G$ to
\begin{equation}
G=(q_3)^{2}H \left (q_3^{-1}q_4^{\frac{3}{4}},q_3^{-1}q_5^{\frac{3}{5}}
\right ),
\label{ch2:cs23}
\end{equation}
where $H$ is an arbitrary function of its argument.
The further invariance of Eq. (\ref{ch2:cs21}) under $X_6$ implies
that 
\begin{equation}
\label{ch2:cs24}
H=-\frac{40}{9}q_3^{-4}q_4^3+5q_3^{-3}q_4q_5+P(s),
\end{equation}
where $P$ is an arbitrary function of $s$ with
$s=q_3^{-\frac{8}{3}}q_4^2-\frac{3}{5}q_3^{-\frac{5}{3}}q_5$.
Hence we arrive at the sixth-order ode with six--dimensional
contact symmetry algebra given by
\begin{equation}
q_6=5q_3^{-1}q_4q_5-\frac{40}{9}q_3^{-2}q_4^3+q_3^2P(s).
\label{ch2:cs25}
\end{equation}
If we further assume that Eq. (\ref{ch2:cs25}) is invariant under
$X_7$, then $P$  becomes
\[P(s)=Cs^{\frac{3}{2}}.\]
Thus the sixth-order ode admitting $\LL_7$ is
\begin{equation}
q_6=5q_3^{-1}q_4q_5-\frac{40}{9}q_3^{-2}q_4^3
+Cq_3^{-2} \left (q_4^2-\frac{3}{5}q_3q_5 \right )^{\frac{3}{2}}.
\label{ch2:cs26}
\end{equation}
A tedious but  simple calculation shows that there is no $C$ for which
Eq. (\ref{ch2:cs26}) is invariant
under $X_9$. This implies that there is no sixth--order equation
admitting $\LL_{10}$. Therefore, we have shown the following:
\begin{theo}
\label{ch2:6cont}
\begin{em}
There are only two classes of sixth--order odes admitting irreducible contact
symmetry algebras. A representative of the first class is Eq. (\ref{ch2:cs25})
which
admits the six--dimensional Lie algebra $\LL_6$ and a representative of the
second is Eq. (\ref{ch2:cs26}) which admits $\LL_7$.
\end{em}
\end{theo}
Next, we deal with seventh--order odes. Consider the seventh--order ode
\begin{equation}
q_7=F(t,q,q_1,q_2,q_3,q_4,q_5,q_6).
\label{ch2:cs27}
\end{equation}
If the generators $X_1$, $X_2$, $X_3$, $X_4$ and $X_5$
leave invariant Eq. (\ref{ch2:cs27}), this implies
that $F$ has the form
\begin{equation}
F=q_3^{\frac{7}{3}}G(\lambda,\kappa,\mu),
\label{ch2:cs28}
\end{equation}
where $\lambda=q_3^{-1}q_4^{\frac{3}{4}}$,
$\kappa=q_3^{-1}q_5^{\frac{3}{5}}$, $\mu=q_3^{-1}q_6^{\frac{1}{2}}$.
Furthermore, Eq. (\ref{ch2:cs27}) with $F$ given by Eq. (\ref{ch2:cs28})
is invariant under $X_6$ if and only if
\begin{equation}
G=\frac{35}{3}\lambda^{\frac{16}{3}}+14z\lambda^{\frac{14}{3}}
-\frac{350}{9}y\lambda^{\frac{8}{3}}+H(y,z),
\label{ch2:cs29}
\end{equation}
where $y=\lambda^{\frac{8}{3}}-\frac{3}{5}\kappa^{\frac{5}{3}}$,
$z=\frac{\mu^2}{2}+\frac{20}{9}\lambda^4-\frac{5}{2}\lambda^{\frac{4}{3}}
\kappa^{\frac{5}{3}}$. Hence the seventh--order ode admitting $\LL_6$ 
is
\begin{equation}
q_7=q_3^{\frac{7}{3}} \left [\frac{35}{3}\lambda^{\frac{16}{3}}+
14z\lambda^{\frac{14}{3}}-\frac{350}{9}y\lambda^{\frac{8}{3}}+H(y,z)\right ].
\label{ch2:cs30}
\end{equation}
By requiring Eq. (\ref{ch2:cs30}) to be invariant under $X_7$, $H$ is restricted
to
\[H=y^2Q(z^{-1}y^{\frac{3}{2}}).\]
So, the seventh--order ode admitting $\LL_7$ is 
\begin{equation}
q_7=\frac{35}{9}q_3^{-3}q_4^4+7q_3^{-1}q_4q_6-\frac{35}{3}q_3^{-2}q_4^2q_5
\nonumber \\
+(q_3^{-3}q_4^4+\frac{9}{25}q_3^{-1}q_5^2-\frac{6}{5}q_3^{-2}q_4^2q_5)Q(s),
\label{ch2:cs31}
\end{equation}
where $s=(\frac{1}{2}q_3^2q_6+\frac{20}{9}q_4^3-\frac{5}{2}q_3q_4q_5)^{-1}
(q_4^2-\frac{3}{5}q_3q_5)^{\frac{3}{2}}$ and
$Q$ is an arbitrary function of $s$.
Finally, to get seventh-order odes admitting
$\LL_{10}$, it suffices to consider the invariance of
Eq. (\ref{ch2:cs31}) under $X_8,\;X_9$ and $X_{10}$. The invariance under
$X_9$ implies that $Q(s)$ satisfies
\begin{equation}
\left [\frac{5}{3}s^2(q_4^2-\frac{3}{5})^{\frac{1}{2}}-3s\right ]
Q_s-4Q+\frac{490}{9}=0.
\label{ch2:cs32}
\end{equation}
The solution of Eq. (\ref{ch2:cs32}) is 
\[Q=\frac{490}{36}\;\cdot\]
Hence we get the equation
\begin{equation}
q_7=7q_3^{-1}q_4q_6+\frac{35}{2}q_3^{-3}q_4^4+\frac{49}{10}q_3^{-1}
q_5^2-28q_3^{-2}q_4^2q_5.
\label{ch2:cs33}
\end{equation}
Routine calculations show that Eq. (\ref{ch2:cs33})
also admits $X_8$ and $X_{10}$. In conclusion we have proved that:
\begin{theo}
\label{ch2:7cont}
\begin{em}
There are  three classes of seventh--order odes admitting
irreducible contact
symmetry algebras. A representative of the first class is Eq. (\ref{ch2:cs30})
which
admits the six--dimensional Lie algebra $\LL_6$, a representative of the
second is Eq. (\ref{ch2:cs31}) which admits $\LL_7$ and a representative of the
third is Eq. (\ref{ch2:cs33}) which admits $\LL_{10}$.
\end{em}
\end{theo}
Consider now the general eighth--order ode
\begin{equation}
q_8=F(t,q,q_1,q_2,q_3,q_4,q_5,q_6,q_7).
\label{ch2:cs34}
\end{equation}
The invariance of Eq. (\ref{ch2:cs34})  under $X_1,X_2,\cdots, X_6$
constrains $F$ to
\begin{equation}
F=\frac{28}{3}q_3^{-1}q_4q_7-\frac{70}{3}q_3^{-2}q_4^2q_6
+\frac{35}{3}q_3^{-2}q_4q_5^2-\frac{70}{3}q_3^{-3}q_4^3q_5\nonumber\\
+\frac{280}{9}q_3^{-4}q_4^5+q_3^{\frac{8}{3}}G(\lambda,\kappa,\mu),
\label{ch2:cs35}
\end{equation}
where $G$ is an arbitrary function of the indicated variables, and
\[ \lambda=q_3^{-\frac{8}{3}}(q_4^2-\frac{3}{5}q_3q_5),\;\;
\kappa=\frac{1}{2}q_3^{-2}q_6+\frac{20}{9}q_3^{-4}q_4^3-
\frac{5}{2}q_3^{-3}q_4q_5,\]
\[ \mu=q_3^{-\frac{7}{3}}q_7-\frac{35}{9}q_3^{-\frac{16}{3}}q_4^4
-7q_3^{-\frac{10}{3}}q_4q_6+\frac{35}{3}q_3^{-\frac{13}{3}}q_4^2q_5.\]
Moreover, by requiring Eq. (\ref{ch2:cs34}), with  $F$ given by
Eq. (\ref{ch2:cs35}), to be invariant with respect to $X_7$, we obtain
\begin{equation}
G=q_3^{-\frac{20}{3}}(q_4^2-\frac{3}{5}q_3q_5)^{\frac{5}{2}}H(y,z),
\label{ch2:cs36}
\end{equation}
where is arbitrary and
\[ y=(q_4^2-\frac{3}{5}q_3q_5)^{-\frac{3}{2}}(\frac{1}{2}q_3^2q_6
+\frac{20}{9}q_4^3-\frac{5}{2}q_3q_4q_5),\]

\[z=(q_4^2-\frac{3}{5}q_3q_5)^{-2}(q_3^2q_7-7q_3^2q_4q_6
+\frac{35}{3}q_3q_4^3q_5+\frac{280}{9}q_4^4).\]

Thus we get the eighth--order ode with seven-dimensional contact
symmetry algebra given by
\begin{equation}
q_8=\frac{28}{3}q_3^{-1}q_4q_7-\frac{70}{3}q_3^{-2}q_4^2q_6
+\frac{35}{3}q_3^{-2}q_4q_5^2-\frac{70}{3}q_3^{-3}q_4^3q_5\nonumber\\
+\frac{280}{9}q_3^{-4}q_4^5+q_3^{-4}(q_4^2
-\frac{3}{5}q_3q_5)^{\frac{5}{2}}H(y,z).
\label{ch2:cs37}
\end{equation}
Finally, Eq. (\ref{ch2:cs37}) does not admit $\LL_{10}$ for any choice of $H$.
Indeed it cannot be invariant under $X_9$. To see this, note that the
invariance under $X_9$ requires that $H$  satisfies 
the following system of first-order partial differential equations.

\begin{equation}
H_y-\frac{8}{5}z+\frac{700}{9}=0,\;\;\nonumber\\
3yH_y+(4z-\frac{490}{9})H_z-5H-112y=0.
\label{ch2:cs38}
\end{equation}

It is easy to see that Eqs. (\ref{ch2:cs38}) have no solutions.
Indeed, by integrating the first equation of (\ref{ch2:cs38}), we get 

\begin{equation}
H=\frac{8}{5}yz-\frac{700}{9}y+P(z),
\label{ch2:cs39}
\end{equation}
where $P(z)$ is an arbitrary function. The substitution of
Eq. (\ref{ch2:cs39}) into the second equation of (\ref{ch2:cs38}) implies that
$P$ satisfies 

\[(4y-\frac{490}{9})P'-5P+\frac{16}{5}yz-\frac{446}{9}y=0,\]

which is clearly impossible. In conclusion we have established that:

\begin{theo}
\label{ch2:8cont}
\begin{em}
There are  two classes of eighth--order odes admitting
irreducible contact
symmetry algebras. A representative of the first class is Eq. (\ref{ch2:cs34})
with $F$ given by Eq. (\ref{ch2:cs35}) which admits the six--dimensional
Lie algebra $\LL_6$ and  a representative of the
second is Eq. (\ref{ch2:cs37}) which admits $\LL_7$.
\end{em}
\end{theo}
To complete the classification, we next deal with $n$th--order odes,
$n \ge 9 $. Direct calculations in this case are quite cumbersome. We shall
thus use the calculation of fundamental invariants and invariant one--forms
of the Lie algebras $\LL_6,\;\LL_7$ and $\LL_{10}$ given in Olver (1994).

\begin{theo}
\label{ch2:ncont}
\begin{em}
For $n$th--order odes, $n \ge 9$, there are three classes of equations with
irreducible contact symmetry algebras  corresponding to $\LL_6,\;\LL_7,\;
\LL_{10}$ respectively.

(i) A representative of the first class is
\begin{equation}
\label{ch2:cs40}
F(I, \Gamma [I],\ldots , \Gamma^{n-5}[I])=0,
\end{equation}
where
\[ I=q_3^{-8/3} (3q_3q_5-5q_4^2),\;\;\Gamma=q_3^{-1/3}\frac{d}{dt}\;\cdot\]
(ii) A representative of the second is
\begin{equation}
\label{ch2:cs42}
F(I, \Gamma [I],\ldots ,\Gamma^{n-6}[I])=0,
\end{equation}
where
\[I=(3q_3q_5-5q_4^2)^{-3/2}(9q_3^2q_6-45q_3q_4q_5+40q_4^3),\]
\[\Gamma=q_3(3q_3q_5-5q_4^2)^{-1/2}\frac{d}{dt}\;\cdot\]
(iii) A representative of the third is
\begin{equation}
\label{ch2:cs43}
F(I, \Gamma [I],\ldots , \Gamma^{n-9}[I])=0,
\end{equation}
where
\[I=J^{-5/2}K,\;\;\Gamma=q_3J^{-1/4}\frac{d}{dt},\]
\[J=10q_3^7-70q_3^2q_4q_6-49q_3^2+280q_3q_4^2q_5-175q_4^4,\]
\[K=q_3^2[JD^2_tJ-\frac{9}{8}(D_t J)^2]+q_3q_4JD_tJ-\frac{4}{5}
(7q_3q_5-5q_4^2)J^2,\;\;D_t\equiv d/dt.\]

\end{em}
\end{theo}
{\bf Proof.} The $I$s appearing in the theorem are the fundamental invariants
and the $\Gamma$s  are operators of invariant differentiation
(Ovsiannikov 1982)
of the corresponding Lie algebras. All these were given in Olver (1994).
The result
of the theorem follows then from Lie's classical invariant
representation of differential equations (Lie 1884).

In summary, from the above discussion, we infer that fourth--order odes
do not admit irreducible contact Lie algebras and that nontrivial odes with
irreducible contact symmetry algebras occur from order $n\ge 5$ only.

\begin{table}
\label{ch2:tab1}
\begin{center}
\caption{Point symmetry algebras of third--order odes in the complex domain} 
\begin{tabular}{|l|l|l|}
\hline
Algebras  & Realizations in the complex plane & Representative Equations \\
\hline \hline
$ L_1$  &  $X_1=\dt$ & $q^{(3)}=f(q,\dot q,\ddot q)$ \\
\hline
$ L_{2,1}^I$ &  $X_1=\dt,\;\;X_2=\dq$ & $q^{(3)}=f(\dot q,\ddot q)$  \\
\hline
$L_{2,1}^{II}$  & $X_1=\dq,\;\;X_2=t\dq$ & $q^{(3)}=f(t,\ddot q)$\\
\hline
$L_{2,2}^{I}$ & $X_1=\dq,\;\;X_2=t\dt+q\dq$ &
$q^{(3)}=\ddot q^2 f(\dot q,t\ddot q)$\\
\hline
$L_{2,2}^{II}$ & $X_1=\dq,\;\;X_2=q\dq$ & $q^{(3)}=\dot q f(t,\ddot q/\dot q)$ \\
\hline
$L_{3,1}$ & $X_1=\dq,\;\;X_2=\dt,\;\;X_3=t\dq$ & $q^{(3)}=f(\ddot q)$\\
\hline
$L_{3,2}^I$ & $X_1=\dq,\;\;X_2=\dt,$ &
$q^{(3)}=\ddot q^2f(\ddot q\exp \dot q)$\\
& $X_3=t\dt+(t+q)\dq$    &       \\
\hline
$L_{3,2}^{II}$ & $X_1=\dq,\;\;X_2=t\dq,\;\;X_3=\dt+q\dq$ &
$q^{(3)}\ddot q=f(\exp t/\ddot q)$ \\
\hline
$L_{3,3}^I $ & $X_1=\dt,\;\;X_2=\dq,\;\;X_3=t\dt$ &
$q^{(3)}=\ddot q ^{3/2}f(\ddot q\dot q^{-2})$\\
\hline
$L_{3,3}^{II} $ & $X_1=\dq,\;\;X_2=t\dq,\;\;X_3=t\dt+q\dq$ &
$q^{(3)}=\ddot q^2f(t\ddot q)$\\
\hline
$L_{3,4} $ & $X_1=\dt,\;\;X_2=\dq,\;\;X_3=t\dt+q\dq$  &
$q^{(3)}=\ddot q^2f(\dot q)$\\
\hline
$L_{3,5}^{I,a} $ & $X_1=\dt,\;\;X_2=\dq,$ &
$q^{(3)}=\ddot q^{(a-3)/(a-2)}f(\ddot q^{1-a}\dot q^{a-2}),\;a\ne 0,1,2;$\\
 & $X_3=t\dt+aq\dq$ & $q^{(3)}=\dot q^{-1}f(\ddot q),\; a=2$ \\
\hline
$L_{3,5}^{II,a} $ & $X_1=\dq,\;\;X_2=t\dq,$ &
$q^{(3)}=\ddot q^{(3a-2)/(2a-1)}f(t^{1-2a}\ddot q^{1-a}),$\\
 & $X_3=(1-a)t\dt+q\dq$  &  $\;a\ne 0,1,1/2;$\\
 &  & $q^{(3)}=t^{-1}f(\ddot q),\; a=1/2$ \\
\hline
$L_{3,6}^I$ & $ X_1=\dq,\;\;X_2=t\dt+q\dq,$ &
$ q^{(3)}= 3\ddot q^2\dot q^{-1}
+t^{-2}\dot q^4 f[(2t\ddot q+\dot q)\dot q^{-3}]$\\
& $X_3=2tq\dt+q^2\dq$ &  \\
\hline
$L_{3,6}^{II}$ & $ X_1=\dq,\;\;X_2=t\dt+q\dq,$ &
$ q^{(3)}=3\dot q\ddot q^2 (1+\dot q^2)^{-1}$\\
& $X_3=2tq\dt+(t^2+q^2)\dq$  & $+t^{-2}(1+\dot q^2)^2
f[(t\ddot q-\dot q-\dot q^3)(1+\dot q^2)^{-3/2}]$ \\
\hline
$L_{3,6}^{III}$ & $ X_1=\dq,\;\;X_2=q\dq,\;\;X_3=q^2\dq$ &
$q^{(3)}=\displaystyle{\frac{3}{2}\frac{\ddot q^2}{\dot q}}+\dot qf(t)$ \\
\hline
\end{tabular}
\end{center}
\end{table}

\setcounter{table}{0}
\begin{table}
\label{ch2:tab2}
\begin{center}
\caption{Point symmetry algebras of third--order odes in the complex domain
(continued)}
\begin{tabular}{|l|l|l|}
\hline
Algebras  & Realizations in the complex plane & Representative Equations \\
\hline \hline
$L_{4,1}$ & $X_1=\dq,\;\;X_2=t\dq,$ &
$q^{(3)}= \left ( h^{(3)}(t)/\ddot h \right )\ddot q,\;\; h^{(3)}\ne 0$\\
& $X_3=h(t)\dq,\;\;X_4=q\dq$ &  \\
\hline
$L_{4,2}$ & $X_1=\dt,\;\;X_2=\dq,$ &
$q^{(3)}=C\ddot q^2/\dot q,\;\;C\ne 0,3/2$\\
& $X_3=q\dq,\;\;X_4=t\dt$ & \\
\hline
$L_{4,3}^b$ & $X_1=\dq,\;X_2=\dt,$ &
$ q^{(3)}=C\ddot q^{(b-2)/(b-1)},\;b\ne 1,\;2;\;C\ne 0$ \\
& $X_3=t\dq,\;X_4=t\dt+(1+b)q\dq$ & \\
\hline
$L_{4,4}$ & $X_1=\dq,\;X_2=\dt,$ &
$q^{(3)}=3\dot q\ddot q^2(1+\dot q^2)^{-1}+C\ddot q^2 (1+\dot q^2)^{-1},\;
C\ne 0 $\\
& $X_3=t\dt+q\dq,\;X_4=q\dt-t\dq$ & \\
\hline
$L_{4,5}^I$ & $X_1=\dt,\;X_2=\dq,$ &
$ q^{(3)}=\displaystyle{\frac{3}{2}\frac{\ddot q^2}{\dot q}}
+C\dot q,\;\; C\ne 0$\\
& $X_3=q\dq,\;X_4=q^2\dq$   &   \\
\hline
$L_{4,5}^{II}$ & $X_1=\dq,\;X_2=t\dt+q\dq,$ &
$ q^{(3)}=\displaystyle{\frac{3}{2}\frac{\ddot q^2}{\dot q}
+C\frac{(2t\ddot q+\dot q)^{3/2}}
{t^2\sqrt{\dot q}}}$\\
& $\;X_3=2tq\dt+q^2\dq,\;X_4=t\dt$ & \\
\hline
$L_{4,6}$ & $X_1=\dq,\;X_2=t\dq,$ &
$q^{(3)}=C\exp (-\ddot q)$ \\
& $X_3=\dt,\;X_4=t\dt+(2q+t^2/2)\dq$ & \\
\hline
$L_{5}$ & $ X_1=\dt,\;X_2=q\dq,\;X_i=\eta_i (t)\dq,$ &
$ q^{(3)}+k\dot q+lq=0$\\
 & where the $\eta_i$s are independent & \\
 & solutions of  & \\
 & $\eta '''+k\eta '+l\eta=0,\;\;k,\;l=\mbox{ const.}$ & \\
\hline
$L_6$ & $X_1=\dt,\;X_2=t\dt,\;X_3=t^2\dt,$ & $q^{(3)}=\displaystyle{
\frac{3}{2}\frac{\ddot q^2}{\dot q}}$\\
 & $X_4=\dq,\;X_5=q\dq,\;\;X_6=q^2\dq$ &    \\
\hline
$L_7$ & $X_1=q\dq,\;X_2=\dq,\;X_3=t\dq,$ & \\
& $\;X_4=t^2\dq,\;X_5=\dt$ & $q^{(3)}=0$ \\ 
   & $\;X_6=t\dt,\; X_7=t^2\dt+2tq\dq$ & \\
\hline
\end{tabular}
\end{center}
\end{table}


\end{document}


