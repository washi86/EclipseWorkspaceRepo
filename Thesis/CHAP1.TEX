\documentstyle[12pt]{report}
\setlength{\parindent}{0mm}
\setlength{\parskip}{14pt}
\renewcommand{\baselinestretch}{1.5}
\setlength{\topmargin}{0pt}
\setlength{\headheight}{0pt}
\setlength{\headsep}{0pt}
\setlength{\footskip}{45pt}
\setlength{\footheight}{0pt}
\setlength{\textwidth}{465pt}
\setlength{\textheight}{660pt}
\setlength{\oddsidemargin}{10pt}
\newcommand{\RR}{\mathrm{I\!R\!}}
\newcommand{\FF}{\mathrm{I\!F\!}}
\newcommand{\dt}{\frac{\partial}{\partial t}}
\newcommand{\dq}{\frac{\partial}{\partial q }}
\newcommand{\dr}{\frac{\partial}{\partial p }}
\newtheorem{defi}{Definition}[chapter]
\newtheorem{theo}{Theorem}[chapter]
\begin{document}

\chapter{Symmetries of Differential Equations}

This preliminary chapter is concerned with the introduction of basic concepts
about symmetry of differential equations. The material discussed is designed
to suit our needs in forthcoming chapters. For an exhaustive treatment of the
subject the reader is referred to Lie and Scheffers (1891, 1896),
Lie and Engel (1893), 
Ovsiannikov (1982), Olver (1986), Bluman and Kumei (1989), Stephani (1989),
Hill (1992) and recently
Ibragimov (1999). Also, a large and up to date collection of existing results
in symmetry analysis is given in the hanbook edited by Ibragimov (1994--1996).

\section{One--parameter group of transformations}
In what follows here and in the chapters to come, by a transformation it will
be understood  to mean an invertible transformation,
i.e. a one--to--one and onto map. Also, functions will be assumed
sufficiently smooth and summation over repeated indices is adopted when
necessary.

Now consider a set of transformations $T_a$ of $\RR^{\;n}$  defined by:
\begin{equation}
\label{ch1:t1}
\bar z^i=f^i(z,a),\;\;i=1,\ldots,n,                               
\end{equation}
where $z=(z^1,\ldots,z^n)\in \RR^{\;n}$,
$\bar z=(\bar z^1,\ldots,\bar z^n)\in \RR^{\;n}$
and $a$ is a real parameter from a
neighbourhood $U\subset \RR\;$ of $a=0$.
We impose the condition  that $T_a$ is the identity
if and only if $a=0 $, i.e.
\begin{equation}
\label{ch1:t2}
 f^i(z,a)=z^i  \mbox{ for all } z\in \RR^{\;n} \mbox{ if and only if } a=0.
\end{equation}
This implies in particular that $f^i(z,0)=z^i$.
\begin{defi}
\begin{em}
\label{def1}
The set $G$ of transformations $T_a$ given by (\ref{ch1:t1}) and
satisfying Condition (\ref{ch1:t2}) is
called a {\em (local) continuous one--parameter group} of transformations
in $\RR^{\;n}$ if
the function $f=(f^1,\ldots,f^n)$ satisfies the following property:
for all $a,\;b\in U\subset U' \subset \RR\;$, there exists $c\in U'$ such that
\begin{equation}
\label{ch1:t3}
f(f(z,a),b)=f(z,c),\mbox{ for all $z\in \RR^{\;n} $},
\end{equation}
where
\begin{equation}
\label{ch1:t4}
c=\varphi (a,b)
\end{equation}
is a smooth function of its arguments such that for all $a\in U$, the equation
\[ \varphi (a,b)=0\]
admits a unique solution $b\in U$.
\end{em}
\end{defi}
Condition
(\ref{ch1:t3}) is sometimes referred to as the {\em group property} and $\varphi$
is termed the {\em group composition law}. According to Definition \ref{def1},
a continuous one--parameter group $G$  of transformations $T_a$ contains the
(unique) identity transformation $Id_{\RR^{\;n}}=T_0$. Furthermore, the group
property (\ref{ch1:t3}) means that any two transformations $T_a,\;T_b\in G$ carried
out one after the other results in a transformation which belongs again to $G$
for any $a,\;b\in U$. The solvability of  $\varphi (a,b)=0$ for all $a\in U$,
together with the group property (\ref{ch1:t3}), provide the inverse transformation
$T_a^{-1}\in G$ to $T_a\in G$. Thus $G$ is a group for the usual composition
of maps. The term continuous group comes from the fact that the parameter
$a$ varies continuously in $U\subset \RR$ and any transformation $T_a$
is continuously connected to the identity $T_0$ within $G$.

From now on, by a one--parameter group we will mean  a continuous one--parameter
group.
\begin{defi}
\label{def2}
\begin{em}
The group parameter  $a$ is said to be {\em canonical} if the composition law
(\ref{ch1:t4}) is $\varphi (a,b)=a+b$, i.e. if the group property (\ref{ch1:t3})
has the form
\begin{equation}
\label{ch1:t5}
f(f(z,a),b)=f(z,a+b) .
\end{equation}
\end{em}
\end{defi}
The next theorem shows that we can always assume that the parameter of an
arbitrary one--parameter group is canonical.
\begin{theo}[Lie--Scheffers 1891]
\begin{em}
Given an arbitrary composition law (\ref{ch1:t4}), the canonical parameter $\tilde a$
is defined by the formula
\begin{equation}
\label{ch1:t6}
\tilde a=\int^a_0 \frac{da}{\omega (a)},
\end{equation}
where
\[\omega (a)=\frac{\partial \varphi (a,b)}{\partial b}|_{b=0}.\]
\end{em}
\end{theo}
{\bf Proof.} See for example Lie and Scheffers (1891), Dickson (1924),
Eisenhart (1933), Ibragimov (1999).

Henceforth, we shall adopt the canonical parameter when referring to
one--parameter groups. 

Consider the first--order Taylor expansion of $T_a$ about $a=0$:
\begin{equation}
\label{ch1:t7}
\bar z^i \approx z^i+a\xi^i (z),
\end{equation}
where we have used the fact that $ f(z,0)=z$ and
\begin{equation}
\label{ch1:t8}
\xi^i (z)=\frac{\partial f^i (z,a)}{\partial a}|_{a=0},\; \; i=1,\ldots,n.
\end{equation}
Note that $\xi=(\xi^1,\ldots,\xi^n)$ is the vector tangent at $z$
to the orbit defined by Eq. (\ref{ch1:t1}).
\begin{defi}
\label{def3}
\begin{em}
The first--order linear differential operator
\begin{equation}
\label{ch1:t9}
X=\xi^i (z)\frac{\partial}{\partial z^i}
\end{equation}
is known as the {\em symbol} or the {\em generator} of the group $G$.
\end{em}
\end{defi}
 {\bf Example.} Consider the set $\cal{S}$ of transformations
\[\bar z=(a+1)z\equiv f(z,a),\]
where $a\in U=\RR-\{-1\}$.

(i) $f(z,a)=z$ for all $z\in \RR^{\;n}$ if and only if $a=0$.

(ii) For $a,\; b\in U$ and $z\in \RR^{\;n}$,
\[f(f(z,a),b)=f(z,a+b+ab)=f(z,c),\]
where $c=\varphi (a,b)=a+b+ab$.

(iii) For $a\in U$, $\varphi (a,b)=0$ if and only if $b=-a/(a+1) \in U$.

Hence $\cal{S}$ is a one--parameter group called the {\em scaling group}.
The group parameter $a$ is not canonical.
Using the formula (\ref{ch1:t6}), we find that the canonical parameter is
\[\tilde a=\int^{a}_{0} \frac{d\tau}{1+\tau}=\ln (1+a).\]
The  symbol of $\cal{S}$ is then given by
\[X=z^i\frac{\partial}{\partial z^i}\;\cdot \]
The interesting fact about the symbol of a one--parameter group  is that
it defines completely the finite transformations of the group as shown
by Lie's theorem:
\begin{theo}[Lie--Scheffers 1891]
\label{lie}
\begin{em}
The transformations (\ref{ch1:t1}) of any one parameter group whose
identity has the parameter $a=0$ are put into one--to--one correspondence,
by the introduction of the canonical parameter (\ref{ch1:t6}), with the
transformations (\ref{ch1:t1}) represented with respect to the canonical
parameter $\tilde a$. Let (\ref{ch1:t1}) be represented in the canonical
parameter $a$ with infinitesimal transformations (\ref{ch1:t7}), where
(\ref{ch1:t8}) holds.
The functions $f^i$ are solutions of the
initial value problem
\begin{equation}
\label{ch1:t10}
\frac{d\bar z}{da}=\xi (\bar z),\;\; \bar z|_{a=0}=z.
\end{equation}
\end{em}
\end{theo}
{\bf Proof.} Consult for example  Lie and Scheffers (1891), Dickson (1924),
Eisenhart (1933), Ibragimov (1999).

The system (\ref{ch1:t10}) is known as Lie's equations. 

{\bf Remark.} The initial value problem (\ref{ch1:t10}) has the formal
solution ({\em Lie's series})
\[\bar z=\mbox{e}^{a X}z\equiv \sum_{k=0}^{\infty}
\frac{a^k}{k!}X^k z,\]
where $X$ is given by Eq. (\ref{ch1:t9}). 

Theorem \ref{lie} provides a simplification in the investigation of
continuous
one--parameter groups: it reduces the study of continuous one--parameter
groups to that of infinitesimal transformations (\ref{ch1:t7}). Thus it offers a
`linearization' to  continuous one--parameter groups. This fact will
be crucial in the calculation of symmetries of differential equations.

{\bf Example.} Consider the one--parameter group whose generator is
\[X=-y\frac{\partial}{\partial x}+ x \frac{\partial}{\partial y}\;\cdot\]
According to Lie's theorem, its transformations are solutions of the initial
value problem
\[ \frac{d\bar x}{da}=-\bar y,\;\; \frac{d\bar y}{da}=\bar x,\;\bar x|_{a=0}=x,\;
\bar y|_{a=0}=y,\]
which are
\[ \bar x=x\cos a-y\sin a,\;\; \bar y= x\sin a+ y\cos a .\]

\begin{defi}
\begin{em}
A function $F(z)$ is invariant under the group of transformations
(\ref{ch1:t1})
if
\[F(z)=F(\bar z) \mbox{ for all $z\in \RR^{\;n}$} \mbox{ and } a\in U.\]
\end{em}
\end{defi}
The following theorem provides an infinitesimal criterion for invariance.
\begin{theo}
\begin{em}
A function $F(z)$ is invariant under the group of transformations (\ref{ch1:t1})
if and only if
\[XF=0,\]
where  $X$ is the generator of the group.
\end{em}
\end{theo}
{\bf Proof.} See for example Ovsiannikov (1982).
\begin{theo}
\label{canv}
\begin{em}
Every continuous one--parameter group of transformations whose transformations
are given by (\ref{ch1:t1}), reduces to the group of translations
\begin{equation}
\label{ch1:t11}
\bar y^1=y^1+a,\;\;\bar y^i=y^i,\;\; i=2,\ldots,n,
\end{equation}
under a suitable change of variables
\begin{equation}
\label{ch1:t12}
y^i=y^i (z),\;\; i=1,\ldots,n.
\end{equation}
The new variables $y^i$ are called {\em canonical variables}. They are
singular
for values of $z$ such that $\xi (z)=0$ (invariant points).
\end{em}
\end{theo}
{\bf Proof.} Using the chain rule, we find that under the change of variables
(\ref{ch1:t12}), the differential
operator (\ref{ch1:t9}) transforms according to the formula
\[X=X(y^i)\frac{\partial}{\partial y^i}\; \cdot\]
Therefore canonical variables are found from the system of linear partial
differential equations
\[X(y^1)=1,\;\; X(y^2)=0,\ldots,X(y^n)=0.\]
{\bf Example.}  Consider the one--parameter group with symbol
\[X=-y\frac{\partial}{\partial x}+ x \frac{\partial}{\partial y}\; \cdot\]
The canonical variables $(\theta, r)$ are solutions of the system
\[X\theta =1,\;Xr=0\]
i.e.
\[\frac{dx}{-y}=\frac{dy}{x}=\frac{d\theta}{1},\;\;
  \frac{dx}{-y}=\frac{dy}{x}=\frac{dr}{0},\]
i.e.
\[\theta =\arctan (y/x)+F(x^2+y^2),\;\; r=G(x^2+y^2),\]
where $F$ and $G$ are arbitrary functions. We can choose 
\[\theta=\arctan (y/x),\;\;r=\sqrt{x^2+y^2},\]
which are the usual polar coordinates.

\section{Extension of one--parameter groups: prolongation formulas}
Consider a one--parameter group $G$ of point transformations
\begin{eqnarray}
\bar t = f(t,q,a),  & & f(t,q,0)=t, \label{ch1:t13}\\
\bar q_i =\varphi_i (t,q,a), & & \varphi_i (t,q,0)=q_i,\;\; i=1,\ldots,n,
\label{ch1:t14}
\end{eqnarray}
where $t$ is the independent variable and the $q_i$s in
$q=(q_1,\dots, q_n)$ are the dependent
variables. Let the generator of $G$ be
\begin{equation}
\label{ch1:t15}
X=\xi (t,q)\frac{\partial}{\partial t}+
\eta_i (t,q)\frac{\partial}{\partial q_i}\; \cdot
\end{equation}
We wish to extend the action of the group $G$ to the space of
extended variables 
$(t,q, \dot q,\ldots, q^{(k)})$, $k\ge 1$, where the
overdot stands for differentiation with respect to $t$ and
\begin{equation}
\label{ch1:t16}
q^{(k)}=\left (\frac{d^k q_1}{dt^k},\ldots,\frac{d^k q_n}{dt^k}\right).
\end{equation}
Let us denote the operator of total differentiation with respect to $t$ by
$D$, i.e.
\begin{equation}
\label{ch1:t17}
D = \frac{\partial}{\partial t}+q_i\frac{\partial}{\partial \dot q_i}
      + \dot q_i\frac{\partial}{\partial \ddot q_i}+\cdots
\end{equation}
and by $\bar D$ the operator of total differentiation with respect
to $\bar t$. An application of the chain rule gives
\begin{equation}
\label{ch1:t18}
D=D(f)\bar D.
\end{equation}
Now
\begin{equation}
\label{ch1:t19}
D \varphi_j =D(f)\bar D\bar q_j, \;\; j=1,\ldots ,n
\end{equation}
and  we obtain
\begin{equation}
\label{ch1:t20}
D\varphi_j  = \bar{\dot q_j} D(f),
\end{equation}
where $\bar{\dot q_j}= d\bar q_j/d\bar t$. Next, the expansion of
Eq. (\ref{ch1:t20}) yields
\begin{equation}
\label{ch1:t21}
\left (\frac{\partial f}{\partial t}+\dot q_i \frac{\partial f}{\partial q_i}
\right )\bar{\dot q_j}= \frac{\partial \varphi_j}{\partial t} +
\dot q_i \frac{\partial \varphi_j}{\partial q_i}\; \cdot
\end{equation}
Eq. (\ref{ch1:t20}) defines the extension of the group actions
(\ref{ch1:t13})--(\ref{ch1:t14}) to the first derivatives $\dot q_i$.
Transformations in the $(t,q,\dot q)$--space,
given by Eqs. (\ref{ch1:t13}), (\ref{ch1:t14}) and
(\ref{ch1:t20}),
determine a one--parameter group (Lie and Scheffers 1891) called
the {\em first prolongation} of the group $G$ denoted by $G^{[1]}$.
Higher--order  extensions of the point transformations are obtained by
successive differentiation of Eq. (\ref{ch1:t20}) taking into account
Eq. (\ref{ch1:t18}).

We now invoke Eq. (\ref{ch1:t20}) for the infinitesimal transformations
corresponding to
(\ref{ch1:t13}) and (\ref{ch1:t14}), i.e., for $f=t+a\xi$ and $\varphi_i=q_i+a\eta_i$.
If we set $\bar{\dot q_i} =\dot q_i+a\eta_i^{[1]}$ for the infinitesimal
transfomations of the first derivatives, simple calculations show, from Eq.
(\ref{ch1:t20}), that
\begin{equation}
\label{ch1:t22}
\eta_i^{[1]}=D(\eta_i)-\dot q_i D(\xi).
\end{equation}
Thus, given a group $G$ with the symbol (\ref{ch1:t15}), its first
prolongation $G^{[1]}$ has the generator
\begin{equation}
\label{ch1:t23}
X^{[1]}=X+\eta_i^{[1]}\frac{\partial}{\partial \dot q_i}\; \cdot
\end{equation}
The higher--order prolongation $G^{[k]}$, $k\ge 2$, has its generator
defined recursively as 
\begin{equation}
\label{ch1:t24}
X^{[k]}=X^{[k-1]}+\eta_i^{[k]}\frac{\partial}{\partial  q_i^{(k)}},
\end{equation}
where
\begin{equation}
\label{ch1:t25}
\eta_i^{[k]}=D(\eta_i^{[k-1]})- q_i^{(k)}D(\xi).
\end{equation}
Eqs. (\ref{ch1:t22}) and (\ref{ch1:t25}) are known as
{\em prolongation formulas}.
\begin{defi}
\begin{em}
A function $I(t,q,\ldots, q^{(k)})$ with $k \ge 1$, is a differential
invariant of order $k$ of a group $G$ of transformations (\ref{ch1:t13}) and
(\ref{ch1:t14}) if
\[ I(t,q,\ldots, q^{(k)})=I(\bar t,\bar q,\ldots, \overline{q^{(k)}})\]
for all $(t,q)\in \RR^{\; n+1}$ and $a\in U\subset \RR$, i.e. if $I$ is
invariant under the prolonged group $G^{[k]}$.
\end{em}
\end{defi}
\begin{theo}
\begin{em}
A  function $I(t,q,\ldots,q^{(k)})$  is invariant under the prolonged group
$G^{[k]},\;k\ge 1$
if and only if
\[X^{[k]}I=0,\]
where $X^{[k]}$ is the generator of $G^{[k]}$. 
\end{em}
\end{theo}
{\bf Proof. } See for instance Ovsiannikov (1982).
\section{Point symmetries of differential equations: Lie's algorithm}
Basically there are two ways of interpreting a differential equation (de).
Firstly, it can be seen as a relation between  unknown
functions and their derivatives. Secondly it may be seen as defining a manifold
in the space of extended variables. The second approach has the advantage of
not requiring the explicit form of solutions in order to describe their
symmetry properties. This is actually how Lie interpreted des in his study of their
symmetries.

Loosely, a {\em point symmetry} (or simply {\em symmetry} when there is no
ambiguity) of a de is an invertible
transformation (of independent and dependent variables)
which leaves it unchanged or invariant. But this definition has the drawback
of being too general and it does not provide an efficient way of obtaining
the symmetries. By considering only symmetries which form
continuous one--parameter groups of
transformations, Lie was able to derive an algorithm for
the calculation of symmetries. These kind of symmetries are known as {\em
continuous symmetries}. Henceforth, by symmetry, it should be understood
as continuous symmetry.

Since we have seen that a continuous one--parameter
group of transformations can equivalently be described by its generator, the
generator of a continuous symmetry will also be called a symmetry.

Consider a system of ordinary differential equations (odes)
\begin{equation}
\label{ch1:s1}
q_i^{(n)}=\omega_i (t,q,\dot q,\ldots,q^{(n-1)}),
\;\;i=1,\ldots,k,\;\;k,\;n \ge 1.
\end{equation}
Assume that a symmetry of Eqs. (\ref{ch1:s1}) is a one--parameter group of
transformations whose infinitesimal transformations are
\begin{eqnarray}
\bar t &\approx & t+a\xi (t,q), \label{ch1:s2}\\
\bar q_i &\approx & q_i+a\eta_i (t,q),\;\;i=1,\ldots,k \label{ch1:s3}
\end{eqnarray}
or equivalently whose generator is
\begin{equation}
\label{ch1:s4}
X=\xi (t,q)\frac{\partial}{\partial t}
+\eta_i (t,q)\frac{\partial}{\partial q_i}\; \cdot
\end{equation}
By definition of a symmetry, we must have
\begin{equation}
\label{ch1:s5}
\overline{q_i^{(n)}}=\omega_i (\bar t,\bar q,\bar{\dot q},\ldots,\overline{q^{(n-1)}}),
\;\;i=1,\ldots,k,\;\;k,\;n \ge 1.
\end{equation}
Now, differentiate Eqs. (\ref{ch1:s5}) with respect to the
group parameter $a$:
\begin{equation}
\label{ch1:s6}
\frac{d \overline{q_i^{(n)}}}{da}=
\frac{\partial \omega_i}{\partial \bar t}
\frac{d \bar t }{d a}+
\frac{\partial \omega_i}{\partial \bar{q_j}}
\frac{d \bar{q_j}}{d a}+\cdots +
\frac{\partial \omega_i}{\partial \overline{q_j^{(n-1)}}}
\frac{d \overline{q_j^{(n-1)}}}{d a}\; \cdot
\end{equation}
Setting $a=0$ in Eq. (\ref{ch1:s6}) and  using the prolongation formulas, we
obtain
\begin{equation}
\label{ch1:s7}
X^{[n]} (q_i^{(n)}-\omega_i)=0,\;\; i=1,\ldots,k,
\end{equation}
provided Eqs. (\ref{ch1:s1}) are satisfied.  Conversely suppose that
Eqs. (\ref{ch1:s7})
hold. Using canonical variables, we can assume without lost of generality
that $X=\partial /\partial t$. Hence Eqs. (\ref{ch1:s7}) read
\begin{equation}
\label{ch1:s8}
 \frac{\partial \omega_i}{\partial t}=0,\;\; i=1,\ldots,k,
\end{equation}
whenever Eqs. (\ref{ch1:s1}) hold. Since Eqs. (\ref{ch1:s1}) define a manifold in the
space of extended variables and the manifold is connected\footnote{i.e.
it cannot be partitioned into two non--empty open subsets
each of which has no points in common with the closure of the other.},
Eqs. (\ref{ch1:s8})
imply that the $\omega_i$s are independent of $t$. Thus the one--parameter
group of translations
\[ \bar t=t+a,\;\; \bar{q_i}=q_i,\;\;i=1,\dots, k,\]
leaves Eqs. (\ref{ch1:s1}) unchanged, i.e. $X$ is their symmetry. In essence
we have proved the following:
\begin{theo}[Infinitesimal criteria of invariance]
\label{inv}
\begin{em}
The operator $X$ defines a symmetry of Eqs. (\ref{ch1:s1}) if and only if
\begin{equation}
\label{ch1:s9}
X^{[n]} (q_i^{(n)}-\omega_i)|_{(\ref{ch1:s1})}=0,\;\; i=1,\ldots,k,
\end{equation}
where $|_{(\ref{ch1:s1})}$ means evaluated on the manifold
defined by (\ref{ch1:s1}).
\end{em}
\end{theo}
{\bf Remark.} Eqs. (\ref{ch1:s9}) are known as the {\em defining} or the
{\em determining} equations for the symmetries of Eqs. (\ref{ch1:s1})
as their solutions
lead to the full set of symmetries. For $n \ge 2$, the determining equations
are in general an overdetermined system of linear partial differential
equations which are much easier to solve than the initial
equations (\ref{ch1:s1}). This is
where the strength of Lie's method resides.

The infinitesimal criteria of invariance apply to all types of
locally solvable differential equations (see Ovsiannikov 1982).
The restriction to odes made here is designed to suit our needs.
\begin{defi}
\label{al}
\begin{em}
A {\em Lie algebra} consists of a vector space $L$ over a field $\FF$
together with a binary operation $[\;\;,\;\;]$, called {\em Lie bracket} or
{\em commutator}, defined on $L$ such that the following axioms are
satisfied:

(a) Bilinearity: for any $X,\;Y,\;Z\in L$ and $a,\;b\in \FF$,
\[ [aX+bY,Z]=a[X,Z]+b[Y,Z],\]
\[ [X,aY+bZ]=a[X,Y]+b[X,Z];\]
(b) Skew-symmetry: for any $X,\;Y\in L$,
\[ [X,Y]=-[Y,X];\]
(c) Jacobi identity: for any $X,\;Y,\;Z \in L$,
\[ [\;[X,Y],Z]+[\;[Y,Z],X]+[\;[Z,X],Y]=0.\]
\end{em}
\end{defi}
We shall take $\FF$  to be the field of real numbers $\RR\;$  or complex numbers
$\mathbf{C}$ (Chapter 2 only). We define the Lie bracket $[\;\;,\;\;]$ on the set of
vector fields $\vartheta$ (first--order linear differential operators) as
\begin{equation}
\label{ch1:s10}
[X,Y]=XY-YX   \mbox{  for any $X,\;Y\in \vartheta .$}
\end{equation}
Definition (\ref{ch1:s10}) introduces a binary operation in the space of vector
fields $\vartheta$ which makes it into a Lie algebra, i.e., axioms (a), (b) and
(c) of Definition \ref{al} hold (Ovsiannikov 1982).
\begin{defi}
\label{ch1:sub}
\begin{em}
A subalgebra of a Lie algebra $L$ is a subspace of $L$ which contains the
commutator of any two of its elements. 
\end{em}
\end{defi}
{\bf Remark.} It can be verified that a subalgebra of a Lie algebra is also
a Lie algebra.
\begin{defi}
\label{id}
\begin{em}
An ideal of a Lie algebra $L$ is a subspace $I$ of $L$  such that
$[X,Y]\in I$ for all $X\in I$ and $Y\in L$.
\end{em}
\end{defi}
{\bf Remark.} An ideal of $L$ is also a subalgebra of $L$.
\begin{defi}
\label{ch1:sol}
\begin{em}
An $n$--dimensional Lie algebra $L_n$ is solvable if there exist
$r$--dimensional subalgebras
$L_r,\; r=1,\ldots,n$, such that $L_r$ is an ideal of $L_{r+1}$.
\end{em}
\end{defi}
\begin{theo}
\begin{em}
A Lie algebra $L_n$ is solvable if and only if there is an integer $p$ such
that  $L_n ^{(p)}=\{0\}$, where $L_n^{(p)}$ is the $p$th derived algebra
defined by $L_n^{(1)}=[L_n,L_n]$,\ldots, $L_n^{(p)}=[L_n^{(p-1)},L_n^{(p-1)}].$
\end{em}
\end{theo}
{\bf Proof.} See for example Jacobson (1955).

In the chapters to come, we will be dealing mainly with finite--dimensional Lie
algebras. These algebras are completely described when a basis is specified
together with the commutators of its elements (Jacobson  1955).
\begin{theo}
\begin{em}
The set of all symmetries of a given differential equation form a Lie algebra
called the {\em symmetry Lie algebra} of the underlying equation.
\end{em}
\end{theo}
{\bf Proof.} See for instance Lie and Scheffers (1891), Ovsiannikov (1982).

So far, we have treated only point symmetries, i.e. symmetries whose generators
are independent of derivatives. The next section deals with symmetries
depending on first derivatives.
\section{Contact symmetries}
Here, we restrict ourselves to contact transformations of the plane. For a more
general treatment of the subject, the reader is referred to
Lie and Engel (1890), Lie and Scheffers (1896), Anderson and Ibragimov (1979).

Consider the one--parameter group $G$ of transformations
\begin{equation}
\label{ch1:s11}
\left \{ \begin{array}{lll}
\bar t & = & f(t,q,\dot q,a),\\
\bar q & = & g(t,q,\dot q,a),\\
\bar{\dot q} & = & h(t,q,\dot q,a),
\end{array}
\right.
\end{equation}

where $a$ is the group parameter and  $a=0$ corresponds to the identity
transformation. This group can be extented to the
``variables'' $dt,\; dq,\;d\dot q$ by means of the formulas
\begin{equation}
\label{ch1:s12}
\left \{ \begin{array}{lll}
d\bar t & = & f_tdt+f_qdq+f_{\dot q} d\dot q,\\
d\bar q & = & g_tdt+g_qdq+g_{\dot q} d\dot q,\\
d\bar{\dot q} &= & h_tdt+h_qdq+h_{\dot q} d\dot q,
\end{array}
\right.
\end{equation}
where the suffices stand for partial differentiations.
It can be verified that
Eqs. (\ref{ch1:s11})--(\ref{ch1:s12}) define a one--parameter group
$\tilde G $ of transformations  in the extended space
$(t,\;q,\;\dot q,\;dt,\;dq,\;d\dot q)$.
\begin{defi}
\begin{em}
The set $G$ is a group  of contact transformations if the equation
\begin{equation}
\label{ch1:s13}
dq-\dot q dt =0
\end{equation}
is invariant with respect to $\tilde G$.
\end{em}
\end{defi}
Eq. (\ref{ch1:s13}) is sometimes called the {\em contact condition}.
We have seen that a one--parameter group of transformations is equivalently
defined by the vector tangent to its orbit, i.e. the generator of the group.
So, let
\begin{equation}
\label{ch1:s14}
X=\xi (t,q,p)\dt+\eta (t,q,p)\dq+\eta^{[1]} (t,q,p)\dr,
\end{equation}
where
\[\dot q=p,\;\xi=\frac{\partial f}{\partial a}|_{a=0}
,\; \eta=\frac{\partial g}{\partial a}|_{a=0},
\;\eta^{[1]}=\frac{\partial h}{\partial a}|_{a=0}, \]
be the generator of $G$. Obviously, the generator of $\tilde G$ is
\begin{equation}
\label{ch1:s15}
\tilde X= X+d\xi \frac{\partial}{\partial (dt)}+
            d\eta \frac{\partial}{\partial (dq)}+
           d\eta^{[1]} \frac{\partial}{\partial (dp )}\; \cdot
\end{equation}
Now the invariance of Eq. (\ref{ch1:s13}) under $\tilde X$ reads
\[ \tilde X (dq-pdt)|_{(\ref{ch1:s13})}=0,\]
i.e.
\[(\eta_t dt+p\eta_qdt+\eta_pdp)-\eta^{[1]} dt-p(\xi_tdt+\xi_q pdt+\xi_pdp)=0,\]
viz.
\begin{equation}
\label{ch1:s16}
\left \{ \begin{array}{lll}
\eta_p &= & p\xi_p,\\
\eta^{[1]} & = & \eta_t+p(\eta_q-\xi_t)-p^2\xi_q.
\end{array}
\right.
\end{equation}
Whence the theorem.
\begin{theo}
\begin{em}
A vector field
\[X=\xi (t,q,p)\dt+\eta (t,q,p)\dq+\eta^{[1]} (t,q,p)\dr\]
generates a one--parameter group of contact transformations provided its
components satisfy Eqs. (\ref{ch1:s16}).

We will call such a vector field a {\em contact vector field}.
\end{em}
\end{theo}

{\bf Remark.} A contact vector field can also be written in terms of 
Lie's {\em characteristic function} $W =\eta -p\xi$ as
\[X=-W_p\dt+(W-pW_p)\dq+(W_t+pW_q)\dr\; \cdot \]
{\bf Example.} The first prolongation of a point vector field defines a
one--parameter  group of contact transformations, i.e. it is a contact
vector field.
\begin{defi}
\begin{em}
A Lie algebra of contact vector fields is {\em reducible} if there
is a contact
transformation which maps these vector fields into mere first prolongations
of point vector fields. It is {\em irreducible} otherwise.
\end{em}
\end{defi}
The next theorem gives an important criteria for reducibility and is due
to Lie (see Lie and Engel 1890). This is also to be found in Campbell (1903).
\begin{theo}
\begin{em}
A Lie algebra of contact vector fields is reducible if and only if its
operators leave unaltered a system of the form
\begin{equation}
\label{ch1:s17}
\frac{dt}{a}=\frac{dp}{\beta}=\frac{dq}{ap},
\end{equation}
where $a$ and $\beta$ are functions of $t,\;q,\;p$.
\end{em}
\end{theo}
{\bf Proof.} See  Lie and Engel (1890), Cambpell (1903). 

{\bf Example.} Consider the three--dimensional abelian Lie algebra $3L_1$ of
contact vector fields with basis operators
\[X_1=\dt,\;X_2=\dq,\;X_3=p\dt+\frac{p^2}{2}\dq \; \cdot \]
Although $X_3$ is a nontrivial contact vector field, $3L_1$ is reducible.
Indeed, it leaves invariant the system
\[\frac{dt}{1}=\frac{dp}{0}=\frac{dq}{p}\; \cdot \]
The contact transformation
\[\bar t=pt-q,\;\bar q=p,\;\bar p=1/t\]
maps  the basis operators of $3L_1$ to

\[X_1=\bar q\frac{\partial}{\partial \bar t}-\bar p^2\frac{\partial}{\partial  \bar p},\;
X_2=-\frac{\partial}{\partial \bar t},\;
X_3=\frac{\bar q^2}{2}\frac{\partial}{\partial \bar t}-\bar q\bar p^2
\frac{\partial}{\partial \bar p},\]
which are first prolongations of point vector fields.

The prolongations of contact vector fields are obtained by applying the same
recursive formulas (\ref{ch1:t25}) as for point vector fields.
\begin{theo}
\begin{em}
A contact vector field $X$ is a contact symmetry of a scalar ode
\begin{equation}
\label{ch1:s18}
 q^{(n)}=\omega (t,q,\dot q,\ldots,q^{(n-1)}),
\end{equation}
if and only if
\begin{equation}
\label{ch1:s19}
X^{[n]}(q^{(n)}-\omega)|_{(\ref{ch1:s18})}=0.
\end{equation}
\end{em}
\end{theo}
{\bf Proof.} Similar to the proof of Theorem \ref{inv}.

{\bf Remark.} Only scalar des have contact symmetries (see for example
Ovsiannikov 1982, Stephani 1989, Ibragimov 1999).
\section{Noether point symmetries}
Consider the variational principle
\begin{equation}
\label{ch1:n1}
\delta  \int_{t_1}^{t_2} L(t,q(t),\dot q(t))dt=0,
\end{equation}
yielding  the Euler-Lagrange equations
\begin{equation}
\label{ch1:n2}
\frac{d}{dt}\left (\frac{\partial L}{\partial \dot q_i}\right )
-\frac{\partial L}{\partial q_i}=0,\;\;i=1,\ldots ,k.
\end{equation}
Consider also infinitesimal transformations in the $(t,q)$--space, defined
by
\begin{equation}
\label{ch1:n3}
\bar t\approx t+a\xi (t,q),\;\; \bar q_i\approx q_i+a\eta_i (t,q), \;\;i=1,\ldots,k.
\end{equation}
By means of Eqs. (\ref{ch1:n3}), each curve $t\rightarrow q(t)$, defined on an
interval $[\alpha,\beta]$ is transformed (for sufficiently small $a$) into
a (parameter--dependent) curve $\bar t\rightarrow \bar q(\bar t)$ in the new
variables (see Logan 1977, Sarlet and Cantrijn 1981).
\begin{defi}
\begin{em}
\label{ch1:noe}
The infinitesimal transformations (\ref{ch1:n3}) define
a Noether point symmetry (generator) of Eqs. (\ref{ch1:n2}) if there is a
function $A(t,q)$, such  that for
each differentiable curve $t\rightarrow q(t)$, we have
\begin{equation}
\label{ch1:n4}
\int_{\bar t_1}^{\bar t_2} L(\bar t,\bar q(\bar t),\frac{d\bar q}{d\bar t}
(\bar t))d\bar t =\int_{t_1}^{t_2} L(t,q(t),\dot q(t))dt
+  a\int_{t_1}^{t_2} \frac{dA}{dt}(t,q(t))dt + O(a^2),
\end{equation}
where $[t_1,t_2]$ is any subinterval of the interval $[\alpha,\beta]$ on
which $q(t)$ is defined.
\end{em}
\end{defi}
From Definition \ref{ch1:noe} and the usual formula for the
change of variables in integrals, we infer that
\begin{equation}
\label{ch1:n5}
L(\bar t,\bar q(\bar t),\frac{d\bar q}{d\bar t} (\bar t) )
=L(t,q(t),\dot q(t))+ a\frac{dA}{dt}(t,q(t))+ O(a^2).
\end{equation}
Expanding the lefthand side of Eq. (\ref{ch1:n5}) about $a=0$ and equating
coefficients of like powers of $a$, we  obtain
\[ X^{[1]}L+D(\xi)L=DA,\]
where $X$ is the symbol of the infinitesimal transformations (\ref{ch1:n3}),
$D=d/dt$. Thus we have proved the following result.
\begin{theo}
\begin{em}
A vector field $X$ in the $(t,q)$--space defines a Noether point symmetry of
Eqs. (\ref{ch1:n2}) if and only if there exists a function $A(t,q)$ such that
\begin{equation}
\label{ch1:n6}
X^{[1]}L+D(\xi)L=DA.
\end{equation}
\end{em}
\end{theo}
{\bf Remark.} Eq. (\ref{ch1:n6}) plays the role of determining equations for
Noether point symmetries. Indeed Noether point symmetries are found by solving
Eq. (\ref{ch1:n6}) for $\xi,\;\eta_i$ and $A$. Here again, Eq. (\ref{ch1:n6}) is,
in general an  overdetermined system of linear partial differential equations.

When $A=0$, the Noether point symmetry $X$ is sometimes called
a {\em variational symmetry} of Eqs. (\ref{ch1:n2}) since it leaves the
functional $L$ unchanged.

\begin{theo}[Noether 1918]
\begin{em}
To each Noether point symmmety (\ref{ch1:n3}) corresponds a constant of motion or
first integral
$I(t,q,\dot q)$, given by
\begin{equation}
\label{ch1:n7}
I(t,q,\dot q)=\left [ L\xi+\frac{\partial L}{\partial \dot q_i}
(\eta_i-\dot q_i \xi) \right ]-A(t,q).
\end{equation}
\end{em}
\end{theo}
{\bf Proof.} See for instance Sarlet and Cantrijn (1981), Ovsiannikov (1982),
Olver (1986), Ibragimov (1999).
\section{Conclusion}
In this introductory chapter we have provided salient features of Lie's
theory of differential equations. We have specialised to odes as
this will adequately serve our needs in the chapters to come.
Furthermore, we have provided basic knowledge
on contact  and Noether point symmetries. Other important concepts will be
introduced as the need arises.

\end{document}
