%\documentstyle[12pt]{report}
%\setlength{\parindent}{0mm}
%\setlength{\parskip}{6pt}
%\renewcommand{\baselinestretch}{1.5}
%\setlength{\topmargin}{0pt}
%\setlength{\headheight}{0pt}
%\setlength{\headsep}{0pt}
%\setlength{\footskip}{45pt}
%\setlength{\footheight}{0pt}
%\setlength{\textwidth}{465pt}
%\setlength{\textheight}{660pt}
%\setlength{\oddsidemargin}{10pt}
%\begin{document}

\chapter*{\hspace{3cm} INTRODUCTION}
\addcontentsline{toc}{chapter}{\protect\numberline{}{Introduction}}
Lie's theory of differential equations enjoys, nowadays, a fair amount of
popularity. It
is one of the most systematic ways of searching for exact solutions
of differential equations. Speaking about the new theory he had discovered,
Lie (see Lie and Engel 1888) wrote:
\begin{quote}
I noticed
that the majority of ordinary differential equations which were integrable
by old methods were left invariant under certain transformations, and that
these methods consisted in using that property.
\end{quote}
Indeed, in the study of
invertible transformations leaving a given differential equation unchanged,
Lie found that those forming a one--parameter group can 
equivalently be described by the vector tangent to the orbits of the group.
He called this vector the {\em symbol} of the group and
{\em infinitesimal symmetry} of the
underlying equation. He showed that the set of symmetries of a  differential
equation form an {\em infinitesimal group} (Lie algebra in the modern
terminology) and  that the integrability of the equation depends upon the
properties of this infinitesimal group. Namely, he proved that a scalar
$n$th--order ordinary differential equation (ode) that admits a solvable Lie algebra
of point symmetries is solvable by quadratures. Since two--dimensional Lie algebras
are solvable, Lie gave an algorithm for integrating scalar second--order odes
having two--dimensional Lie point symmetry  algebras.

Lie did not enjoy the fruits of his theory during his lifetime
(1842--1899) as the
following extract from a letter to his friend Mayer shows (Engel 1899):
\begin{quote}
If I only knew how I could get mathematicians interested in transformation
groups and the treatment of differential equations which arises from them,
I am certain, absolutely certain, that at some point in future, these
theories will be recognized as fundamental. If I wish to create such an
understanding {\em sooner}, it is because, among other things, I could then do
ten times more.
\end{quote}
Except for a few students and colleagues who showed interest in his theory,
Lie
did not see his dream come true with the speed he would have liked. At some
stage, he even felt despondent. This is apparent in a correspondence to Klein
in  which he wrote (Lie 1922):
\begin{quote}
It is lonely, frightfully lonely, here in Christiania where nobody
understands my work and interest.
\end{quote}
In another letter to Engel he wrote (Purkert 1984, see also Fritzsche 1999):
\begin{quote}
From 1871--1876, I lived and breathed only transformation groups and
integration problems. But when nobody took any interest in these things,
I grew  a bit weary and turned to geometry for a time.
\end{quote}
Understanding Lie's difficult situation, Klein and Mayer sent their able
student Engel, in September 1884, to help him in editing his work.
Although being a gifted and prolific mathematician (245 publications!),
Lie had problems in translating his conceptual arguments into
analytical language. In some of his earlier works and lectures,
proofs of important theorems where reduced to mere geometrical figures
(Kowalewski 1950). This fact hampered
somewhat the dissemination of his ideas. Thus
the collaboration with Engel was worthwhile and proved fruitful. It
culminated in the publication of three volumes of
Lie's work
under the title {\em Theorie der Transformationsgruppen}
(Lie and Engel 1888, 1890, 1893). Despite this effort at popularisation,
Lie's theory
of differential equations laid relatively dormant until Ovsiannikov (1962)
put a new spin on it in the late 1950s. Since then books (see for instance
Ibragimov 1985, 1999, Olver 1986, Bluman and Kumei 1989, Stephani 1989,
Hill 1992,\ldots) and papers have flourished on the subject. 

Although the theory developed by Lie also applies to systems and other types
of symmetries (contact symmetries, Noether symmetries), recent papers on
symmetry properties of odes have concentrated mainly on Lie point
symmetries of scalar equations. It is the purpose of this work to break this
deadlock.

The first chapter provides rudiments of group analysis of differential
equations. In this preliminary  chapter, emphasis is placed on notions
relevant to this thesis.  

Lie and Scheffers (1891) showed that the equation $y^{(n)}=0$ with $n\ge 4$ does not admit contact
symmetries. Moreover, they proved that $y^{(3)}=0$ admits a ten--dimensional
contact symmetry algebra which has important geometrical properties
(Campbell 1903, Chern 1937). Also,
it is well--known that second--order odes possess infinite--dimensional contact
symmetry algebras. Recently the study of Lie--B\"acklund symmetries of
the general linear scalar odes  was undertaken by Svirshchevskii (1995).
At this point, the existence of nonlinear scalar $n$th--order odes ($n\ge 3$)
with contact symmetries may be questioned. In the second chapter, we use
Lie's classification of contact transformations in the complex plane
(Lie 1874, Lie and Engel 1890, see also  Campbell 1903) to address this issue.
We establish that the only third--order scalar odes  with
contact symmetries are those  reducible via contact transformations  to the
simplest equation $y^{(3)}=0$. This result is then applied to the study of
symmetry breaking of third--order odes.
Further, we show that fourth--order scalar odes do
not admit contact symmetries. In addition, for $n\ge 5$, we classify up to
contact transformations, scalar $n$th--order odes which admit contact
symmetries.

Noether's theorem (Noether 1918) is a powerful tool for constructing first integrals of
Lagrangian systems. Sarlet and Cantrijn (1981) proved that for
Lagrangian systems of
second--order odes, a Noether symmetry is also a symmetry of the associated
first integral. It appears that the importance of this result has not been
fully realized since the literature does not reflect applications of it.
In Chapter
3, we emphasize its use for integration. Namely, we study the integrability of
the equation $y''=f(x)y^{(n)}$ using its Noether point symmetries. This equation
plays  an important role in the study of spherically symmetric perfect fluid
solutions of General Relativity.

It is obvious that for a better insight in the integration of systems of
nonlinear odes, the first step should be the study of linear systems. Thus
Chapters 4 and 5 are devoted to these systems.

Precisely, in Chapter 4 we study
superposition principles for systems of linear odes. Namely, we prove that if
we know $mn-1$ solutions of a system of $m$ $n$th--order linear homogeneous
odes, we can write down its general solution in terms of the known ones only.
This generalises the Abel (1839) and Forsyth (1921) formula.
Our approach is based on symmetry properties of the system.
Moreover we extend our
result to nonhomogenous systems and thus give a symmetry justification for
the well--known method of variation of parameters.

Chapter 5 focuses on the study of symmetry breaking for systems of two
second--order linear odes: using a novel canonical form for systems of two
second--order linear odes, we establish that  this kind of system can have
a $5$--, $6$--, $7$--, $8$-- or $15$--dimensional point symmetry algebra.
This
result emphasizes both the richness and the complexity of  the symmetry
properties of systems of linear odes.

Generally in applications, systems of linear odes occur in disguise form.
Sometimes the linear structure of a system is uncovered only after an
appropriate change of variables. The problem is then to determine under
what circumstances such a change of variables does exist.
In the case of scalar second--order odes,
Lie (1883) and his student Tresse (1894, 1896) tackled completely
the problem of
linearization by using the fact that all scalar second--order linear odes
can be
mapped invertibly to the free particle equation viz. $y''=0$.
However, according to 
Chapter 5, such an argument does not apply to systems of two second--order
odes. In Chapter 6, using symmetry methods and the canonical forms derived in
Chapter 5, we obtain two linearization criteria for systems of two
second--order odes. These criteria are constructive and enable an effective
calculation of the linearizing transformation. Also, we obtain the
generalization of these criteria to general systems of second--order odes.

The last chapter (Chapter 7) is concerned with the classification and the integration of
systems of two second--order odes admitting four--dimensional point symmetry
algebras. In more detail, we obtain nonsimilar realizations of three-- and
four--dimensional Lie algebras in terms of vector fields in
$(1+2)$--dimensional space.
By imposing these realizations on systems of two second--order odes, we are
able to classify those with four--dimensional symmetry algebras. Moreover we
emphasize the difference between the integration of scalar equations and that
of systems. Whereas scalar second--order odes can be integrated by
quadratures once two symmetries are known, the integration of a system of two
second--order odes admitting a four--dimensional symmetry Lie algebra may
depend on that of a first order scalar ode. Furthermore for such systems,
successive reduction of order does not always work even when their symmetry
Lie algebras are solvable. For this reason, we provide two approaches for the
integration of systems of odes.

%\end{document}
