%\documentclass[12pt]{report}
%\usepackage{lscape}
%\setlength{\parindent}{0mm}
%\setlength{\parskip}{6pt}
%\renewcommand{\baselinestretch}{1.5}
%\setlength{\topmargin}{0pt}
%\setlength{\headheight}{0pt}
%\setlength{\headsep}{0pt}
%\setlength{\footheight}{0pt}
%\setlength{\footskip}{45pt}
%\setlength{\textwidth}{430pt}
%\setlength{\textheight}{660pt}
%\setlength{\oddsidemargin}{10pt}
%\newcommand{\RR}{\mathrm{I\!R\!}}
%\newcommand{\T}[1]{\begin{tabular}{|l|} \hline #1\\ \hline \end{tabular}}
%\newcommand{\E}{\mbox{e}}
% \newcommand{\LL}{{\cal L}}
%\newtheorem{prop}{Proposition}[chapter]
%\newtheorem{theo}{Theorem}[chapter]
%\newtheorem{defi}{Definition}[chapter]
%\newtheorem{coro}{Corollary}[chapter]
%\begin{document}

\chapter{Canonical Forms for Systems of two Second--Order Ordinary
Differential Equations admitting  Symmetry Lie Algebras}

We obtain nonsimilar classes of realizations for 
three-- and four--dimensional real Lie algebras in the space of vector fields in
three variables. This is applied to the classification  and integration of
systems of two second--order ordinary differential equations (odes)
admitting  four--dimensional symmetry Lie algebras. Thus we obtain an
analogue of Lie's method of integrating  scalar second--order odes
admitting two--dimensional symmetry Lie algebras. We have reported some
of the problems discussed here in Wafo Soh and Mahomed (1999g).
\section{Introduction}

The theory of Lie groups and algebras originated from Lie's early works 
related to the integration of scalar odes (Lie 1891, Lie and Scheffers 1891).
In the
study of invertible transformations leaving a given differential equation
invariant, Lie noticed that those forming a one--parameter group can be
described equivalently by using the vector tangent to the orbits of the
group. He called this vector the {\em symbol} of the group and {\em
symmetry} of the underlying equation. He showed that the set of symmetries
of a differential equation forms an {\em infinitesimal group} (Lie algebra
in the modern terminology (Jacobson 1955, Belinfante {\em et al.} 1966))
and that the integrability of
the equation depends upon the properties of this {\em infinitesimal
group}. He showed that  a scalar $n$th--order ode admitting an
$n$--dimensional solvable Lie algebra of symmetries is solvable by
quadratures. Since two--dimensional Lie algebras are solvable, Lie gave an
algorithm for integrating second--order  odes having two--dimensional Lie
algebras of symmetries. Further, he explicitly classified in the complex domain
scalar  second--order odes having symmetries. The classification in the
real domain was given recently in Mahomed and Leach (1989).

Since systems of two second--order odes occur frequently in applications
(Classical and Quantum Mechanics, General Relativity,\ldots), it is worth
developing a classification scheme similar to Lie's for such systems.
Lie (see Lie and Engel 1893, Chapter 7) gave the complete classification of complex primitive  Lie
algebras in terms of vector fields in $(1+2)$--dimensional space: he obtained
$8$ nonsimilar classes. For complex imprimitive Lie algebras, he sketched the
method to follow in three steps and he implemented only the first two steps.

In this chapter, we investigate realizations of real three-- and
four--dimensional Lie algebras in terms of vector fields in
$(1+2)$--dimensional space. Furthermore, we classify all systems of
two second--order odes admitting four--dimensional symmetry Lie algebras
and we show  how to integrate the underlying equations. 

\section{Realizations of three-- and four--dimensional real Lie algebras in terms of vector
fields in $(1+2)$--dimensional space}

When we calculate the symmetries
of a given differential equation, we find the generators explicitly in the
form of vector fields (or first--order linear operators) and we only
afterwards compute the commutators to get the structure constants of the
particular Lie algebra we have found. But we could also proceed backwards,
that is, start from a given Lie algebra with a set of structure constants
and ask which vector  fields in at most three variables satisfy the given
set of commutator relations with none of the vector fields vanishing. We
thus ask for possible  {\em realizations} or {\em representations} of our
Lie algebra. Two realizations of the same Lie algebra will be considered
equivalent or similar if there exists an invertible transformation mapping
one of the realization to the other.

In this section we are concerned  with finding all nonsimilar representations
of three-- and
four--dimensional real Lie algebras in $(1+2)$--dimensional space. We
adopt the Mubarakzyanov (1963) classification scheme of  real low--dimensional
Lie algebras reported in Patera and Winternitz (1977) and
we also exploit the enumeration of subalgebras of these Lie algebras given
in the same paper. We shall use $\LL $ as a placeholder for relevant Lie
algebra/s $\LL _{i,j}^{a,b,R}$ (the $j$th algebra of dimension $i$ with
the superscripts $a,b$, if any, indicating  the parameters on which the
algebra
depends; also superscripts $R$, if any, indicating the algebra
realizations). Following Lie, we use the shorthand 
\[l=\frac{\partial}{\partial t},\;\;p=\frac{\partial}{\partial x},
\;\;q=\frac{\partial}{\partial y}\;\cdot\]
Finally the elements of a basis of a
given Lie algebra are named $X_i$, where $i$ is lesser than or equal to
the dimension of the underlying algebra. 

\subsection{Realizations of three--dimensional real Lie algebras} 

We now focus attention on the realizations of real three--dimensional Lie
algebras. For each algebra we write down only the nonzero commutation
relations.
 
\T{{\bf $\LL _{3,1}$ (the abelian three--dimensional Lie algebra).}}
 
Let $r=\mbox{rank}[X_1,X_2,X_3]$. We then consider the following cases.

{\bf (i)} $r=3$.

One can find coordinates in which
\[X_1=l,\;\;X_2=p,\;\;X_3=q.\]
{\bf (ii)} $r=2$.

There is a coordinate
system in which \[X_1=l,\;\; X_2=p,\;\; X_3=a(t,x,y)X_1+b(t,x,y)X_2.\]
Since $[X_1,X_3]=0$ and $[X_2,X_3]=0$, $a=a(y),\;b=b(y)$. Renaming the
variables, we obtain the representation
\[X_1=p,\;\;X_2=q,\;\;X_3=b(t)p+a(t)q.\]
{\bf (ii)} $r=1$

There are
coordinates in which \[X_1=p\;\; X_2=a(t,x,y)p,\;\;X_3=c(t,x,y)p.\]
Now $[X_1,X_2]=0$ and $[X_1,X_3]=0$ imply that $a=a(t,y)$, $b=b(t,y)$. Further
make the change of variables \[\bar t=a(t,y),\;\;\bar x=x,\;\; \bar
y=b(t,y).\] Thus \[X_1=\bar p,\;\;X_2=\bar t \bar p,\;\;X_3=\bar y \bar
p.\] Dropping the bars we obtain the representation
\[X_1=p,\;\;X_2=tp,\;\;X_3=yp.\]
\T{{\bf $\LL _{3,2}$: $[X_1,X_2]=X_2.$}}

By assuming the connectedness and then the unconnectedness of $X_1$ and
$X_2$, we arrive at the following cases.

{\bf(i)} $X_1=-yq$, $X_2=q.$

Let \[X_3=a(t,x,y)l+b(t,x,y)p+c(t,x,y)p.\] Then $[X_1,X_3]=0$ and $[X_2,X_3]=0$
imply that $a=a(t,x),\;b=b(t,x)\mbox{ and } c=0$, i.e.
$X_3=a(t,x)l+b(t,x)p$. There exists a change of variables \[\bar t=\bar t
(t,x),\;\;\bar x=\bar x (t,x),\; \bar y=y, \] in which \[X_1=-\bar y \bar
q,\;\; X_2=\bar q,\;\;X_3=\bar p.\] Leaving out the bars, we obtain the
realization \[X_1=-yq,\;\;X_2=q,\;\;X_3=p.\]
{\bf (ii)} $X_1=-xp-yq$,$X_2=q.$

The commutators $[X_1,X_3]=0$ and $[X_2,X_3]=0$ imply that
$a=a(t),\;b=b(t)x,\;c=c(t)x$, i.e.
\[X_3=a(t)l+b(t)xp+c(t)xq.\]
If $a(t)\ne
0$, make the change \[\bar t=\int dt/a ,\;\;\bar x= \alpha
(t)x,\;\;\bar y= \beta (t)y,\] where $\alpha$ and $\beta$ are solutions of
the equations \[a(t)\alpha '+b(t)\alpha=0,\;\;a(t)\beta '+c(t)\beta.\]
Omitting the bars, we obtain \[X_1=-xp-yq,\;\;X_2=cq,\;\;X_3=l,\] where
$c$ is a nonzero constant. Then replace $X_2$ by $(1/c) X_2$. Thus we
obtain the realization \[X_1=-xp-yq,\;\;X_2=q,\;\;X_3=l.\] If $a=0$ we
obtain the representation \[X_1=-xp-yq,\;\;X_2=q,\;\; X_3=b(t)xp+c(t)xq,\]
where $ (b(t),c(t))\ne (0,0).$

\T{{\bf $\LL _{3,3}$: $[X_2,X_3]=X_1$ (Weyl's algebra).}} 

The vector spaces $<X_1,X_2>$ and $<X_1,X_3>$ are two--dimensional abelian
subalgebras of
$\LL _{3,3}$. Since the operations
\[ X_2 \longrightarrow X_3,\;\; X_3
\longrightarrow -X_2\]
do not affect the structure of the algebra, the
following cases are relevant.\\ {\bf (i)} $X_1=l$, $X_2=p.$ \\ Let
$X_3=a(t,x,y)l+b(t,x,y)p+c(t,x,y)q$. Then $[X_1,X_3]=0,\;\;[X_2,X_3]=0$
imply that \[a=x+a(y),\;\;b=b(y),\;\;c=c(y).\] If $c=0$ and $b'(y)\ne 0$,
effect the change of variables
\[\bar t=t,\;\; \bar x= x+a(y),\;\;\bar y=b(y).\]
Dropping the bars, we obtain the realization
\[X_1=l,\;\;X_2=p,\;\;X_3=xl+yp.\] If $c=0$ and $ b=\mbox{const.}$, make
the transformation \[\bar t=t,\; \bar x=x+a(y),\;\;\bar y=y,\] and
replace $X_3$ by $X_3-\mbox{const }X_1$. Then, dropping the bars, we
obtain \[X_1=l,\;\;X_2=p,\;\;X_3=xl.\] If $c\ne 0$, make the change of
variables \[\bar t=t+\alpha (y),\;\;\bar x=x+\beta (y),\;\;y=\int dy/c,\]
where $\alpha$ and $\beta$ are solutions of the equations \[c(y)\alpha
'(y)+a(y)-\alpha (y)=0,\;\; c(y)\beta '(y)+b(y)=0.\] Without the bars, we
obtain \[X_1=l,\;\;X_2=p,\;\;X_3=xl+q.\] {\bf (ii)} $X_1=p$, $X_2=tp$\\
Let $X_3=a(t,x,y)l+b(t,x,y)p+c(t,x,y)q$. Then
$[X_1,X_3]=0,\;\;[X_2,X_3]=0$ yield \[a=-1,\;\;b=b(t,y),\;\;c=c(t,y).\]
Perform the change of variables \[\bar t=t,\;\; \bar x=x+\alpha (t,y),\;\;
\bar y =\beta(t,y),\] where \[\alpha_t-c(t,y)\alpha_y-b(t,y)=0,\;\;
\beta_t-c(t,y)\beta_y=0.\]
We find
\[X_1=p,\;\;X_2=tp,\;\;X_3=-l.\]
\T{{\bf $\LL _{3,4}$: $[X_1,X_3]=X_1$, $[X_2,X_3]=X_1+X_2.$}}

The vector space $<X_1,X_2>$ is the only abelian subalgebra of $\LL _{3,4}$.
Whence the cases:

{\bf (i)} $X_1=l$, $X_2=p.$ 

Let $X_3=a(t,x,y)l+b(t,x,y)p+c(t,x,y)q$. Then
$[X_1,X_3]=X_1$ and $[X_2,X_3]=X_1+X_2$ imply that
\[a=t+x+a(y),\;\;b=x+b(y),\;\;c=c(y).\]
If $c=0$, make the reduction using
\[\bar t=t+a(y)-b(y),\;\; \bar x=x+b(y).\] 
Leaving out the bars, we obtain
\[X_1=l,\;\;X_2=p,\;\;X_3=(t+x)l+xp.\]
If $c\ne 0$, invoke the change of variables
\[\bar t=t+\alpha (y),\;\;x=x+\beta (y),\;\; y=\int dy/c,\]
where $\alpha$ and $\beta$ satisfy the following equations 
\[c(y)\alpha '-\alpha-\beta+a(y)=0,\;\; c(y)\beta '-\alpha+b(y)=0.\]
We deduce 
\[X_1=l,\;\;X_2=p,\;\;X_3=(t+x)l+xp+q.\]
{\bf (ii)} $X_1=p$, $X_2=tp.$\\
Let $X_3=a(t,x,y)l+b(t,x,y)p+c(t,x,y)q$. Then
$[X_1,X_3]=X_1$ and $[X_2,X_3]=X_1+X_2$ give
\[a=-1,\;\;b=x+b(t,y),\;\;c=c(t,y).\]
Effect the change of variables
\[\bar t=t,\;\;\bar x=x+\alpha (t,y),\;\; y=\beta (t,y),\]
where $\alpha$ and $\beta$ satisfy the equations
\[\alpha_t-c(t,y)\alpha_y+\alpha-b(t,y)=0,\;\; \beta_t-c(t,y)\beta_y=0.\]
Supressing the bars, we obtain \[X_1=p,\;\;X_2=tp,\;\;X_3=-l+xp.\]
\T{{\bf $\LL _{3,5}$: $[X_1,X_3]=X_1$, $[X_2,X_3]=X_2.$}}

The vector space $<X_1,X_2>$  is the only two--dimensional abelian subalgebra of
$\LL _{3,5}$. Using the same method as before, we obtain the following
representations: \[X_1=l,\;\;X_2=p,\;\;X_3=tl+xp,\]
\[X_1=l,\;\;X_2=p,\;\;X_3=tl+xp+q,\] \[X_1=p,\;\;X_2=tp,\;\;X_3=xp,\]
\[X_1=p,\;\;X_2=tp,\;\;X_3=xp+q.\]
\T{{\bf $\LL _{3,6}^{a},\;\; a\in [-1,1)
$: $[X_1,X_3]=X_1$, $[X_2,X_3]=aX_2.$}}

After the same kind of reasoning as before we find the realizations:
\[X_1=l,\;\;X_2=p,\;\;X_3=tl+axp\] \[X_1=l,\;\;X_2=p,\;\;X_3=tl+axp+q\]
\[X_1=p,\;\;X_2=tp,\;\;X_3=t(1-a)l+xp\]
\T{{\bf $\LL _{3,7}^{a},\;\; a\ge 0
$: $[X_1,X_3]=aX_1-X_2$, $[X_2,X_3]=X_1+aX_2.$}}

We get the following realizations:
\[X_1=l,\;\; X_2=p,\;\;X_3=(at+x)l-(t-ax)p,\]
\[X_1=l,\;\;X_2=p,\;\;X_3=(at+x)l-(t-ax)p+q,\]
\[X_1=p,\;\;X_2=tp,\;\;X_3=-(t^2+1)l+(a-t)xp.\]
\T{{\bf $\LL _{3,8}$:
$[X_1,X_2]=X_1,\;\;[X_2,X_3]=X_3,\;\;[X_3,X_1]=2X_1.$}}

The vector space $<X_1,X_2>$ is
the only two--dimensional subalgebra of $\LL _{3,8}$. By assuming the
connectedness and then the unconnectedness of $X_1$ and $X_2$, we arrive
at the following cases:

{\bf (i)} $X_1=q$, $X_2=yq.$

Suppose
$X_3=a(t,x,y)l+b(t,x,y)p+c(t,x,y)q$. Then $[X_2,X_3]=X_3$ and
$[X_3,X_1]=2X_1$ result in \[a=0,\;\;b=0,\;c=-y^2.\] Whence the
realization \[X_1=q,\;\;X_2=yq,\;\;X_3=-y^2q.\] {\bf (ii)} $X_1=q$,
$X_2=xp+yq.$\\ Let $X_3=a(t,x,y)l+b(t,x,y)p+c(t,x,y)q$. Then
$[X_2,X_3]=X_3$ and $[X_3,X_1]=2X_1$ imply that
\[a=a(t)x,\;\;b=b(t)x^2-2xy,c=c(t)x^2-y^2.\] Hence the representation
\[X_1=q,\;\;X_2=yq,\;\;X_3=a(t)xl+\left (b(t)x^2-2xy\right)p +\left (
c(t)x^2-y^2 \right )q,\] where $a,\;b,\; c$ are arbitrary functions. By
considering the cases $a\ne 0$ and $a=0$, we determine, after suitable changes
of variables, the following realizations
\[X_1=l,\;X_2=tl+xp,\;X_3=-t^2l-2xtp+xq,\]
\[X_1=l+p,\;X_2=tl+xp,\;X_3=-t^2l-x^2p,\]
\[X_1=-tp,\;X_2=\frac{1}{2}(-tl+xp),\;X_3=-xl.\]
\T{{\bf $ \LL _{3,9}$:
$[X_1,X_2]=X_3,\;\;[X_3,X_1]=X_2,\;\;[X_2,X_3]=X_1.$}}

Note that if
$X_1,\;\;X_2$ and $X_3$ are connected, there is no real representation of
$\LL _{3,9}$. Henceforth, we assume that $X_1,\;X_2\mbox{ and}\;X_3$ are
not connected. There is a change of variables in which $X_1=l$. Now let
\[X_2=a(t,x,y)l+b(t,x,y)p+c(t,x,y)q\] and \[X_3=\alpha (t,x,y)l+\beta
(t,x,y)p+\gamma (t,x,y)q.\] Then
$[X_1,X_2]=X_3,\;\;[X_1,X_2]=X_2$ and $[X_2,X_3]=X_1$  give rise to 
\begin{eqnarray*} a=A_1(x,y)\cos t+A_2(x,y)\sin t, & &\alpha=A_2(x,y)\cos
t-A_1(x,y)\sin t ,\\ b=B_1(x,y)\cos t+B_2(x,y)\sin t, &
&\beta=B_2(x,y)\cos t-B_1(x,y)\sin t ,\\ c=A_1(x,y)\cos t+A_2(x,y)\sin t,
& &\gamma=C_2(x,y)\cos t-C_1(x,y)\sin t , \end{eqnarray*} where
$A_i,\;B_i,C_i;\;\;i=1,2$  satisfy the system \begin{equation} \label{ch7:eq1}
\left \{ \begin{array}{lll} B_1A_{2,x}-B_2A_{1,x}+C_1A_{2,y}-C_2A_{1,y}
&=& 1+A_1^2+A_2^2,\\ B_1B_{2,x}-B_2B_{1,x}+C_1B_{2,y}-C_2B_{1,y} &=&
A_1B_1+A_2B_2,\\ C_1C_{2,y}-C_2C_{1,y}+B_1C_{2,y}-B_2C_{1,y} &=&
1+A_1C_1+A_2C_2. \end{array} \right. \end{equation} Consider the change 
\[\bar x=\bar x(x,y),\;\;\bar y= \bar y(x,y),\;\;\bar
t=t+\lambda (x,y).\] In this coordinate system, \begin{eqnarray*} X_1
&=&\bar l, \\ X_2 &=& (\bar A_1\cos \bar t +\bar A_2\sin \bar t)\bar l
        +(\bar B_1\cos \bar t +\bar B_2\sin \bar t)\bar p   
        +(\bar C_1\cos \bar t +\bar C_2\sin \bar t)\bar q,\\   
X_3 &=& (\bar A_2\cos \bar t -\bar A_1\sin \bar t)\bar l
        +(\bar B_2\cos \bar t -\bar B_1\sin \bar t)\bar p   
        +(\bar C_2\cos \bar t -\bar C_1\sin \bar t)\bar q,   
\end{eqnarray*}        
where
\begin{eqnarray*}
\bar A_1 &=& A_1\cos \lambda-A_2\sin \lambda+\lambda_x(B_1\cos
\lambda-B_2\sin \lambda+\lambda_y(C_1 \cos \lambda-C_2\sin \lambda),\\
\bar A_2 &=& A_1\sin \lambda+A_2\cos \lambda+\lambda_x(B_1\sin
\lambda+B_2\cos \lambda+\lambda_y(C_1 \sin \lambda+C_2\cos \lambda),\\
\bar B_1 &=& \bar x_x(B_1\cos \lambda-B_2\sin \lambda)+\bar x_y(C_1\cos
\lambda-C_2\sin \lambda),\\ \bar B_2 &=& \bar x_x(B_1\sin \lambda+B_2\cos
\lambda) +\bar x_y(C_1\sin \lambda+C_2\cos \lambda),\\ \bar C_1 &=& \bar
y_x(B_1\cos \lambda-B_2\sin \lambda)+\bar y_y(C_1\cos \lambda-C_2\sin
\lambda),\\ \bar C_2 &=& \bar y_x(B_1\sin \lambda+B_2\cos \lambda) +\bar
y_y(C_1\sin \lambda+C_2\cos \lambda). \end{eqnarray*} Now we show that we
can assume that $B_2=C_1=0$.\\ If $B_1C_2-B_2C_1\ne 0$, by choosing $\bar
x$ and $\bar y $ as solutions of the following equations \begin{eqnarray*}
\bar x_x (B_1\sin \lambda+B_2\cos \lambda)+\bar x_y (C_1\sin \lambda+
C_2\cos \lambda) &=& 0, \\ \bar y_x (B_1\cos \lambda-B_2\sin \lambda)+\bar
y_y (C_1\cos \lambda- C_2\sin \lambda) &=& 0, \end{eqnarray*} we have that
$\bar B_2=0=\bar C_1$.\\ If $B_1C_2-B_2C_1=0$. Then for $(B_1,B_2)=(0,0)$,
we choose $\bar x=x,\;\bar y=y$ and $\lambda$ a solution of \[C_1\cos
\lambda+C_2\sin \lambda=0.\] Note that in this case $(C_1,C_2)\ne (0,0)$
otherwise the operators would be connected and this is excluded. Now if
$(B_1,B_2)\ne (0,0)$, we choose $\bar x=x$, $\lambda$ a solution of
\[B_1\sin \lambda+B_1 \cos \lambda=0\] and $\bar y$ a solution of \[\bar y_x
(B_1\cos \lambda-B_2\sin \lambda)+\bar y_y (C_1\cos \lambda- C_2\sin
\lambda) = 0.\] Henceforth we assume that $B_2=C_1=0$. Thus the system
(\ref{ch7:eq1}) reduces to \[\left \{ \begin{array}{lll} B_1A_{2,x}-C_2A_{1,y}
&=& 1+A_1^2+A_2^2,\\ -C_2B_{1,y} &=& A_1B_1, \\ B_1C_{2,y} &=& A_2C_2.
\end{array} \right. \] If $B_1\ne 0$, make the change of variables \[\bar
t=t+\pi/4,\;\;\bar x=x,\;\; \bar y=\bar y (x,y),\] where $\bar y$ is a
solution of $ B_1\bar y_x+C_2\bar y_y=0.$ Hence $\bar C_2=0$. Thus we can
asssume that $B_1=0$ or $C_2=0$. Both cases lead to realizations
equivalent to the following one:
\[X_1=l,\;\; X_2=x(\cos t) l-(1+x^2)(\sin t)
p,\;\;X_3=-x(\sin t) l-(1+x^2)(\cos t) p.\]
Now perform the change of
variables \[\bar t= \tan t,\;\;\bar x= \frac{x}{\cos t},\;\; \bar y=y.\]
Omitting the bars, we obtain the representation
\[X_1=(1+t^2)l+xtp,\;\;X_2=xl-tp,\;\;X_3=-xtl-(1+x^2)p.\] Note that the
same realization was obtained in Mahomed and Leach (1989), where the
authors were looking for representations in $(1+1)$--dimensional space.
The results obtained so far are summarised in Table 7.1.  

\subsection{Realizations of four--dimensional real Lie algebras}

According to the classification of Mubarakzyanov (1963), there are
$30$ real four--dimensional Lie algebras. As far as representation (and to
some extent odes) is concerned, some of these algebras (namely those
depending on  parameters) may be treated simultaneously. All
four--dimensional  real Lie algebras contain three--dimensional
subalgebras. Since we already have represented real three--dimensional Lie
algebras in the previous subsection, we can build representations of
four--dimensional Lie algebras on them. We shall not present details of the
calculations for all the cases. We shall explicitly work out
representations for few cases and all the representations  will be
summarised in  Table 7.2.

\T{{\bf $\LL_{4,1}$: the four--dimensional abelian Lie algebra.}}

The algebra $\LL_{4,1}$ contains $\LL_{3,1}$. Since permuting $X_1,\;X_2,\;X_3$
and $X_4$ does not  affect the structure of $\LL_{4,1}$, the following
cases are to be distinguished:

{\bf (i) } $X_1=l,\;\;X_2=p,\;\;X_3=q.$

Let $X_4=a(t,x,y)l+b(t,x,y)p+c(t,x,y)q$. Then  $[X_i,X_4]=0,\;i=1,2,3$
imply that $a=\mbox{const.},\;b=\mbox{const.},\;c=\mbox{const}$. This
condradicts the fact that $X_1,\;X_2,\;X_3$ and $X_4$  form a basis of
$\LL_{4,1}$.

{\bf (ii)} $X_1=p,\;\;X_2=q,\;\;X_3=f(t)p+g(t)q$.

The relations $[X_1,X_4]=0$ and $[X_2,X_4]=0$ imply that $a=a(t),\;b=b(t),\;c=c(t)$. Now
$[X_3,X_4]=0$ implies that $a(t)f'(t)=0,\;\;a(t)g'(t)=0$. If $a\ne 0 $
then $f=\mbox{const.}$ and $g=\mbox{const}$. However, this would imply that 
$X_1,\; X_2$ and $X_3$ are linearly dependent. Thus $a=0$. Whence the
realization \[X_1=p,\;\;X_2=q,\;\;X_3=f(t)p+g(t)q,\;\;X_4=b(t)p+c(t)q,\]
where $f'(t)c'(t)-g'(t)b'(t)\ne 0$ ensures that $X_1,\;X_2,\;X_3,\;X_4$
are linearly independent.

{\bf (iii)} $X_1=p,\;\;X_2=tp,\;\;X_3=yp.$

$[X_i,X_4]=0,\;i=1,2,3$ imply that $a=0,\;b=(t,y),\;c=0$. Hence the
realization \[X_1=p,\;\;X_2=tp,\;\;X_3=yp,\;\;X_4=b(t,y)p,\] where $b\ne
\mbox{const.}t+\mbox{const.}y+\mbox{const.}$

See Chapter 6 for another approach.

\T{{\bf ${\cal L }_{4,2}$: $[X_1,X_2]=X_2.$}}

The algebra $\LL_{4,2}$ contains $\LL_{3,1}$ with basis
$\left \{X_1,\;X_3,\;X_4 \right \}$ or $\left \{X_2,\;X_3,\;X_4 \right
\}$. Hence we must consider the following cases.

{\bf (i)} $X_1=l,\;\;X_3=p,\;\;X_4=q.$

Let $X_2=a(t,x,y)l+b(t,x,y)p+c(t,x,y)q$. Then
$[X_2,X_3]=0=[X_2,X_4]$ and $[X_1,X_2]=X_2$ imply that $
a=a_0\E^t,\;b=b_0\E^t,\; c=c_0\E^t$, where $a_0,\;b_0,\;c_0$ are
constants. Now perform the transformation
\[ \bar t=
\E^t,\;\;\bar x =x,\;\; \bar y=y.\]
Dropping the bars, we obtain
\[X_1=tl,\;\;X_2=a_0t^2l+b_0t+c_0tq,\;\;X_3=p,\;\;X_4=q.\] This
realization may be further simplified  by considering the cases
$(b_0,c_0)=(0,0)$ and $(b_0,c_0)\ne (0,0)$.

{\bf (ii)} $X_1=p,\;\;X_3=q,\;\;X_4=f(t)p+g(t)q.$

Here $[X_2,X_3]=0 $  and
$[X_1,X_2]=X_2$ imply that $ a=a(t)\E^x,\;b=b(t)\E^x,\;c=c(t)\E^x$. Also
$[X_2,X_4]=0$ gives rise to \[\left \{ \begin{array}{lll} a(t)f(t) & =
& 0, \\ a(t)f'(t)-f(t)b(t) & = & 0,\\ a(t)g'(t)-f(t)c(t) & = & 0.
\end{array} \right.\] If $a=0$ then $f=0$. If $a\ne 0$ then $f=0$ and
$g=\mbox{const}$. But this is inconsistent with the fact that $X_3$ and
$X_4$ are independent. Hence
\[X_1=p,\;\;X_2=b(t)\E^xp+c(t)\E^xq,\;\;X_3=q,\;\; X_4=g(t)q,\] where
$g'(t)\ne 0$ since $X_3$ and $X_4$ must be linearly independent. Now,
perform the transformation \[\bar t= g(t),\;\;\bar x=x, \;\;\bar
y=y.\] Thus we obtain the realization \[X_1=p,\;\;X_2=\eta (t)\E^xp+\mu
(t) \E^xq,\;\; X_3=q,\;\;X_4=tq.\]
{\bf (iii)}
$X_1=p,\;\;X_2=tp,\;\;X_4=yp.$

This case leads to an inconsistency.

{\bf (iv)} $X_2=l,\;\;X_3=p,\;\;X_4=q.$

Let $X_4=a(t,x,y)l+b(t,x,y)p+c(t,x,y)q$. Then $[X_1,X_3]=0=[X_1,X_4]$ imply
that $a=a(t),\;b=b(t),\;c=c(t)$. Now $[X_1,X_2]=X_2$ implies
$a=-t+a_0,\;b=b_0,\; c=c_0$. Replacing $X_1$ by
$X_1-a_0X_2-b_0X_3-c_0X_4$, we obtain the realization
\[X_1=-tl,\;\;X_2=l,\;\;X_3=p,\;\;X_4=q.\] {\bf (v)}
$X_2=p,\;\;X_3=q,\;\;X_4=f(t)p+g(t)q.$\\ Now $[X_1,X_3]=0$ and $[X_1,X_2]$
imply that $ a=a(t),\;b=-x+b(t),\;c=c(t)$. The relation $[X_1,X_4]=0$
constrains
$a,\;f,\;g$ by \[\left \{ \begin{array}{lll} a(t)f'(t)+f(t) & = & 0,\\
a(t)g'(t) & =& 0. \end{array} \right. \] If $a=0$ then $f=0$. Make the
change of variables \[\bar t=t,\;\;\bar x =-x+b(t),\;\; \bar y=y.\]
Omitting the bars, we obtain the realization \[X_1=-xp+\mu
(t)q,\;\;X_2=p,\;\;X_3=q,\;\;X_4=tq.\] If $a\ne 0$, a suitable change of
variables ($\bar t=\int dt/a,\;\bar x=x,\; \bar y =y$) leads to $a=1$.
Hence $f=f_0\E^{-t},\;\;g=g_0$, where $f_0\ne 0$. Perform the change 
\[ \bar t= \E^{-t},\;\;\bar x=x+\alpha (t),\;\; \bar y=y+\beta
(t),\] where  $\alpha$ satisfies $\alpha '(t)+\alpha(t)+b(t)=0$ and
$\beta$ satisfies $\beta '(t)+c(t)=0$. Then replace $X_4$ by
$(X_4-g_0X_3)/f_0$. Dropping the bars, we find
\[X_1=-tl-xp,\;\;X_2=p,\;\;X_3=q,\;\;X_4=tp.\]
{\bf (vi)}
$X_2=p,\;\;X_3=tp,\;\;X_4=yp.$

We have that $[X_1,X_3]=0=[X_1,X_4]$ and
$[X_1,X_2]=X_2$ imply  $a=-t,\;\;b=-x+b(t,y),\;\;c=-y$. Now, invoke the
change of variables \[\bar t=t,\;\;\bar x=x+\alpha (t,y),\;\;\bar y=y,\]
where  $\alpha$ satisfies $-t\alpha_t-y\alpha_y+\alpha+b=0$. Dropping the
bars, we obtain the realization
\[X_1=-tl-xp-yq,\;\;X_2=p,\;\;X_3=tp,\;\;X_4=yp.\]
\T{{\bf $\LL_{4,3}$:
$[X_1,X_2]=X_2,\;\;[X_3,X_4]=X_4.$}}

Permuting the pairs $(X_1,X_2)$ and
$(X_3,X_4)$ do not affect the structure of $\LL_{4,3}$. Therefore the
following cases arise:

{\bf (i)} $X_1=-yq,\;\;X_2=q.$

Let $X_3=a(t,x,y)l+b(t,x,y)p+ c(t,x,y)q$. Then $[X_2,X_3]=0 $ and $[X_1,X_3]$
imply that $X_1=a(t,x)l+b(t,x)p$. Similarly, $X_4=\xi (t,x)l+\eta (t,x) p
$. Hence $X_3$ and $X_4$ depend only on $t$ and $x$. Since
$[X_3,X_4]=X_4$, we deduce that $X_3=-xp $ and $X_4=p$ or $X_3=-tl-xp$ and
$X_4=p$. Whence the representations \[
X_1=-yq,\;\;X_3=q,\;\;X_3=-xp,\;\;X_4=p\] and
\[X_1=-yq,\;\;X_2=q,\;\;X_3=-tl-xp,\;\;X_4=p.\]
{\bf (ii)} $X_1=-xp-yq,\;\;X_2=q.$

Here $[X_1,X_3]=0$ and $[X_2,X_3]=0$ imply that $
a=a(t),\;b=b(t)x,\;\;c=c(t)x$. Hence $X_3=a(t)l+b(t)xp+c(t)xq$. Similarly,
$X_4=\xi (t)l+\eta (t)xp+\mu (t)xq$ and  $[X_3,X_4]=X_4$ imply \[\left \{
\begin{array}{lll} a\xi '-a'\xi & = & \xi,\\ a\eta '-a'\xi & = & \eta, \\
a\mu '+b\mu  -\xi c'-\eta c & = & \mu . \end{array} \right. \] If $ a=0$
then $\xi=0\;\mbox{and}\;\eta =0$. Thus  $X_3=xp+c(t)xq$, $X_4=\mu (t)xq$.
Make the transformation \[\bar t= t,\;\; \bar x= \mu (t)x,\;\;\bar y
=y-c(t)x.\]
We find
\[X_1=-xp-yq,\;\;X_2=q,\;\;X_3=xp,\;\;X_4=xq.\] If $a\ne 0$, then 
make the change of variables \[\bar t =\int dt/a,\;\;\bar x = \alpha
(t)x,\;\;\bar y=y+\beta (t)x,\] where  $\alpha$ satisfies $ a\alpha
'+b\alpha=0$ and $\beta$ the equation $a\beta+b\beta+c=0$. We may assume that
$a=1$ and $b=0=c$. Hence $\xi=\xi_0 \E^t,\;\;
\eta=\eta_0\E^t,\;\;\mu=\mu_0\E^t$. Whence the realization
\[X_1=-xp-yq,\;\; X_2=q,\;\;X_3=l,\;\;X_4=\E^t(\xi_0 l+\eta_0 xp+\mu_0
xq),\] where $\xi_0,\;\eta_0,\;\mu_0$ are constants. This  realization can
be further simplified  by considering the cases $(\eta_0,\mu_0)=(0,0)$ and
$(\eta_0,\mu_0)\ne (0,0)$.

We proceed the same way as above for the remaining four--dimensional
Lie algebras. The results are summarised in Table 7.2.

\section{Canonical forms and reduction of systems of two second--order
odes having four--dimensional symmetry Lie algebras}

The Lie algorithm for calculating 
the symmetry vectors of a given differential equation is well--known
and is described in Chapter 1.
Assume now that the symmetry vectors (of some
unknown equation) are given without recourse to that equation. Can one recover the equation?  This  
question can be understood as a sort of inverse problem in symmetry 
analysis and is sometimes referred to as {\em group--theoretic modelling}.

In this section we aim at classifying systems of two second--order odes
admitting exactly or maximally four--dimensional symmetry Lie algebras. We
also discuss the integrability of such systems. 

\subsection{Canonical forms} 

Here we present the details
of calculations for one case only. The other cases can be dealt with
in a similar manner. Let us begin with the following theorem which was proved in
Chapter 6.

\begin{theo}
\label{ch7:th1}
\begin{em}
A system of two second--order odes is linearizable via a
point transformation if and only if it admits $\LL_{4,1}$ or
$\LL_{4,15}^{1,1}$.
\end{em}
\end{theo}
Consider the algebra $\LL_{4,2}^1$ : \[
X_1=-t\frac{\partial}{\partial t}- x\frac{\partial}{\partial x}
-y\frac{\partial}{\partial y} , \; X_2= \frac{\partial}{\partial x},\;
X_3=t\frac{\partial}{\partial x},\;  X_4=y\frac{\partial}{\partial x}\;\cdot \]
Let us also investigate a system admitting $\LL_{4,2}^1$: \begin{equation}
\label{ch7:c1} \left \{ \begin{array}{lll} \ddot x &= & f(t,x,y,\dot x,\dot
y),\\ \ddot y &= & g(t,x,y,\dot x,\dot y). \end{array} \right.
\end{equation} Following the Lie algorithm, (\ref{ch7:c1})
is invariant under $X_2$ if and only if \[X_2^{[2]}(\ddot
x-f)|_{(\ref{ch7:c1})}=0,\; X_2^{[2]}(\ddot y-g)|_{(\ref{ch7:c1})}=0, \] i.e.
\[f_x=0,\;g_x=0.\] Hence $f=f(t,y,\dot x,\dot y),\; g=g(t,y,\dot x,\dot
y)$ and the system (\ref{ch7:c1}) becomes

\begin{equation}
\label{ch7:c2}
\left \{ \begin{array}{lll}
\ddot x &= & f(t,y,\dot x,\dot y),\\
\ddot y &= & g(t,y,\dot x,\dot y).
\end{array} \right.
\end{equation}

Eqs. (\ref{ch7:c2}) are invariant under
$X_3$ if and only if
\[X_3^{[2]}(\ddot x-f)|_{(\ref{ch7:c2})}=0,\; 
X_3^{[2]}(\ddot y-g)|_{(\ref{ch7:c2})}=0 \]
and
\[f_{\dot x}=0,\;g_{\dot x}=0.\]

Hence $f=f(t,y,\dot y),\;g=g(t,y,\dot y)$ and the system (\ref{ch7:c2})
reduces to
\begin{equation} \label{ch7:c3} \left \{ \begin{array}{lll} \ddot x &= &
f(t,y,\dot y),\\ \ddot y &= & g(t,y,\dot y). \end{array} \right.
\end{equation} Similarly, invariance under $X_4$ leads to  $g=0$ and we
obtain the system \begin{equation} \label{ch7:c4} \left \{ \begin{array}{lll}
\ddot x &= & f(t,y,\dot y),\\ \ddot y &= & 0. \end{array} \right.
\end{equation} Finally, invariance under $X_1$ requires that
\[f=-f_t-yf_y,\] i.e.,  $f=t^{-1}F(y/t,\dot y)$. Thus the system which
admits $\LL_{4,2}^1$ is \begin{equation} \label{ch7:c5} \left \{
\begin{array}{lll} \ddot x &= &t^{-1} f(y/t,\dot y),\\ \ddot y &= & 0.
\end{array} \right. \end{equation} Other cases are treated in a like manner
and
the results are listed in Table 7.3. Note that certain  algebras
or realizations do not appear in this table simply because either they
are not admitted by any equation ($\LL_{4,10}$ for instance) or the
equations which admit them have more than four symmetries ($\LL_{4,19}^8$,
$\LL_{4,20}^{0,1}$ for example).

\subsection{Integrability}

{\bf \em  (a) First Approach}

Assume that we want to integrate or reduce a system of two second--order odes
admitting maximally a four--dimensional symmetry Lie algebra.
How does the knowledge of the symmetries help us in the solution of
this problem?
If we assume that the four--dimensional Lie algebra in question is
solvable, then we can try successive reduction as in the case of scalar
equations. But immediately we face a major problem: a vector field in
three variables has a basis of first--order invariants formed by four elements.
Hence in performing reduction of order there will be ambiguity in
the choice of the  new variables as invariants.
Indeed there are four possibilities. This fact emphasizes a difference
between scalar and systems of odes. In oder to avoid the situation we have
mentioned, we  may proceed
as follows. First we reduce the system to one of the canonical forms given in 
Table 7.3. If we can solve the system in its canonical form, 
we proceed backwards to recover the solution of the initial system. Note that
the transformation bringing the system to its canonical
form is just the transformation which maps its symmetry Lie algebra to one
of the realizations listed in Table 7.2.  Thus the problem of 
integrating systems admitting four-dimensional Lie algebras is reduced to
that of integrating canonical forms. By analyzing the canonical forms
obtained, the following result  can be stated.
\begin{prop}
\label{ch7:p1}
\begin{em}
If a system of two odes admits maximally  a four--dimensional symmetry
Lie algebra, then this algebra  has one or two functionally independant
first--order differential invariants. Furthermore, the underlying equation,
in terms of the invariants, can either be integrated by quadratures or its
integration depends on that of a first--order scalar ode.
\end{em}
\end{prop}
As illustration, we deal with two examples. 
Consider for instance the system
\begin{equation}
\label{ch7:z1}
\ddot x=\dot x^2f(\dot x/\dot y),\;\ddot y=\dot x^2g(\dot x/\dot y)
\end{equation}
which admits $\LL_{4,2}^2$ with basis
\[X_1=-tl,\;X_2=l,\;X_3=p,\;X_4=q.\]
Note that $\LL_{3,1}^1=<X_2,X_3,X_4>$ is a subalgebra of $\LL_{4,2}^2$ 
. The first--order differential invariants of
$\LL_{3,1}^1$ are 
\[u=\dot x,\; v=\dot y.\]
In the variables $t,\;u,\;v$, the system becomes
\begin{equation}
\label{ch7:z2}
\dot u=u^2f(u/v),\;\dot v=u^2g(u/v).
\end{equation}
This system inherits the symmetries
\[X_1=-t\frac{\partial}{\partial t}+u\frac{\partial}{\partial u } +
v\frac{\partial}{\partial v},\;\; X_2=\frac{\partial}{\partial t}.\]
Note also that
\[\frac{du}{dv}=\frac{f(u/v)}{g(u/v)}\]
and $u/v$ is an invariant of $X_1$. This suggests the change of variable
$w=u/v$:
\[\frac{dw}{dv}=\frac{1}{v}\left (\frac{f(w)}{g(w)}-w \right ).\]
Finally, we have reduced the initial system to 
\begin{equation}
\label{ch7:z3}
\left \{ \begin{array}{lll}
\displaystyle{\frac{dv}{dt}} &=&v^2w^2g(w)\\
\displaystyle{\frac{dw}{dv}} &= &
\displaystyle{\frac{1}{v}\left (\frac{f(w)}{g(w)}-w \right )}.
\end{array} \right.
\end{equation}
System (\ref{ch7:z3}) is obviously solvable by quadratures. This example
should  not mislead one into thinking that any system admitting a four--dimensional
Lie algebra is solvable by quadratures. Let us next investigate a case
where the integration of the system depends on that of a first--order scalar
ode. Consider the system
\begin{equation}
\label{ch7:z4}
\ddot x=xf(t,\dot x/x),\;\ddot y=x\dot yf(t,\dot x/x)
\end{equation}
which admits $\LL_{4,3}^2$.
In (\ref{ch7:z4}a), perform the change of variable
\[u=\dot x/x.\]
It becomes
\begin{equation}
\label{ch7:z5}
\dot u=f(t,u)-u^2.
\end{equation}
The integration of (\ref{ch7:z4}) obviously depends  on that of
(\ref{ch7:z5}).

{\bf Remarks.} Few important remarks are now in order.

(i) For functions occuring in Table 7.3, the omitted arguments are just the
previous ones.

(ii) The canonical  forms in Table 7.3 are not necessarily the simplest ones.
There can be changes of variables that cast them into simpler forms.

(iii) The realization $\LL_{4,2}^7$ gives rise to Riccati equations solvable
by quadratures.

(iv) Some systems appearing in Table 7.3 are uncoupled. This  fact
gives insight into uncoupling systems of two second--order odes possessing
four--dimensional Lie algebras.

{\bf \em  (a) Second Approach}

In the first approach, we pointed out the failure of the traditional successive
reduction of order when it is applied to systems. Here we use a
constructive theorem due to Einsenhart (1933) to overcome this problem.

\begin{defi}
\begin{em}
A vector field in $n$ variables $X=\xi^i (x)\partial/\partial x^i$ is a
symmetry of a  first--order linear homogeneous partial differential equation
(pde) $Af\equiv a^i (x)\partial f/\partial x^i =0$ if there exists
$\lambda (x)$ such that $[X,A]=\lambda (x)A$.
\end{em}
\end{defi}

\begin{theo}[Eisenhart 1933]
\label{ch7:eins33}
\begin{em}
If a first--order linear homogeneous pde in $n$ variables $Af=0$
admits a solvable symmetry algebra
$L_{n-1}$ for which the generators and $A$ are unconnected, the integration
of the equation reduces to quadratures.
\end{em}
\end{theo}

{\bf Proof.}  Let $X_1,\;X_2,\ldots,\; X_{n-1}$ be the symbols of $L_{n-1}$,
chosen  such that the first $r$ symbols for any $r$ from $1$ up to at most
$n-2$ generate a subalgebra which is an ideal $L_r$ of the Lie algebra
$L_{r+1}$ spanned by the
first $r+1$ symbols. This choice is possible because $L_{n-1}$ is
solvable (see Chapter 1). Now, if $Af=0$ is the equation, then
\begin{equation}
\label{ch7:pr1}
Af=0,\quad X_1f=0,\ldots,\quad X_{n-2}f=0
\end{equation}
form a complete system which admits $X_{n-1}$. Thus, if $u$ is a solution of
(\ref{ch7:pr1}) different from a constant, so is $X_{n-1}u$. If $X_{n-1}u$ were
zero, $X_1,\ldots, X_{n-1}$ and $A$ would be connected. Hence
$X_{n-1}u=\varphi (u)$, since any two solutions of (\ref{ch7:pr1}) are functions
of one another (for, $\mbox{rank} [A,\; X_1,\ldots, \;X_{n-2}]=n-1$). If we
put $u_1=\int du/\varphi (u)$, we find that $u_1$ is a solution of
(\ref{ch7:pr1}) and that $X_{n-1}u_1=1$. This last equation and (\ref{ch7:pr1}), in
which $f$ is replaced by $u_1$, is compatible ($[X_a,X_b]u_1=0,\;\;
[X_a,A]u_1=0,\;\;a,b=1,\ldots,\;n-1$). Thus it can be solved for
the derivatives of $u_1$. Therefore, $u_1$ is  obtainable by a single
quadrature.

If we use the coordinates $x^{'1}=u_1,\;x^{'2}=x^2,\ldots,\;x^{'n}=x^n$, then
in the new coordinate system, the equations (omitting the primes)
\begin{equation}
\label{ch7:pr2}
Af=0,\quad,X_1f=0,\ldots,X_{n-3}f=0
\end{equation}
do not involve derivatives with respect to $x^1$. Hence, we may treat these
equations as a set in $n-1$ variables, with $x^1$ entering possibly as  a
parameter. Since $X_1,\ldots,\;X_{n-3}$ are the symbols of the ideal
$L_{n-3}$ of $L_{n-2}$, we may apply the same process to find  $u_2$,
which is independent of $u_1$ previouly found since it involves variables
other than $x^1$. The $x^1$ turns out to be $u_1$ in the present coordinate
system.
By repeating this process, we obtain by $n-1$ quadratures, $n-1$ independent
solutions of $Af=0$, that is, the complete solution.

{\bf Remark.} Note that in the proof of Theorem \ref{ch7:eins33}, symmetries are
used to reduce the number of independent variables.

\begin{coro}
\label{ch7:int}
\begin{em}
A system of $n$ second--order odes $\ddot x^i=f^i (t,x,\dot x),\;i=1,\ldots,n$,
which admits a $2n$--dimensional solvable symmetry algebra $L_{2n}$
for which the first prologation of its symbols and $A=\partial /\partial t+
\dot x^i\partial/\partial x^i+
f^i (t,x,\dot x)\partial /\partial \dot x^i$ are unconnected, is solvable by
quadratures.
\end{em}
\end{coro}

{\bf Proof.} Write the system of second--order odes as a first--order linear
homogeneous pde and use Theorem \ref{ch7:eins33}.

{\bf Remarks.} In Corollary \ref{ch7:int}, the unconnectedness is redundant when
$n=1$. Indeed, in this case we obtain  Lie's classical result. However, for
$n \ge 2 $ the unconnectedness is vital: as we have seen above,
although $\LL_{4,3}^2$ is solvable, the resulting system is not integrable by
quadratures. When the hypotheses of Corollary \ref{ch7:int} are satisfied, the
second approach is preferable since it does not require the use
of canonical variables. Notwithstanding, the second approach can be used to integrate
canonical forms for which the integration might not be straightforward. From
this standpoint, the two approaches are complementary. Finally, in the
second approach, symmetries are used to reduce the number of dependent
variables  since when the system of second--order odes is cast as a
first--order linear homogeneous pde, its dependent variables become independent
variables of the resulting pde (see also the  remark  after the proof
of Theorem \ref{ch7:eins33}).

\section{Conclusion}
We have obtained nonsimilar realizations of three-- and four--dimensional
real Lie algebras in $(1+2)$--dimensional space. This has been applied to
the classification of all systems of two second--order odes admitting
four--dimensional symmetry Lie algebras. Moreover, we have shown via
examples how this classification can be used for integrating systems of
two second--order odes admitting four--dimensional symmetry Lie algebras.
Note that the classification of systems with $5$ symmetries can be
achieved by combining our results with the following theorem.
\begin{theo}[Egorov's theorem, see Petrov 1966]
\label{ch7:th2}
\begin{em}
Every five--dimensional real
Lie algebra contains a four--dimensional real subalgebra.
\end{em}
\end{theo}
A question that needs attention is the study of symmetry breaking for
systems of  two second--order odes: it is well--known (see
Mahomed and Leach 1989)
that a scalar second--order ode can admit $0,\;1,\;2,\;3$ or $8$
point symmetries.
A system of two second-order linear odes admits $5,\;6,\;7,\;8$ or $15$
point symmetries (Wafo Soh and Mahomed 1999e, see Chapter 5).
We conjecture that a system of two second--order odes can admit
$0,\;1,\;2,\;3,\;4,\;5,\;6,\;7,\;8$ or $15$ point symmetries. This conjecture
is motivated by the results obtained in Chapter 5. Indeed nonlinear des are
in general less symmetric than the linear ones.
\input{table1}
\input{table1b}
\input{table2}
\input{table3}
\input{table4}
\input{table5}
\input{table6}

%\end{document}














