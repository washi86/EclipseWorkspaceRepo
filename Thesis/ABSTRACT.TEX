%\documentstyle[12pt]{report}
%\usepackage{lscape}
%\setlength{\parindent}{0mm}
%\setlength{\parskip}{6pt}
%\renewcommand{\baselinestretch}{1.5}
%\setlength{\topmargin}{0pt}
%\setlength{\headheight}{0pt}
%\setlength{\headsep}{0pt}
%\setlength{\footheight}{0pt}
%\setlength{\footskip}{45pt}
%\setlength{\textwidth}{430pt}
%\setlength{\textheight}{660pt}
%\setlength{\oddsidemargin}{10pt}
%\begin{document}

\section*{\begin{center} ABSTRACT \end{center}}  

We begin by classifying scalar ordinary differential equations (odes)
admitting
irreducible contact symmetry algebras. The following facts emerge
from this classification: second--order odes admit infinite--dimensional
contact symmetry
algebras. Third--order odes reducible to  the simplest equation $q^{(3)}=0$
are the only
third--order odes admitting irreducible contact symmetry algebras. More
importantly, except for third--order linear odes which have four-- and
five--dimensional Lie point symmetry algebras, the contact symmetry algebras
of third--order odes are subalgebras of the ten--dimensional contact symmetry
algebra of the equation $q^{(3)}=0$. Fourth--order odes are devoid
of nontrivial contact symmetries and nontrivial equations with irreducible
contact symmetry algebras occur from order $n \ge 5$.

We also advocate the use of Noether symmetries as an alternative approach
to the Lie approach for the integration of a class of equations.
Indeed the invariance of the Noether
integral under the Noether point symmetry with which it is associated provides an
elegant integration procedure.
In order to illustrate how useful this invariance property can be, we
classify Noether
point symmetries of $y''=f(x)y^n$ with  respect to its natural Lagrangian.
The cases $n=-1/3$ and $n=2$ which play an important role in General
Relativity in the context of spherically symmetric perfect fluid solutions
are studied in detail. For $n=-1/3$, we obtain, in a unified way, all
existing solutions and for $n=2$ we recover known solutions and discover new
solutions given in parametric form.

Furthermore, using symmetry methods, we generalize the celebrated
Abel--Forsyth formula. This generalization is applied to projective Riccati equations. As
a result, we establish that if $n$ linearly independent solutions of the
$n$--dimensional projective Riccati equation are known, its general solution
can be obtained by quadratures. This extends the well--known result of
scalar Riccati equations.

In addition, we provide a symmetry approach to the method of variation of
parameters for systems. Our approach weakens the traditional requirements. 

By introducing a novel canonical form for systems of two second--order linear
odes, we prove that such systems can have $5$--, $6$--, $7$--, $8$-- or
$15$--dimensional Lie point symmetry algebras. This clearly shows that they
are not equivalent to each other. Also, this fact prompts the study of
linearizability  for systems of second--order odes.

We prove that a system of two second--order odes is linearizable via point
transformations if and only if it admits the four--dimensional abelian Lie
algebra or the four--dimensional Lie algebra with commutators
$[X_i,X_j]=0,\;[X_i,X_4]=X_i,\;i,\;j=1,\;2,\;3$. This theorem is then
extended to systems of $n> 2$ second--order odes.

Finally, we obtain all nonsimilar realizations of three-- and
four--dimensional real Lie algebras in terms of vector fields in
$(1+2)$--dimensional space. This  is applied to the classification and
integration of
systems of two second--order odes admitting four--dimensional point symmetry
algebras. To effect integration, we give two routes as the traditional
approach is ambiguous for systems.

%\end{document}
