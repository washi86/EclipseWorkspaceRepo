%\documentstyle[12pt]{report}
%\setlength{\parindent}{0mm}
%\setlength{\parskip}{14pt}
%\renewcommand{\baselinestretch}{1.5}
%\setlength{\topmargin}{0pt}
%\setlength{\headheight}{0pt}
%\setlength{\headsep}{0pt}
%\setlength{\footskip}{45pt}
%\setlength{\footheight}{0pt}
%\setlength{\textwidth}{430pt}
%\setlength{\textheight}{660pt}
%\setlength{\oddsidemargin}{10pt}
%\newcommand{\RR}{\mathrm{I\!R\!}}
%\newtheorem{defi}{Definition}[chapter]
%\newtheorem{theo}{Theorem}[chapter]
%\begin{document}

\chapter{ Linearization of Systems of Second--order
Nonlinear Ordinary Differential Equations}

Firstly, we prove two linearization criteria for systems of
two second--order ordinary
differential equations (odes). The first one says that a system of two 
second--order odes is reducible via a point transformation to a linear system
if and only if it admits the four--dimensional abelian Lie algebra 
$L_{4,1}$. The second one states that for a system of
two second--order odes to be linearizable, it is necessary and sufficient that
it admits the four--dimensional  Lie algebra
$L_{4,2}$ with commutators
$[X_i,X_j]=0,\;[X_i,X_4]=X_i,\;i,\;j=1,\;2,\;3.$ The approach used is
contructive and enables us to explicitly work out
the transformation leading to linearization.
Secondly, we give necessary and sufficient conditions under which
a system of two second--order linear odes is reducible to the
free particle equation $x''=0,\;y''=0$. 
The linearization criteria  are then generalized to a system of $n$
($n > 2$) second--order odes. Finally, we give examples of how one can
effect linearization for a system. The results contained here also appear
in Wafo Soh and Mahomed (1999f).
\section{Introduction}
Systems of two second--order odes have enjoyed much interest over the years.
In particular, the newtonian system
\begin{equation}
\label{ch6:h}
x''=-\frac{\partial V}{\partial x},\quad
y''=-\frac{\partial V}{\partial y},
\end{equation}
where the potential function $V$ depends on the position $(x,y)$, have been
investigated for Lie point symmetry groups in Sen (1987). It was shown in
Sen (1987) that (\ref{ch6:h}) can admit from $1$ to $7$ or $15$ Lie point
symmetries. The symmetry classification of Sen (1987) has recently been
rediscovered by Damianou and Sophocleous (1999). The group classification
of arbitrary systems of second--order odes is a vast open problem.
In particular, the investigation of symmetry properties of arbitrary systems
of two second--order odes will yield a natural generalization of the above
cited work. A partial solution to this problem will be presented
in Chapter 7.

In applications, linear equations often occur in disguised forms.
Quite often  it is only after a point transformation that the linear 
structure of a nonlinear differential equation is uncovered. Therefore, it is  
important to have workable criteria for linearization of differential 
equations via point transformations. This problem was tackled 
completely for the case of
scalar second--order odes by Lie (1883) and his student Tresse (1894, 1896).
They mainly used the fact that all scalar linear second--order odes
are reducible by
means of point transformations to the free particle equation $y^{''}=0$. 
A similar argument does not  apply to scalar linear higher order ($n\ge 3$)
odes. For, according to recent work by Mahomed and Leach (1990)
(see also Krause and Michel 1988), a scalar
linear higher order ode ($n\ge 3$) can belong to one of three 
classes depending upon whether it has $n+1$, $n+2$ or the maximum number
$n+4$ of point symmetries. These authors also studied the linearization 
problem for scalar higher order ($n\ge3$) odes. Their methods exploit both
the symmetry structure and the Laguerre--Forsyth canonical form of scalar
linear odes. In this chapter, we follow almost the same approach in the study 
of linearizability of systems of two second--order odes. It is worth noting
that symmetry properties of scalar second--order odes substantially
differ from that of
systems of two second--order odes: according to Chapter 5, a system of
two second--order linear  odes can have $5,\;6,\;7,\;8$ or
$15$ point symmetries. The maximum number of symmetries being achieved  by the
two--dimensional free particle equation, viz. $x''=0,\;y''=0$.
This clearly shows that systems of two linear second--order odes are not
equivalent to each other.

The outline of this chapter is as follows. In Section 6.2 we
discuss realizations of the abelian four--dimensional algebra $L_{4,1}$ in
$(1+2)$--dimensional space and the linearizability of systems of two
second--order odes. Section 6.3 deals with representations of the
four--dimensional Lie
algebra $L_{4,2}$ with commutators $[X_i,X_j]=0,\;[X_i,X_4]=X_i,
\;i,j=1,2,3$, and  the linearizability of systems 
of two second--order odes. Section 6.4  extends the results of
Section 6.2 and 6.3  to
systems of $n$ ($n > 2$) second--order odes.
Finally in Section 6.5 we give a few applications
of the linearization criteria obtained in the previous sections.

\section{Abelian structure and linearization}
We begin with some definitions that will be useful in the sequel.
\begin{defi}
\begin{em}
Let
\[X_i=\xi_{ij}\frac{\partial}{\partial x_j},\;\;i=1,\ldots,p.\]
{\bf 1.} The  operators $X_1,\ldots,X_p$ are {\em connected} if there
are  functions $\lambda_1 (x),\ldots,\lambda_p (x)$  such that
\[\lambda_1 (x) X_1+\cdots+\lambda_p (x)X_p=0\]
and $(\lambda_1 (x),\ldots,\lambda_p (x))\ne 0.$
They are {\em unconnected}  otherwise.

{\bf 2.} The operators $X_1,\ldots, X_p $ are
{\em linearly dependent} if there  are constants $c_1,\ldots,c_p$ such that
\[c_1X_1+\cdots+c_pX_p=0\]
and $(c_1,\ldots,c_p)\ne 0$.  They are {\em  linearly independent}  
otherwise.
\end{em}
\end{defi}
If  $r^{*}=\mbox{rank }[\xi_{ij}]$, it is obvious that at most $r^{*}$ 
of the operators $X_1,\ldots,X_p$ are unconnected.

{\bf Example. } Consider the operators
\[X_1=\frac{\partial}{\partial x},\;\;\;X_2=x\frac{\partial}{\partial x}
\;\cdot\]
The operators $X_1$ and $X_2$ are linearly independent but they are
connected as $X_2=xX_1$.

Given the four--dimensional abelian Lie algebra $L_{4,1}$, we look for its  
realizations in the space of vector fields in $\RR^{\;3}$ with coordinates
$(t,x,y)$. Let
$\left \{ X_1,\;X_2,\;X_3,\;X_4 \right \}$ be a basis of $L_{4,1}$ and
$r^{*}=\mbox{rank }[X_1,X_2,X_3,X_4]$. It is clear that $1\le r^{*} \le 3$. 
Hence the cases: 

{\bf (i)} $r^{*}=3.$ 

This means that at most three of the operators $X_1,\;X_2,\;X_3,\;X_4$ are
unconnected. Since renaming these operators does not affect the structure
of the algebra, we may assume without loss of generality that 
$X_1,\;X_2,\;X_3$ are unconnected. Therefore there is a change of variables 
$T=T(t,x,y),\;X=X(t,x,y),\;Y(t,x,y)$ in which
\begin{eqnarray*}
X_1 &=&\frac{\partial}{\partial T},\;\; X_2=\frac{\partial}{\partial X},\;\;
X_3=\frac{\partial}{\partial Y},\\
X_4 &=&\xi(T,X,Y)\frac{\partial}{\partial T}+
\eta (T,X,Y)\frac{\partial}{\partial X}+\mu (T,X,Y)\frac{\partial}{\partial Y}
\;\cdot
\end{eqnarray*}
Invoking the abelian structure, we obtain that $\xi,\;\eta,\;\mu$  are
constants. This contradicts the fact that $X_1,\;X_2,\;X_3,\;X_4$ are
linearly independent. Hence there is no transitive representations of
the four--dimensional abelian algebra $L_{4,1}$ in a three--dimensional space.

{\bf (ii)} $r^{*}=2.$

Without loss of generality, we may assume that $X_1$ and $X_2$ are 
unconnected. Thus there is a change of variables $T=T(t,x,y),\;X=(t,x,y),\;   
Y=Y(t,x,y)$ such that
\begin{eqnarray*}
X_1 &=& \frac{\partial }{\partial T},\;\;\;\;
X_2=\frac{\partial}{\partial X},\\
X_3 & =& \xi (T,X,Y)X_1+\eta (T,X,Y)X_2=
\xi (T,X,Y) \frac{\partial}{\partial T}+\eta (T,X,Y)\frac{\partial}
{\partial X}, \\
X_4 &=& a(T,X,Y)X_1+b(T,X,Y)X_2=
a(T,X,Y) \frac{\partial}{\partial T}+b (T,X,Y)\frac{\partial}
{\partial X}\,\cdot 
\end{eqnarray*}
Since the commutators of these operators must vanish, we have that
$\xi=\xi(Y),\;\eta=\eta (Y),\;a=a(Y),\;b=b(Y)$.
Since $X_1,\; X_2,\;X_3,\; X_4$ must be
linearly independent, we have the additional condition
\[\left | \begin{array}{cc}
\xi^{'}(Y) & a'(Y)\\
\eta^{'}(Y) & b'(Y) 
\end{array}\right |\ne 0.\]
Renaming the variables and the operators and using lower case variables, 
we obtain the realization
\begin{equation}
\label{ch6:eq6}
X_1=\frac{\partial}{\partial x},\;\; X_2=\frac{\partial}{\partial y},\;\;
X_3=\xi (t) \frac{\partial }{\partial x}+\eta (t)\frac{\partial}{\partial y}
,\;\; X_4=a(t) \frac{\partial }{\partial x}+b(t)\frac{\partial}{\partial y},
\end{equation}
where $\xi^{'}(t)b'(t)-a'(t)\eta^{'}(t)\ne 0$.

{\bf (iii)} $r^{*}=1.$

There is a change of variables $T=T(t,x,y),\;X=X(t,x,y),\;Y=Y(t,x,y)$
in which
\[X_1=\frac{\partial}{\partial T},\;\;X_2=f(T,X,Y)\frac{\partial}{\partial T}
,\;\;X_3=g(T,X,Y)\frac{\partial}{\partial T},\;\;
X_4=h(T,X,Y)\frac{\partial}{\partial T},\]
where the set of functions $\{1,f,g,h\}$ is linearly independent. 
Since the commutators of these oparators must vanish, we have that
$f=f(X,Y),\;g=g(X,Y),\;h=h(X,Y)$.
Now perform the change of coordinates
\[\bar T=T,\;\;\bar X=f(X,Y),\;\;\bar Y=g(X,Y).\]
In the new coordinates, the operators read
\[X_1=\frac{\partial}{\partial \bar T},\;\;
X_2=\bar X \frac{\partial}{\partial \bar T}
,\;\;X_3=\bar Y\frac{\partial}{\partial \bar T},\;\;
X_4=\bar h(\bar X,\bar Y)\frac{\partial}{\partial \bar T},\]
where the set of functions $\{1, \bar X, \bar Y, \bar h\}$ is linearly 
independent. Thus at least one second derivative of $\bar h$ does not
vanish. Dropping the bars, renaming the variables and using lower case 
variables, we obtain the realization
\begin{equation}
X_1=\frac{\partial}{\partial x},\;\; X_2=t\frac{\partial}{\partial x},\;\;
X_3=y\frac{\partial}{\partial x},\;\;X_4=h(t,y)\frac{\partial}{\partial x},
\label{ch6:eq7}
\end{equation}
where  at least one second derivative of $h$ does not vanish.  

Hence we have proved the following theorem:
\begin{theo} 
\label{ch6:theo1}
\begin{em}
In (1+2)--dimensional space, the abelian Lie algebra $L_{4,1}$ has two classes
of realizations\\
\begin{equation}
\mbox{{\bf(i)} } X_1=\frac{\partial}{\partial x},\;\; 
X_2=\frac{\partial}{\partial y},\;\;
X_3=f(t)\frac{\partial }{\partial x}+g(t)\frac{\partial}{\partial y},\;\;
X_4=h(t)\frac{\partial }{\partial x}+k(t)\frac{\partial}{\partial y},
\label{ch6:eq8}
\end{equation}
where
\[\left | \begin{array}{cc}
f'(t) & h'(t) \\
g'(t) &  k'(t) 
\end{array} \right |\ne 0,\]
\begin{equation}
\mbox{{\bf (ii)} }X_1=\frac{\partial}{\partial x},\;\;
X_2=t\frac{\partial}{\partial x},\;\;X_3=y\frac{\partial}{\partial x},\;\;
X_4=f(t,y)\frac{\partial}{\partial x},
\label{ch6:eq9}
\end{equation}
where $f(t,y)\ne c_1t+c_2y+c_3$ and $c_1,\;c_2,\;c_3$ are constants.
\end{em}
\end{theo}
An application of this theorem is given in the following result.
\begin{theo}
\label{ch6:theo2}
\begin{em}
A system of two second--order odes is linearizable by means of a point
transformation if and only if it admits the abelian Lie algebra $L_{4,1}$.
\end{em}
\end{theo}
{\bf Proof.} The necessity is straightfoward since any linearizable 
system of two second--order odes  put in the first canonical form
(cf. Chapter 5, Theorem 5.1) admits $L_{4,1}$. For the proof of sufficiency,
we need
to show that  if a system admits a realization of $L_{4,1}$ then this system 
must be linear. Consider a system admitting the first realization
(\ref{ch6:eq8})
\begin{equation}
\label{ch6:e1}
\left \{ \begin{array}{ccc} x'' &=&E(t,x,y,x',y'),\\ y''&=&F(t,x,y,x',y'). 
\end{array} \right.
\end{equation}
The invariance of (\ref{ch6:e1}) under $X_1$ is equivalent to (see Chapter 1)
\[X_1^{[2]}(x''-E)|_{(\ref{ch6:e1})}=0,\;\;X_1^{[2]}(y''-F)|_{(\ref{ch6:e1})}=0,\]
i.e.
\[\frac{\partial E}{\partial x}=0,\;\;\frac{\partial F}{\partial x}=0,\]
which give
\[E=E_1(t,y,x',y'),\;\;F=F_1(t,y,x',y').\]
Thus
\begin{equation}
\label{ch6:e2}
\left \{ \begin{array}{ccc} x'' &=&E_1(t,y,x',y'),\\ y''&=&F_1(t,y,x',y'). 
\end{array} \right.
\end{equation}
The invariance of (\ref{ch6:e2}) under $X_2$ implies
\[X_2^{[2]}(x''-E_1)|_{(\ref{ch6:e2})}=0,\;\;X_2^{[2]}(y''-F_1)|_{(\ref{ch6:e2})}=0,\]
i.e.
\[\frac{\partial E_1}{\partial y}=0,\;\;\frac{\partial F_1}{\partial y}=0,\] 
which result in
\[E_1=E_2(t,x',y'),\;\;F_1=F_2(t,x',y').\]
Hence
\begin{equation}
\label{ch6:e3}
\left \{ \begin{array}{ccc} x'' &=&E_2(t,x',y'), \\ y''&=&F_2(t,x',y'). 
\end{array} \right.
\end{equation}
The invariance of (\ref{ch6:e3}) under $X_3$ is equivalent to
\[X_3^{[2]}(x''-E_2)|_{(\ref{ch6:e3})}=0,\;\;X_3^{[2]}(y''-F_2)|_{(\ref{ch6:e3})}=0,\]
where
\[X_3^{[2]}=X_3+f'\frac{\partial}{\partial x'}+g'\frac{\partial}{\partial y'}
+f''\frac{\partial}{\partial x''}+g''\frac{\partial}{\partial y''}\;\cdot\]
The invariance of (\ref{ch6:e3}) under $X_3$ then yields
\[f'\frac{\partial E_2}{\partial x'}+g'\frac{\partial E_2}{\partial y'}=f'',\;\;
f'\frac{\partial F_2}{\partial x'}+g'\frac{\partial F_2}{\partial y'}=g''.\]
Since $f'k'-g'h'\ne 0$, we may assume without loss of generality that 
$f'\ne 0.$  Whence
\[E_2=\frac{f''}{f'}x'+\frac{1}{f'}E_3(t,g'x'-f'y'),\;\;
F_2=\frac{g''}{f'}x'+\frac{1}{f'}F_3(t,g'x'-f'y')\]
and
\begin{equation}
\label{ch6:e4}
\left \{ \begin{array}{ccc}
x'' & = & \displaystyle{\frac{f''}{f'}x'+\frac{1}{f'}E_3(t,g'x'-f'y')},\\
y'' & = & \displaystyle{\frac{g''}{f'}x'+\frac{1}{f'}F_3(t,g'x'-f'y')}.
\end{array}\right.
\end{equation}
The invariance of (\ref{ch6:e4}) under $X_4$ is equivalent to 
\[
X_4^{[2]}(x''-\frac{f''}{f'}x'-\frac{1}{f'}E_3)|_{(\ref{ch6:e4})}=0,\;\;
X_4^{[2]}(y''-\frac{g''}{f'}x'-\frac{1}{f'}F_3)|_{(\ref{ch6:e4})}=0,\]
where
\[X_4^{[2]}=X_4+h'\frac{\partial}{\partial x'}+k'\frac{\partial}{\partial y'}
+h''\frac{\partial}{\partial x''}+k''\frac{\partial}{\partial y''}\;\cdot\]
Thus
\[E_3=\frac{h''f'-f''h'}{g'h'-f'k'}(g'x'-f'y')+f'A(t),\]
\[F_3=\frac{k''f'-g''h'}{g'h'-f'k'}(g'x'-f'y')+f'B(t),\]
where $A(t)$ and $B(t)$ are arbitrary functions. Finally we obtain
the system
\[ \left \{  \begin{array}{ccc}
x^{''}&=& \frac{\left |\begin{array}{cc} f''& h''\\g'& k'\end{array}\right |}
{\left |\begin{array}{cc} f'& h'\\ g'& k'\end{array}\right |}x'+
\frac{\left |\begin{array}{cc} f''& h''\\ f'& h'\end{array}\right |}
{\left |\begin{array}{cc} f'& h'\\ g'& k'\end{array}\right |}y'+ A(t),\\
 & & \\
y^{''} &=& \frac{\left |\begin{array}{cc} g''& k''\\g'& k'\end{array}\right |}
{\left |\begin{array}{cc} f'& h'\\ g'& k'\end{array}\right |}x'+
\frac{\left |\begin{array}{cc} g''& k''\\ f'& h'\end{array}\right |}
{\left |\begin{array}{cc} f'& h'\\ g'& k'\end{array}\right |}y'+ B(t), 
\end{array} \right.\]
where $A(t)$ and $B(t)$ are arbitrary functions. It can be  easily checked
that $(f,g)$ and $(h,k)$ are two particular solutions of the 
homogeneous system  associated with the above system.
Using the same method as above, we can prove that there is no system of 
two second--order odes admitting the second realization. This completes
the proof of the theorem.

{\bf Remark. }  If a system admits a four--dimensional Abelian Lie
algebra, the transformation which reduces it to its linear form
is just the change of coordinates which brings this Lie algebra 
to the realization (\ref{ch6:eq8}).  This change of variables 
is obtained by solving first--order linear partial differential equations only.
\section{Representations of the Lie algebra $L_{4,2}$ 
and \\
linearization}
We consider the Lie algebra  $L_{4,2}$ with commutators
\begin{equation}
[X_i,X_j]=0,\;\;\;[X_i,X_4]=X_i,\;\;i,j=1,2,3. \label{ch6:eq10}
\end{equation}
In Patera and Winternitz (1977), this algebra 
corresponds to $A_{4,5}^{1,1}$.
We aim at constructing realizations of this algebra in (1+2)--dimensional
space. We proceed as in the preceding section. Let
$r^{*}=\mbox{rank }[X_1,X_2,X_3,X_4]$. Clearly $1\le r^{*}\le 3$. Whence
the following cases emerge.

{\bf (i)} $r^{*}=3.$

This implies that at most three of the operators $X_1,\ldots,X_4$ are
unconnected. Since renaming the operators $X_1,X_2,X_3$ does not affect
the structure of the Lie algebra, the following two subcases should be
considered:

{\bf (a)} $X_1,\;X_2,\;X_3$ are unconnected,

{\bf (b)} $X_1,\;X_2,\;X_4$ are unconnected and $X_1,\;X_2,\;X_3$ are
connected.

The second subcase leads to an inconsistency (which is that $X_1,\ldots ,X_4$
are linearly dependent). In the first subcase, there is a change of variables
$T=T(t,x,y),\;\;X=X(t,x,y),\;\;Y=Y(t,x,y)$ in which the operators have
the forms
\begin{eqnarray*}
X_1 & = &\frac{\partial}{\partial T},\;\;X_2=\frac{\partial}{\partial X},\;\; 
X_3=\frac{\partial}{\partial Y},\\
X_4 & = & \xi (T,X,Y)\frac{\partial}{\partial T}+\eta (T,X,Y)\frac{\partial}
{\partial X}+\mu (T,X,Y)\frac{\partial}{\partial Y}\,\cdot
\end{eqnarray*}
The commutation relations $[X_i,X_4]=X_i,\;i=1,2,3$, imply
\[\xi=T+C_1,\;\;\eta=X+C_2,\;\;\mu=Y+C_3.\]
Hence
\[X_4=(T+C_1)\frac{\partial}{\partial T}+ (X+C_2)\frac{\partial}{\partial X}
+(Y+C_3)\frac{\partial}{\partial Y}\;\cdot\]
Now perform the change of coordinates
\[\bar T=T+C_1,\;\;\bar X=X+C_2,\;\;\bar Y=Y+C_3.\]
The operators become
\[
X_1 =\frac{\partial}{\partial \bar T},\;\;X_2=\frac{\partial}{\partial \bar X},\;\; 
X_3=\frac{\partial}{\partial \bar Y},\;\;
X_4 = \bar T \frac{\partial}{\partial \bar T}+\bar X \frac{\partial}
{\partial \bar X}+\bar Y\frac{\partial}{\partial \bar Y}\;\cdot
\]
Dropping the bars and using lower case variables we obtain the realization
\begin{equation}
\label{ch6:eq11'}
X_1 = \frac{\partial}{\partial t},\;\;X_2=\frac{\partial}{\partial x},\;\; 
X_3=\frac{\partial}{\partial y},\;\;
X_4 = t\frac{\partial}{\partial t}+ x\frac{\partial}
{\partial x}+y\frac{\partial}{\partial y}\;\cdot
\end{equation}
{\bf (ii)} $r^{*}=2.$

This means at most two of the operators $X_1,\ldots,X_4$ are unconnected. Since
renaming the operators $X_1,\;X_2,\;X_3$ does not affect the structure of
the algebra, the following subcases should be considered:

{\bf (a)} $X_1,\;X_2$  are unconnected,

{\bf (b)} $X_1,\;X_4$ are unconnected and two arbitrary  operators of
$X_1,\;X_2,\;X_3$ are connected.

The second subcase leads to an inconsistency (which is  $r^{*}=1$). In the
first subcase, there is a change of variables $T=T(t,x,y),\;X=X(t,x,y),\;
Y=Y(t,x,y)$ in which the operators become
\begin{eqnarray*}
X_1 &=& \frac{\partial}{\partial T},\;\; X_2=\frac{\partial}{\partial X},\;\;
X_3=\xi (T,X,Y)\frac{\partial}{\partial T}+\eta (T,X,Y)\frac{\partial}
{\partial X},\\
X_4 &= & a(T,X,Y)\frac{\partial}{\partial T}+b(T,X,Y)\frac{\partial}
{\partial X}\;\cdot
\end{eqnarray*}
The commutation relations $[X_1,X_3]=0$ and $[X_2,X_3]=0$ imply that $\xi=\xi (Y)$
and $\eta =\eta (Y)$.
From the equations $[X_i,X_4]=X_i,\;\;i=1,\;2,\;3$, we deduce that
\[a=T+\alpha (Y) \mbox{ and } b=X+\beta (Y).\]
Now make the change
\[\bar T=T+\alpha (Y),\;\;\bar X=X+\beta (Y),\;\; \bar Y=Y.\]
In the new variables, the operators become
\[X_1=\frac{\partial}{\partial \bar T},\;\;X_2=\frac{\partial}{\partial \bar X},
\;\;X_3=\xi (\bar Y)\frac{\partial}{\partial \bar T}+\eta (\bar Y)
\frac{\partial}{\partial \bar X},\;\;X_4=\bar T\frac{\partial}{\partial \bar T}
+\bar X\frac{\partial}{\partial \bar X}\,\cdot\]
Omitting the bars, renaming the variables and using lower case variables, we  
obtain the representation
\begin{equation}
\label{ch6:eq11}
X_1 = \frac{\partial}{\partial x},\;\;X_2=\frac{\partial}{\partial y},\;\; 
X_3=\xi (t) \frac{\partial}{\partial x}+\eta (t)\frac{\partial}{\partial y},\;\;
X_4 = x\frac{\partial}{\partial x}+ y\frac{\partial}
{\partial y},
\end{equation}
where $(\xi^{'}(t),\eta^{'}(t))\ne 0$ to ensure that the operators are
linearly independent.

{\bf (iii)} $r^{*}=1.$

In this case, the four operators are connected. Thus there is
a change of variables $T=T(t,x,y),\;X=X(t,,y),\;Y=Y(t,x,y)$ in which the
operators become
\[X_1=\frac{\partial}{\partial T},
\;\;X_i=\xi_{i}(T,X,Y)\frac{\partial}{\partial T},\;i=2,\;3,\;4,\]
where the set of functions $\{1,\xi_i\}$ is linearly independent to ensure
the independency of the operators. Now $[X_1,X_2]=0,\;[X_2,X_3]=0$ imply
that $\xi_2=\xi_2(X,Y),\;\;\xi_3=\xi_3(X,Y)$. Futhermore, $[X_i,X_4]=X_i,\; 
i=1,2,3$ imply  $\xi_4=T+a(X,Y)$. Make the change of variables
\[\bar T=T+a(X,Y),\;\bar X=\xi_2(X,Y),\; \bar Y=\xi_3 (X,Y).\]
The invertibility of this changes of variables is guaranteed by the fact that
the set of functions $\{1,\xi_2,\xi_3\}$ is linearly independent. In the new
coordinates, the operators read
\[X_1=\frac{\partial}{\partial \bar T},\;\;
X_2=\bar X \frac{\partial}{\partial \bar T},\;\;
X_3=\bar Y \frac{\partial}{\partial \bar T},\;\;
X_4=\bar T\frac{\partial}{\partial \bar T}\;\cdot\]
Leaving out the bars, renaming the variables and using lower case variables, we
obtain the realization
\begin{equation}
\label{ch6:eq13}
X_1=\frac{\partial}{\partial  x},\;\;
X_2=t \frac{\partial}{\partial x},\;\;
X_3=y \frac{\partial}{\partial x},\;\;
X_4=x \frac{\partial}{\partial x}\;\cdot
\end{equation}
Finally we have proved the following theorem.
\begin{theo}
\label{ch6:theo3}
\begin{em}
In (1+2)--dimensional space, the Lie algebra $L_{4,2}$ has three classes
of repsesentations:
\begin{eqnarray}
\mbox{{\bf (i)}  } & & X_1 = \frac{\partial}{\partial t},\;\;X_2=\frac{\partial}{\partial x},\;\; 
X_3=\frac{\partial}{\partial y},\;\;
X_4 = t\frac{\partial}{\partial t}+ x\frac{\partial}
{\partial x}+y\frac{\partial}{\partial y}, \label{ch6:eq14}\\
\mbox {{\bf (ii)}  } & & X_1 = \frac{\partial}{\partial x},\;
X_2=\frac{\partial}{\partial y},\; 
X_3=f(t)\frac{\partial}{\partial x}+g(t)\frac{\partial}{\partial y},\;
X_4 = x\frac{\partial}{\partial x}+y\frac{\partial}{\partial y}, \label{ch6:eq15}\\
 & & \mbox{where $(f'(t),g'(t))\ne 0$,}\nonumber\\
\mbox{{\bf (iii)}  } & & X_1=\frac{\partial}{\partial  x},\;\;
X_2=t \frac{\partial}{\partial x},\;\;
X_3=y \frac{\partial}{\partial x},\;\;
X_4=x \frac{\partial}{\partial x}\;\cdot \label{ch6:eq16}
\end{eqnarray}
\end{em}
\end{theo}
An immediate application of this theorem is the following one.
\begin{theo}
\label{ch6:theo4}
\begin{em}
A system of two second--order odes is linearizable via a point transformation
if and only if it admits $L_{4,2}$.
\end{em}
\end{theo}
{\bf Proof.} The necessity is straightfoward since any linearizable 
system of two second--order odes  put in the first canonical 
form (cf. Chapter 5, Theorem 5.1) admits $L_{4,2}$. For the proof of sufficiency, we need
to show that realizations given in Theorem \ref{ch6:theo3} yield linear equations.
The first and third realizations give the two--dimensional free particle
equation $x''=0,\;\;y''=0$. The second one gives the linear system of
second--order odes
\[\left \{ \begin{array}{cc}
x''= &\left (\displaystyle{\frac{f''}{f'}+\frac{g'}{f'}A(t)} \right ) x'-A(t)y',\\
  & \\
y''= &\left (\displaystyle{\frac{g''}{f'}+\frac{g'}{f'}B(t)} \right ) x'-B(t)y',
\end{array} \right.\]
where $A(t)$ and $B(t)$ are two arbitrary functions and where we have assumed
without loss of generality that $f'\ne 0$. Note that  $(f,g)$ 
is a particular solution of the above system.

{\bf Remark. }  If a system admits a four--dimensional Lie
algebra with the commutators $[X_i,X_j]=0,\;\;[X_i,X_4]=X_i,\;\;i,j=1,2,3$, 
the transformation which reduces it to its linear form
is just the change of coordinates which brings this Lie algebra to one of 
the realizations (\ref{ch6:eq14}), (\ref{ch6:eq15}), (\ref{ch6:eq16}). 
This change of variables is obtained by solving first--order 
linear partial differential equations only.

The proof of Theorem \ref{ch6:theo4} gives rise to the following one.
\begin{theo}
\label{ch6:theo5}
\begin{em}
A system of two second--order odes is reducible to the free particle equation 
\[x''=0,\;\;y''=0\] if and only if it admits a four--dimensional Lie algebra
with commutators 
\[[X_i,X_j]=0,\;\;[X_i,X_4]=X_i,\;\;i,j=1,2,3\] 
and
\[\mbox{rank }[X_1,X_2,X_3,X_4]\in \{1,3\}.\]
\end{em}
\end{theo}
\section{Linearization of systems of $n>2$ second-order odes}
Here we extend Theorems \ref{ch6:theo2} and \ref{ch6:theo4} to systems of
$n > 2$ second--order odes.
\begin{theo}
\label{ch6:theo6}
\begin{em}
A system of $n> 2$ second--order odes is reducible by means of a point 
transformation to a linear system if and only if its symmetry Lie algebra
contains a $2n$--dimensional subalgebra which is similar to one of the
following. 

(i) $L_{2n,1}:\;\displaystyle{X_i=\frac{\partial}{\partial x_i},\;
X_{n+i}=f_{ij}(t)\frac{\partial}{\partial x_j}},\;i=1,\ldots ,n,$ \\
where $\mbox{det}[f_{ij}'(t)]\ne 0.$ 

(ii) $L_{2n,2}:\;\displaystyle{X_i=\frac{\partial}{\partial x_i},\;
X_{n+j}=f_{jk}(t)\frac{\partial}{\partial x_j},\;i=1,\ldots ,n,\;
j=1,\ldots,n-1;\; X_{2n}=x_k\frac{\partial}{\partial x_k}},$ \\
where $\mbox{rank}[f_{jk}'(t)]=n-1.$ 
\end{em}
\end{theo}
{\bf Proof.}

(i) The necessity follows from the fact  that any linearizable system 
put in the
first canonical form (cf. Chapter 5, Theorem 5.1) admits $L_{2n,1}$. For the sufficiency, if we
impose the operators of $L_{2n,1}$ on a system of $n$ second--order odes, we
obtain
\[ x_{k}''=-\frac{ \left |
\begin{array}{llll}
f_{11}' & \ldots & f_{1n}' & f_{1k}''\\
\vdots &         & \vdots  & \vdots \\    
f_{n1}' & \ldots  & f_{nn}' & f_{nk}''\\
x_1'   & \ldots  & x_n'    &  0  
\end{array} \right |}{|f_{ij}'|} +A_k(t),\]
where $k=1,\dots ,n$ and the $A_k$s are arbitrary functions.

(ii) The proof here is similar to (i).

\section{Applications}
It is quite clear from the preceding theorems and discussions that the
Henon--Heiles and Kepler problems are not linearizable via point
transformations. This is also true for a large class of newtonian systems
investigated in the literature (see, e.g., Hietarinta 1987 and references
therein).
It is interesting to note that none of the nonlinear cases in Sen (1987) and
Damianou and Sophocleous (1999) are linearizable via point transformations.
This means that the forces need to be time and/or velocity dependent for
linearization to be possible. In the following examples of systems of
second--order odes, we illustrate the foregoing theorems by using velocity
dependence for the forces.

{\bf (a) } Consider the system
\begin{equation}
\label{ch6:eq17}
\left \{ \begin{array}{ccc}
x'' & = & y'^{2},\\
y'' & = & 0.
\end{array} \right.
\end{equation}
Using Lie's  algorithm (Chapter 1), we can obtain all the symmetries
of (\ref{ch6:eq17}). We consider operators
\begin{equation}
\label{ch6:eq18}
X_1=\frac{\partial}{\partial x},\;\;X_2=\frac{\partial}{\partial t},
\;\;X_3=y\frac{\partial}{\partial x},\;\;X_4=y\frac{\partial}{\partial t}
\end{equation}
which  satisfy
\[ [X_i,X_j]=0,\;\;i,j=1,\;2,\;3,\;4.\]
Hence we infer from Theorem \ref{ch6:theo2} that (\ref{ch6:eq17}) is linearizable via
a point transformation. In order to find this point transformation, we must
look for a change of variables which maps (\ref{ch6:eq18}) to the realization
(\ref{ch6:eq8}). An example of such change of variables is provided by
\[ T=y,\;\;X=x,\;\;,Y=t.\]
In this new variables,
\begin{equation}
\label{ch6:eq19}
X_1=\frac{\partial}{\partial X},\;\;X_2=\frac{\partial}{\partial Y},
\;\;X_3=T\frac{\partial}{\partial X},\;\;X_4=T\frac{\partial}{\partial Y},
\end{equation}
and
\begin{equation}
\label{ch6:eq20}
\left \{ \begin{array}{ccc}
\displaystyle{\frac{d^2X}{dT^2}} & = & 1,\\
& & \\
\displaystyle{\frac{d^2Y}{dT^2}} & = & 0.
\end{array} \right.
\end{equation}
{\bf (b)} Consider the system
\begin{equation}
\label{ch6:eq21}
\left \{ \begin{array}{ccc}
x'' &=& x'^{2}+y'^{2},\\
y'' &=& 2x'y'.
\end{array} \right.
\end{equation}
From Lie's algorithm  described in Chapter 1, we can obtain all 
the symmetries of (\ref{ch6:eq21}). We consider
\begin{equation}
\label{ch6:eq22}
X_1=\frac{\partial}{\partial t},\;\;X_2=\mbox{e}^{-(x+y)}\frac{\partial}
{\partial t},\;\;X_3=\mbox{e}^{-(x-y)}\frac{\partial}{\partial t},
\;\;X_4=t\frac{\partial}{\partial t}\cdot
\end{equation}
It can easily be checked that
\[[X_i,X_j]=0,\;\;[X_i,X_4]=X_i,\;\;i,j=1,2,3\]
and
\[\mbox{rank }[X_1,X_2,X_3,X_4]=1.\]
Thus according to Theorem \ref{ch6:theo4} and Theorem \ref{ch6:theo5}, Equation 
(\ref{ch6:eq21}) is linearizable and reducible to the free particle equation
via  a point transformation. Since $r^{*}=1$, this point transformation
must map (\ref{ch6:eq22}) to the realization (\ref{ch6:eq16}). Therefore it
satisfies the system:
\begin{eqnarray*}
\frac{\partial T}{\partial t} &=&0,\;\;
\frac{\partial X}{\partial t} =1,\;\;
\frac{\partial Y}{\partial t} =0,\\
\mbox{e}^{-(x+y)}\frac{\partial T}{\partial t} &=& 0,\;\;
\mbox{e}^{-(x+y)}\frac{\partial X}{\partial t} =T,\;\;
\mbox{e}^{-(x+y)}\frac{\partial Y}{\partial t}= 0,\\
\mbox{e}^{-(x-y)}\frac{\partial T}{\partial t} &= &0,\;\;
\mbox{e}^{-(x-y)}\frac{\partial X}{\partial t} =Y,\;\;
\mbox{e}^{-(x-y)}\frac{\partial Y}{\partial t} =0,\\
t\frac{\partial T}{\partial t} &=&0,\;\;
t\frac{\partial X}{\partial t} =X,\;\;
t\frac{\partial Y}{\partial t} =0.
\end{eqnarray*}
The solution of this system is given by
\[T=\mbox{e}^{-(x+y)},\;\;X=t,\;\;Y=\mbox{e}^{-(x-y)}.\]
In the new coordinate system, (\ref{ch6:eq21}) becomes
\begin{equation}
\label{ch6:eq23}
\left \{ \begin{array}{ccc}
\displaystyle{\frac{d^2X}{dT^2}} & = & 0, \\
& &\\
\displaystyle{\frac{d^2Y}{dT^2}} &= & 0.
\end{array} \right.
\end{equation}
\section{Conclusion}
We have given representations of $L_{4,1}$ and $L_{4,2}$ in 
$(1+2)$--dimensional  space. Applications have been made to 
the linearization of systems of second--order odes. We have
emphasised also the connection between the linearizability of a system of 
second--order odes  and the algebraic structure of its symmetry Lie 
algebra. In addition, we gave a natural extension of our linearization criteria
via symmetry Lie algebras to $n$ ($n > 2$) second--order odes.

%\end{document}

